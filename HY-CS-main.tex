%% Follow comments to support use.

%%%%%%%%%%%%%%%%%%%%%%%%%%%%%%%%%%%%%%%%%%%%%%%%%%%%%%%%%
%% STEP 1: Choose options for MSc / BSc / seminar layout and your bibliographic style
%%%%%%%%%%%%%%%%%%%%%%%%%%%%%%%%%%%%%%%%%%%%%%%%%%%%%%%%%

%%  Language: 
%%      finnish, swedish, or english
%%  Pagination (use twoside by default)  
%%      oneside or twoside,
%%  Study programme / kind of report
%%      csm  = Master's thesis in Computer Science Master's Programme;
%%      tkt = Bachelor's thesis in Computer Science Bachelor's Programme;
%%      seminar = seminar report
%%  For Master's thesis choose your line or track:
%%      (30 cr thesis, 2020 onwards, Master's Programme in Computer Science = csm)
%%      software-track-2020 = Software study track
%%      algorithms-track-2020 = Algorithms study track
%%      networking-track-2020 = Networking study track

\documentclass[english,twoside,censored,csm,algorithms-track-2020]{HYthesisML}


% If wanted, open new chapters only at right page.
% By default, "openany".
%\PassOptionsToClass{openright,twoside,a4paper}{report}
\PassOptionsToClass{openany,twoside,a4paper}{report}

\usepackage{csquotes}
%%%%%%%%%%%%%%%%%%%%%%%%%%%%%%%%%%%%%%%%%%%%%%%%%%%%%%%%%
%% REFERENCES
%% Some notes on bibliography usage and options:
%% natbib -> you can use, e.g., \citep{} or \parencite{} for (Einstein, 1905); with APA \cite -> Einstein, 1905 without ()
%% maxcitenames=2 -> only 2 author names in text citations, if more -> et al. is used
%% maxbibnames=99 as no great need to suppress the biliography list in a thesis
%% for more information see biblatex package documentation, e.g., from https://ctan.org/pkg/biblatex 

%% Reference style: select one 
%% for APA = Harvard style = authoryear -> (Einstein, 1905) use:
\usepackage[style=numeric-comp,backend=biber,natbib=true,maxbibnames=99,giveninits=true,uniquename=init,safeinputenc,sorting=none]{biblatex}
%% for numeric = Vancouver style -> [1] use:
%\usepackage[style=numeric,bibstyle=numeric,backend=biber,natbib=true,maxbibnames=99,giveninits=true,uniquename=init]{biblatex}
%% for alpahbetic -> [Ein05] use:
%\usepackage[style=alphabetic,bibstyle=alphabetic,backend=biber,natbib=true,maxbibnames=99,giveninits=true,uniquename=init]{biblatex}
%

\addbibresource{thesis.bib}
% in case you want the final delimiter between authors & -> (Einstein & Zweistein, 1905) 
% \renewcommand{\finalnamedelim}{ \& }
% List the authors in the Bibilipgraphy as Lastname F, Familyname G,
% \DeclareNameAlias{sortname}{family-given}
% remove the punctuation between author names in Bibliography 
%\renewcommand{\revsdnamepunct}{ }

% Only print URL if no DOI or eprint is given
\DeclareSourcemap{
  \maps[datatype=bibtex]{
    \map{
      \step[fieldsource=doi,final]
      \step[fieldset=url,null]
    }
    \map{
      \step[fieldsource=eprint,final]
      \step[fieldset=url,null]
    }
  }
}


%% Block of definitions for fonts and packages for picture management.
%% In some systems, the figure packages may not be happy together.
%% Choose the ones you need.

%\usepackage[utf8]{inputenc}  % For UTF8 support, in some systems. Use UTF8 when saving your file.


\usepackage{lmodern}         % Font package, again in some systems.
\usepackage{textcomp}        % Package for special symbols
\usepackage[pdftex]{color, graphicx} % For pdf output and jpg/png graphics
\usepackage{epsfig}
\usepackage{subfigure}
\usepackage[pdftex, plainpages=false]{hyperref} % For hyperlinks and pdf metadata
\usepackage{fancyhdr}        % For nicer page headers
\usepackage{tikz}            % For making vector graphics (hard to learn but powerful)
%\usepackage{wrapfig}        % For nice text-wrapping figures (use at own discretion)
\usepackage{amsmath, amssymb} % For better math
\usepackage{booktabs}        % For nicer tables
\usepackage{multirow}        % Multirow cells in tables
\usepackage{siunitx}
\usepackage{rotating}        % For sidewaystable

\usepackage[capitalize,noabbrev,nameinlink]{cleveref} % For easier referencing
\Crefname{ALC@unique}{Line}{Lines}
\newcommand{\crefrangeconjunction}{--}


\usepackage{amsthm}          % Theorems
\usepackage{aliascnt}        % Cleveref with shared number theorems
\newtheorem{theorem}{Theorem}[chapter]
\newaliascnt{lemma}{theorem}
\newtheorem{lemma}[lemma]{Lemma}
\aliascntresetthe{lemma}
\newaliascnt{proposition}{theorem}
\newtheorem{proposition}[proposition]{Proposition}
\aliascntresetthe{proposition}
\newaliascnt{definition}{theorem}
\newtheorem{definition}[definition]{Definition}
\aliascntresetthe{definition}
\newaliascnt{observation}{theorem}
\newtheorem{observation}[observation]{Observation}
\aliascntresetthe{observation}
\theoremstyle{definition}
\newaliascnt{example}{theorem}
\newtheorem{example}[example]{Example}
\aliascntresetthe{example}
\Crefname{observation}{Observation}{Observations}

\usepackage{algorithm}       % For pseudocode
\usepackage[noend]{algorithmic} % For pseudocode
\renewcommand{\algorithmiccomment}[1]{\hspace*{\fill}\{#1\}}

\hyphenation{CaDiCaL} % Don't hyphenate work

% TODO: Remove
\usepackage{lipsum}

\graphicspath{{./plots}}

\singlespacing               %line spacing options; normally use single

\fussy
%\sloppy                      % sloppy and fussy commands can be used to avoid overlong text lines
% if you want to see which lines are too long or have too little stuff, comment out the following lines
% \overfullrule=1mm
% to see more info in the detailed log about under/overfull boxes...
% \showboxbreadth=50 
% \showboxdepth=50



%%%%%%%%%%%%%%%%%%%%%%%%%%%%%%%%%%%%%%%%%%%%%%%%%%%%%%%%%
%% STEP 2:
%%%%%%%%%%%%%%%%%%%%%%%%%%%%%%%%%%%%%%%%%%%%%%%%%%%%%%%%%
%% Set up personal information for the title page and the abstract form.
%% Replace parameters with your information.
\title{MaxSAT-Based Bi-Objective Boolean Optimization}

\author{Christoph Jabs}
\date{\today}

% Set supervisors, use the titles according to the thesis language
% in English Prof. or Dr., or in Finnish toht. or tri or FT, TkT, Ph.D. or in Swedish... 
\supervisors{Prof.~Matti~J\"arvisalo, Dr.~Jeremias~Berg, Dr.~Andreas~Niskanen}

\keywords{Multi-objective optimization, bi-objective optimization, maximum satisfiability, incremental SAT}
\additionalinformation{\translate{\track}}

%% For seminar reports:
%%\additionalinformation{Name of the seminar}

%% Provide classification terms, to appear on the abstract page.
%% Replace the classification terms below with the ones that match your work.
%% ACM Digital library provides a taxonomy and a tool for classification
%% in computer science. Use 1-3 paths, and use right arrows between the
%% about three levels in the path; each path requires a new line.

\classification{\protect{\ \\
\  Mathematics of computing $\rightarrow$ Discrete mathematics $\rightarrow$ Combinatorics $\rightarrow$ Combinatorial optimization \\
\  Theory of computation  $\rightarrow$ Logic $\rightarrow$ Constraint and logic programming
}}

%% If you want to quote someone special. You can comment this line out and there will be nothing on the document.
%\quoting{Bachelor's degrees make pretty good placemats if you get them laminated.}{Jeph Jacques}


%% OPTIONAL STEP: Set up properties and metadata for the pdf file that pdfLaTeX makes.
%% Your name, work title, and keywords are recommended.
\hypersetup{
    unicode=true,           % to show non-Latin characters in Acrobat’s bookmarks
    pdftoolbar=true,        % show Acrobat’s toolbar?
    pdfmenubar=true,        % show Acrobat’s menu?
    pdffitwindow=false,     % window fit to page when opened
    pdfstartview={FitH},    % fits the width of the page to the window
    pdftitle={},            % title
    pdfauthor={Christoph Jabs}, % author
    pdfsubject={},          % subject of the document
    pdfcreator={},          % creator of the document
    pdfproducer={pdfLaTeX}, % producer of the document
    pdfkeywords={algorithms}, % list of keywords for
    pdfnewwindow=true,      % links in new window
    colorlinks=true,        % false: boxed links; true: colored links
    linkcolor=black,        % color of internal links
    citecolor=black,        % color of links to bibliography
    filecolor=magenta,      % color of file links
    urlcolor=cyan           % color of external links
}

%%-----------------------------------------------------------------------------------

\newcommand{\formula}{F}
\newcommand{\Obj}{\textsc{O}}
\newcommand{\inc}{\text{I}}
\newcommand{\dec}{\text{D}}
\newcommand{\var}{\textsc{var}}
\newcommand{\lit}{\textsc{lit}}
\newcommand{\Min}{\texttt{Minimize-\allowbreak{}Inc}}
\newcommand{\Simpr}{\texttt{Solution-\allowbreak{}Improving-\allowbreak{}Search}}
\newcommand{\E}{\texttt{EnumSols}}
\newcommand{\Ex}{\texttt{ExistsSol}}
\newcommand{\T}{\mathtt{T}}
\newcommand{\assumps}{\mathcal{A}}
\newcommand{\satsolver}{\texttt{isSAT}}
\newcommand{\res}{\text{res}}
\newcommand{\algname}{\textsc{BiOptSat}}
\newcommand{\tot}{\textsc{Tot}}
\newcommand{\ov}[2]{\langle #1 < #2 \rangle}
\newcommand{\ove}[2]{\langle #1 \leq #2 \rangle}
\newcommand{\satunsat}{\texttt{SAT-\allowbreak{}UNSAT}}
\newcommand{\unsatsat}{\texttt{UNSAT-\allowbreak{}SAT}}
\newcommand{\msu}{\texttt{MSU3}}
\newcommand{\I}{\mathcal{I}}
\newcommand{\Act}{\texttt{Act}}
\newcommand{\oll}{\texttt{OLL}}
\newcommand{\msh}{\texttt{MSHybrid}}
\newcommand{\hs}{\texttt{hs}}
\newcommand{\nsamp}{n}
\newcommand{\nfeat}{m}
\newcommand{\nclauses}{k}
\newcommand{\selector}{s}
\newcommand{\noise}{\eta}
\newcommand{\equals}{e}
\newcommand{\nelems}{n}
\newcommand{\nsets}{m}
\newcommand{\setcard}{s}
\newcommand{\elemprob}{p}
\newcommand{\sets}{\mathcal{S}}
\newcommand{\element}{e}
\newcommand{\cover}{\mathcal{C}}
\newcommand{\cost}{c}
\newcommand{\cores}{\mathcal{K}}
\newcommand{\core}{\kappa}
\newcommand{\sol}{\tau}
\newcommand{\scep}{SetCovering-EP}
\newcommand{\scsc}{SetCovering-SC}
\newcommand{\assump}{\mathcal{A}}
\newcommand{\clause}{C}
\newcommand{\softs}{\textsc{S}}
\newcommand{\generalobj}{f}
\newcommand{\nobj}{p}
\newcommand{\feasible}{\mathcal{X}}
\newcommand{\soloone}{{\{i_2,\allowbreak d_1,\allowbreak d_3,\allowbreak d_4,\allowbreak \lnot i_1,\allowbreak \lnot i_3,\allowbreak \lnot i_4,\allowbreak \lnot d_2\}}}
\newcommand{\solotwo}{{\{i_1,\allowbreak i_2,\allowbreak d_1,\allowbreak d_2,\allowbreak \lnot i_3,\allowbreak \lnot i_4,\allowbreak \lnot d_3,\allowbreak \lnot d_4\}}}
\newcommand{\solothree}{{\{i_2,\allowbreak d_1,\allowbreak d_3,\allowbreak d_4,\allowbreak \lnot i_1,\allowbreak \lnot i_3,\allowbreak \lnot i_4,\allowbreak \lnot d_2\}}}
\newcommand{\solcone}{{\{i_1,\allowbreak i_2,\allowbreak i_3,\allowbreak i_4,\allowbreak d_1,\allowbreak d_2,\allowbreak d_3,\allowbreak d_4\}}}
\newcommand{\solctwo}{{\{i_1,\allowbreak i_2,\allowbreak d_1,\allowbreak d_2,\allowbreak d_3,\allowbreak d_4,\allowbreak \lnot i_3,\allowbreak \lnot i_4\}}}
\newcommand{\solcthree}{{\{i_2,\allowbreak d_1,\allowbreak d_2,\allowbreak d_3,\allowbreak d_4,\allowbreak \lnot i_1,\allowbreak \lnot i_3,\allowbreak \lnot i_4\}}}
\newcommand{\solcfour}{{\{i_1,\allowbreak i_2,\allowbreak i_3,\allowbreak d_1,\allowbreak d_2,\allowbreak \lnot i_4,\allowbreak \lnot d_3,\allowbreak \lnot d_4\}}}
\newcommand{\solmcstrap}{{\{i_1,\allowbreak i_3,\allowbreak i_4,\allowbreak d_1,\allowbreak d_3,\allowbreak d_4,\allowbreak \lnot i_2,\allowbreak \lnot d_2\}}}
\newcommand{\TODO}[1]{\textcolor{red}{#1}}

\begin{document}

% Generate title page.
\maketitle

%%%%%%%%%%%%%%%%%%%%%%%%%%%%%%%%%%%%%%%%%%%%%%%%%%%%%%%%%
%% STEP 3:
%%%%%%%%%%%%%%%%%%%%%%%%%%%%%%%%%%%%%%%%%%%%%%%%%%%%%%%%%
%% Write your abstract in the separate file, to be positioned here.
%% You can make several abstract pages (if you want it in different languages),
%% in which case you should also define the language of the abstract,
%% as below.

\begin{otherlanguage}{english}
\begin{abstract}
Combinatorial optimization problems arise naturally in many real-world problems.
For this reason, and since these problems are typically \NP-hard, there is a need for non-trivial algorithmic solutions for solving combinatorial optimization.
Many such solutions (e.g., local search, evolutionary algorithms or declarative programming) have been presented for optimization problems with a single objective, however, real-world problems are often inherently bi-objective.

In this work, we present \algname{}, an exact declarative programming approach for finding so-called Pareto-optimal solutions to bi-objective optimization problems.
We use propositional logic as the declarative programming language and seek to extend the progress and success in MaxSAT solving to two objectives.
The presented approach follows the lexicographic method and makes use of a single SAT solver that is preserved throughout the entire search procedure.
\algname{} allows for solving three tasks for bi-objective optimization:
finding a single Pareto-optimal solution, one representative solution for each Pareto point or enumerating all Pareto-optimal solutions.

We provide an open source implementation and evaluate five different variants of \algname{}, all building on different algorithms known from MaxSAT solving.
In the empirical evaluation, we compare \algname{} to three competing algorithmic approaches, showing practical benefits of our approach in the contexts of learning interpretable decision rules and bi-objective set covering.
Furthermore, for the best-performing variant of \algname{}, we study the effects of proposed refinements to determine their effectiveness.
\end{abstract}
\end{otherlanguage}


% Place ToC
%\newpage
\mytableofcontents

\mainmatter

%%%%%%%%%%%%%%%%%%%%%%%%%%%%%%%%%%%%%%%%%%%%%%%%%%%%%%%%%
%% STEP 4: Write the thesis.
%%%%%%%%%%%%%%%%%%%%%%%%%%%%%%%%%%%%%%%%%%%%%%%%%%%%%%%%%
%% Your actual text starts here. You shouldn't mess with the code above the line except
%% to change the parameters. Removing the abstract and ToC commands will mess up stuff.
%%
%% Command \include{file} includes the file of name file.tex.
%% A new page will be created at every \include command, 
%% which makes it appropriate to use it for large entities such as book chapters. Cannot be nested.
%% It is useful for a big project, as changing one of the include targets 
%% won't force the regeneration of the outputs of all the rest.
%% Alternatively, \input is a more lower level macro 
%% which simply inputs the content of the given file like it was copy&pasted there manually.

\chapter{Introduction\label{chap:intro}}

% What are optimization problems and where do they occur
% Start by telling a story (running example throughout intro)
Optimization problems can be summarized as the task of finding a ``best'' solution out of a collection of feasible ones.
For example, when looking for a new flat to buy, most people will be comparing prices with the aim to find the cheapest flat possible that fulfils their requirements.
Commonly, the notion of ``best'' that is used in optimization is that a solution with minimum associated ``cost'' is considered optimal.
Speaking in the example from above, this cost is the price of the flat.
If the collection of possible solutions is discrete (as in this example), we speak of \emph{combinatorial} optimization.

% Reveal conflicting second objective
Looking more closely at the flat search example, we can notice a problem emerging:
what does ``fulfilling'' the requirements mean?
Some requirements, like the number of rooms, might be easy to specify, but consider the distance of ones daily commute.
Rather than setting a fixed threshold as ``maximum $d$ kilometres distance'', what we might actually want to do is minimize this distance at the same time as the cost of the flat.
Now there are two objectives to take into account regarding what constitutes a ``best'' solution.
Two objectives give rise to \emph{bi-objective} optimization, which we study in this work.

% Approaches to solving optimization
Since combinatorial optimization is relevant for real-world problems, a plethora of algorithmic approaches to solving optimization problems have been proposed.
These approaches range from local search~\autocite{DBLP:books/daglib/0017492}, over evolutionary~\autocites{DBLP:books/daglib/0087893,DBLP:journals/jgo/StornP97} to declarative programming-based algorithms~\autocite{handbook2-maxsat,ChenEtAl2010-intro,DBLP:reference/fai/2}.
In addition to these generally applicable approaches, there are problem specific algorithms (e.g.,~\autocite{DBLP:conf/aaai/DemirovicS21,DBLP:conf/kdd/NijssenF07,DBLP:conf/nips/HuRS19}).
One other attribute on which the optimization approaches differ is whether they are exact or approximative.
Given enough resources, an exact algorithm is guaranteed to find \emph{the best} solution, while approximative algorithms will return \emph{a good} solution within given resource constraints.

% MaxSAT (and SAT)
In this work, we present a declarative programming approach for finding exact Pareto-optimal solutions to bi-objective optimization problems.
We use propositional logic as the declarative programming language and build on maximum satisfiability (MaxSAT) solving techniques.
MaxSAT is the \NP-hard single-objective optimization variant of the \NP-complete propositional satisfiability (SAT) problem.
It asks for an assignment to a given propositional formula, maximizing the number of satisfied clauses.
Solvers for MaxSAT have made significant progress over the recent years and are by now very efficient for solving many practical optimization problems.
This progress is mainly due to development on the underlying SAT solvers used in MaxSAT solving~\autocites{DBLP:journals/ai/FroleyksHIJS21,handbook2-cdcl}, but also due to improved algorithms for how to apply those SAT solvers in MaxSAT solving.
Some approaches for SAT- and MaxSAT-based bi--objective optimization have been proposed in the past~\autocites{DBLP:conf/cp/SohBTB17,DBLP:conf/ijcai/Terra-NevesLM18a,DBLP:conf/aaai/Terra-NevesLM18,DBLP:conf/ijcai/Terra-NevesLM18,DBLP:conf/cp/JanotaMSM21}, but it is not a very active field of research.
We seek to extend the progress and success in MaxSAT solving to two objectives.

% Hardness of optimization problems
Combinatorial optimization problems are typically hard to solve.
We informally extend the complexity class \NP{}~\autocite{AroraBarak2009-complexity} to optimization problems.
While $\mathcal{P}$ contains all decision problems for which an algorithm that solves the problem in polynomial time is known, such an algorithm is not known for problems that are in \NP{} but not $\mathcal{P}$.
This means that the worst case runtime for \NP-complete (i.e., exactly as hard as the hardest problems in \NP) and \NP-hard (i.e., at least as hard as problems in \NP) problems is exponential.
Informally, we say an optimization problem is \NP-hard if its corresponding decision problem is \NP-complete.
Consider the \NP-complete set covering decision problem~\autocite{DBLP:conf/coco/Karp72} where for a collection $\sets$ of sets, the task is to determine whether a cover $\cover$ with cardinality smaller than a threshold $k$ exists so that the cover intersects all sets, i.e., $S \cap \cover \neq \emptyset$ for all $S\in\sets$.
The corresponding optimization problem, where the task is to find the \emph{smallest} such cover is \NP-hard.
Several other well known \NP-complete decision problems also have corresponding \NP-hard optimization problems~\autocite{KorteVygen2018-15}.

% Applications of (single- and bi-objective) optimization in literature
\NP-hard single- and bi-objective optimization problems arise naturally in many areas, ranging from industrial applications over research to everyday tasks.
While the single-objective case has been studied in depth for tasks including scheduling~\autocites{DBLP:conf/cp/Stojadinovic14,DBLP:conf/cpaior/BofillGSV15,DBLP:journals/ior/Solomon87,DBLP:journals/candie/AkyolB07}, supply chain optimization~\autocite{DBLP:journals/cce/Papageorgiou09}, air traffic management~\autocites{DBLP:journals/ior/BertsimasLO11,RichardsHow2002Aircrafttrajectoryplanning}, clustering~\autocite{DBLP:journals/ai/DaoDV17,DBLP:conf/sdm/DavidsonRS10} and learning optimal classifiers~\autocites{DBLP:conf/cp/MaliotovM18,DBLP:conf/ijcai/NarodytskaIPM18,DBLP:conf/ijcai/Hu0HH20,DBLP:conf/cp/YuISB20,DBLP:conf/aaai/DemirovicS21,DBLP:conf/cp/ShatiCM21,DBLP:conf/cade/IgnatievPNM18}, the bi-objective setting has not been studied as much.
Some of the fewer examples from literature are the following:
When learning interpretable classifiers~\autocites{DBLP:conf/ijcai/Ignatiev0NS21,DBLP:conf/cp/MaliotovM18,DBLP:conf/ijcai/NarodytskaIPM18,DBLP:conf/ijcai/Hu0HH20,DBLP:conf/cp/YuISB20,DBLP:conf/aaai/Ignatiev0S021,DBLP:conf/cade/IgnatievPNM18}, the objectives ``interpretability'' and ``classification error'' are in conflict because a more complex and therefore less interpretable classifier is typically more accurate.
A bi-objective optimization problem arises when wanting to create a portfolio of solvers that together solve a set of benchmarks as fast as possible while also containing as few solvers as possible~\autocite{DBLP:conf/cp/JanotaMSM21}.
There are also bi-objective optimization problems in network routing with the objectives load balancing and latency~\autocite{SilverioEtAl2022biobjectiveoptimization}.
In supply chain optimization, in addition to the economic objective, environmental aspects can be taken into consideration as a second objective~\autocites{DBLP:journals/cce/Pinto-VarelaBN11,DBLP:journals/candie/TautenhainBN19}.

% Solving pipeline for declarative approaches
The declarative programming approach, that is generally applicable to all of these applications, follows the solving pipeline illustrated in \cref{fig:solving-pipeline}.
First, the problem of interest is \emph{encoded} into a set of constraints formulated in a declarative programming language of choice.
Examples of declarative languages for optimization are MaxSAT~\autocite{handbook2-maxsat}, constraint programming~\autocite{DBLP:reference/fai/2} and mixed integer linear programming~\autocites{ChenEtAl2010-intro,KorteVygen2018-5}.
An encoding is hereby a mapping of each instance of the original problem to a set of constraints in the declarative language, where each optimal solution to the encoded instance corresponds to an optimal solution to the original instance.
Choosing a fitting declarative language for solving a given problem is a non-trivial problem and doing so should take considerations such as how succinctly a language can represent a problem into account.
Secondly, a general optimization solver for the declarative language is used to solve the encoded problem and obtain on optimal solution for it.
A solver for a declarative programming language is an algorithm that finds an optimal solution to an instance in said language.
The last step is to map the found solution back to the original problem space.

\begin{figure}
  \centering
  \includegraphics{solving-pipeline.pdf}
  \caption{The solving pipeline of the declarative approach to optimization.}\label{fig:solving-pipeline}
\end{figure}

% Advantages of declarative approaches
When solving an optimization problem with the declarative approach, the task that needs to be solved is \emph{not} coming up with a new algorithm but only with an encoding of the problem into constraints in the used language.
However, this general applicability is not the only advantage.
Improvement in solver performance for a declarative language also immediately maps to better solving performance for \emph{all} problems that can be encoded in said language.

% Runtimes in declarative approaches
In the scope of this work, we focus on using the declarative approach for solving \NP-hard optimization problems.
For this application, \NP-hard declarative languages are used.
Given an existing encoding for the problem, this means that the generic solving step is the only computationally hard step in the solving pipeline.
Both the encoding and the reconstruction of the solution are typically done in polynomial time.
Since the declarative language is \NP-hard, the worst-case runtime of the solving algorithm cannot be better than exponential (assuming $\mathcal{P}\neq\mathcal{NP}$).
However, in practice one can observe significantly better performance from many solving algorithms for ``interesting'' instances, i.e., instances that appear for real-world problems.
Exemplary solving algorithms for \NP-hard declarative languages that achieve good performance on real-world instances are MaxSAT solving algorithms~\autocite{handbook2-maxsat} based on conflict driven clause learning solvers for propositional satisfiability~\autocite{handbook2-cdcl}, and state-of-the-art branch-and-cut algorithms for mixed integer linear programming~\autocite{ChenEtAl2010-branch-and-cut}.

% Conflicting objectives and why there might be no single optimal solution
A crucial difference between single-objective and bi-objective optimization is that there is no single notion of optimality for two or more objectives.
Whereas for a single objective function, there is a clear minimum (or maximum) and objective values can be unambiguously compared, this becomes less defined for the bi-objective case:
coming back to the flat search example, consider a flat with a cost of 300\,000 \texteuro{} and 1-kilometre daily commute and compare it to another flat that costs 240\,000 \texteuro{} and has a 3-kilometre daily commute.
It is not immediately clear which one of these options is better, and the choice would depend on ones personal preference over the two objectives.
This becomes especially difficult if there is no clear preference over the objectives.
Typically, a situation like that occurs when two of the objectives considered are in conflict, as the price of a flat and the corresponding daily commute might be if the commute is towards the city centre and flats in the city centre are more expensive.

% Pareto optimality
% Point out different nomenclature
In the context of our work, the notion of optimality for bi-objective optimization is that of \emph{Pareto optimality} (also called \emph{efficiency} in other contexts)~\autocite{Ehrgott2005-2}.
Intuitively, under Pareto optimality a solution is considered optimal if no solutions that improve some objective without worsening the other exist.
This definition considers the two flats from earlier both equally optimal.
Under Pareto optimality, the task of solving a bi-objective optimization problem can mean multiple things:
finding a single Pareto-optimal solution, finding a representative solution for each Pareto point (also called non-dominated point in literature~\autocite{Ehrgott2005-2}), or finding all Pareto-optimal solutions.
Most approaches~\autocite{DBLP:conf/cp/SohBTB17,DBLP:conf/cp/JanotaMSM21,DBLP:conf/ijcai/Terra-NevesLM18a} to solving multi-objective optimization under Pareto optimality seem to focus on the second approach where a single solution per Pareto point (i.e., tuple of Pareto-optimal objective values) is computed.
The last task goes one step further and enumerates the full Pareto front (i.e., all Pareto-optimal solutions), even if multiple of the solutions might lead to the same objective values.
All three of these tasks can be solved by the algorithmic approach presented in this thesis.

% Bi-objective vs multi-objective
% Why bi-objective is interesting/enough
Going beyond the bi-objective setting, multi-objective optimization studies optimization problems with an arbitrary number of objectives.
The handle on multiple objectives and what the objective values of an optimal solution actually mean can quickly become hard to grasp.
As humans, we can only visualize three dimensions, meaning a Pareto front over four objectives is already an entirely abstract object while even a three-dimensional one is hard to visualize.
Two objectives, however, form a good trade-off between gaining meaningful information from the second objective over just using a single one, being able to intuitively visualize the Pareto front and not resulting in too many Pareto-optimal solutions.
Additionally, objectives that are considered ``less important'' but should still somehow be included in the optimization can still be added as a threshold constraint.

% Contributions
% Algorithm: single SAT solver; builds on MaxSAT; single vs all
The main contribution of this work is the \algname{} algorithm, a MaxSAT-based bi-objective optimization approach.
\algname{} builds on advances in MaxSAT solving, allowing for variants based on different solution-improving~\autocites{handbook2-maxsat,DBLP:journals/jsat/BerreP10,DBLP:journals/jsat/EenS06} and core-guided~\autocites{DBLP:journals/corr/abs-0712-1097,DBLP:conf/sat/AnsoteguiBL09,DBLP:conf/cp/MorgadoDM14,DBLP:journals/jsat/IgnatievMM19} algorithms.
We propose five different variants of \algname{}, the first four building on the SAT-UNSAT~\autocite{DBLP:journals/jsat/BerreP10}, UNSAT-SAT~\autocite{DBLP:conf/sat/FuM06}, MSU3~\autocite{DBLP:journals/corr/abs-0712-1097} and OLL~\autocite{DBLP:conf/cp/MorgadoDM14} MaxSAT algorithms.
The fifth variant is a hybrid between SAT-UNSAT search and MSU3, aiming to combine the advantages of the two approaches.
In addition to five variants of \algname{}, we also propose multiple refinements for improving its performance:
lazily building the cardinality constraints for both objectives, blocking dominated solutions, domain-specific blocking, bound hardening and other refinements known from core-guided MaxSAT solving.
\algname{} allows for solving all three tasks for bi-objective optimization:
finding a single Pareto-optimal solution, one representative solution for each Pareto point or enumerating all Pareto-optimal solutions.

% Evaluation: study efficiency of different MaxSAT algorithms
After proposing \algname{}, we empirically evaluate the performance of it on two benchmark domains:
learning interpretable decision rules from binary data~\autocite{DBLP:conf/cp/MaliotovM18} and bi-objective set covering.
In the experiments, we compare all five variants of \algname{} to three SAT-based competitors:
enumeration of $P$-minimal solutions~\autocite{DBLP:conf/cp/SohBTB17}, ParetoMCS enumeration~\autocite{DBLP:conf/ijcai/Terra-NevesLM18a} and Seesaw~\autocite{DBLP:conf/cp/JanotaMSM21}.
As a result of this evaluation, we determine which variant of \algname{} is the best-performing overall.
Furthermore, for the best-performing variant, we study the effects of the proposed refinements to determine their effectiveness.

% Signposting for chapters
This thesis is structured as follows:
An overview of propositional satisfiability and maximum satisfiability, highlighting the important preliminaries needed to understand the proposed algorithm, is given in \cref{chap:satisfiability}.
In \cref{chap:biobjective-optimization}, we introduce bi-objective optimization, defining the problem and surveying some existing approaches, based on SAT and other declarative paradigms, as well as probabilistic and meta-heuristic approaches.
After that, in \cref{chap:approach}, \algname{} including five variants and some refinements is proposed.
Finally, in \cref{chap:experiments} we outline the experiments and results, and conclude the thesis in \cref{chap:conclusion}.
\chapter{Propositional Satisfiability\label{chap:satisfiability}}

% Signposting
In this chapter, an overview of propositional satisfiability (SAT) is provided.
We start by defining the SAT decision problem.
The algorithmic approach to solving bi-objective optimization presented in this thesis makes heavy use of incremental SAT solving, employs cardinality constraints and builds on techniques from maximum satisfiability (MaxSAT) solving.
\Cref{sec:inc-sat} gives an overview of incremental SAT solving, and \cref{sec:card-const} discusses cardinality constraints as well as their encoding into propositional logic.
MaxSAT is discussed in \cref{sec:max-sat}.

\section{Propositional Satisfiability\label{sec:sat}}

% Propositional logic and SAT
For a Boolean variable $v$ there are two literals, the positive $v$ and the negative $\lnot v$. 
A clause $C$ is a set of (disjunction over) literals, and a CNF formula $\formula$ is a set of (conjunction over) clauses.
The set of variables and literals appearing in $\formula$ are $\var(\formula)$ and $\lit(\formula)$, respectively.  
A truth assignment $\sol$ maps Boolean variables to 1 (true) or 0 (false).
For a negated variable, $\sol(\neg v) = 1 - \sol(v)$.
The semantics of truth assignments are extended to a clause $C$ and a formula $\formula$ in the standard way: $\sol(C) = \max\{ \sol(l) \mid l \in C\}$ and $\sol(\formula) = \min\{\sol(C) \mid C \in \formula\}$.
When convenient, we view assignments $\sol$ over a set $\var(\formula)$ of variables as sets of literals $\sol = \{ v \mid v \in \var(\formula),  \sol(v) = 1\} \cup \{ \lnot v \mid v \in \var(\formula), \sol(v) = 0\}$.
An assignment $\sol$ for which $\sol(\formula) = 1$ is a solution to $\formula$.
The propositional satisfiability (SAT) problem asks whether a given formula $\formula$ has a solution.
A formula $\formula$ is satisfiable if it has a solution, otherwise it is unsatisfiable.

% Example: SAT
\begin{example}
  Consider the propositional formula $\formula_1 = a \land \lnot b$ over variables $\var(\formula_1) = \{a,b\}$.
  This formula is satisfiable since for $\sol=\{a,\lnot b\}$, $\sol(\formula_1)=1$.
  The formula $\formula_2 = \formula_1 \land (\lnot a \lor b)$ on the other hand is not satisfiable.
  This is because the third clause is the negation of $\formula_1$.
\end{example}

% SAT solvers
A SAT solver is an implementation of an algorithm that determines the satisfiability of a given formula $\formula$.
If $\formula$ is satisfiable, the solver returns ``satisfiable'' (\sat{}) and a solution $\sol$ with $\sol(\formula)=1$;
if $\formula$ is unsatisfiable, the return value is ``unsatisfiable'' (\unsat{}).

% CDCL solvers
So-called Conflict-Driven Clause Learning (CDCL) solvers are the state of the art in terms of SAT solving.
They have been found to solve many real-world problems significantly more efficiently than the exponential worst-case runtime.
Two central techniques in modern CDCL solvers that enable this success are clause learning and search restarts~\autocite{handbook2-cdcl}.
SAT solving algorithms typically assign variables through either decisions or implications.
In a decision assignment, the solver ``chooses'' a value for a given variable based on some heuristics.
On the other hand, an implication assignment is an assignment that is \emph{implied} by the set of clauses and the current (partial) assignment, i.e., an assignment that must hold, otherwise a clause would be falsified.
During the search procedure, the decision assignments made at earlier stages might lead to conflicts, meaning that the formula becomes unsatisfiable with these assignments.
In this case, the solver needs to backtrack and start over from an earlier point.
CDCL solvers add conflict analysis to this process.
In conflict analysis, the solver finds the set of decisions that lead to the conflict and forms a clause that can be added to the formula.
This clause contains information learned from the found conflict and can be retained for the remainder of the search process.
In addition to that, the search of a CDCL solver is restarted based on heuristics to improve performance.

% SAT as a declarative modelling language
The SAT problem was proved to be \NP-complete by~\textcite{DBLP:conf/stoc/Cook71}.
This result is central to the modern day use of SAT in the declarative programming approach to solving other \NP-complete problems by encoding them as a propositional formula, solving them with a SAT solver and then decoding the solution to the original problem domain.
The advantage of using SAT as a declarative programming language for solving other problems comes from the fact that CDCL solvers for SAT are efficient in practice and can solve real-world instances with up to millions of variables and clauses~\autocite{handbook2-cdcl}.
This holds even though propositional logic can only express a very limited set of constraints (compared to other declarative languages) natively.

% Example: SAT modelling
\begin{example}\label{ex:sat-modelling}
  We give an example of a real-world problem, modelling and solving it with the help of SAT.
  Assume the following situation:
  you have three guests coming over and want to make a pizza where each guest likes at least one topping.
  Guest A tells you they like pepperoni, courgette, and bell pepper;
  Guest B likes chicken, avocado, and prawns;
  Guest C likes mushrooms and chilli pepper.
  However, you only have enough money to get two toppings from the store.
  The described situation is an instantiation of the set covering decision problem~\autocite{DBLP:conf/coco/Karp72} that can be encoded into SAT.
  The groups of toppings the guests like form three \emph{sets}, and you want to know if a \emph{cover} (another collection of toppings) of at most size two exists, so that this cover intersects with all three of the sets.
  The set covering problem can be encoded into SAT by introducing a variable for each element and encoding each set as a clause.
  For this example, this yields the following three clauses: $\clause_\text{A} = (v_\text{pepperoni} \lor v_\text{courgette} \lor v_\text{bell pepper})$, $\clause_\text{B} = (v_\text{chicken} \lor v_\text{avocado} \lor v_\text{prawns})$, $\clause_\text{C} = (v_\text{mushroom} \lor v_\text{chilli pepper})$.
  The semantics of these variables are that an assignment $\sol$ assigns $\sol(v)=1$ if and only if the element that this variable encodes is included in the cover the assignment represents.
  In addition to the sets, we need to encode the constraint that the cover can be at most of size two.
  For now, assume that $\texttt{As-CNF}\left(\sum_{v \in V)} v \le 2\right)$ (where $V$ is the set of all variables, i.e., $V = \var(\clause_\text{A}) \cup \var(\clause_\text{B}) \cup \var(\clause_\text{C})$) is a set of clauses that encodes exactly that.
  A set of clauses like this is known as a cardinality constraint and possible ways of representing it in propositional logic will be discussed in \cref{sec:card-const}.\\
  We can now build the formula $\clause_\text{A} \allowbreak \land \clause_\text{B} \allowbreak \land \clause_\text{C} \allowbreak \land \texttt{As-CNF}\left(\sum_{v \in V} v \le 2\right)$ and query a SAT solver for a solution to it.
  The solver will return \unsat{}, telling us that there is no solution to the given problem instance, i.e., that there are no two toppings so that every guest likes at least one of them.
  Now if Guest C says that they also like prawns, $\clause_\text{C}$ changes to $\clause_\text{C} = (v_\text{mushrooms} \lor v_\text{chilli pepper} \lor v_\text{prawns})$ and the formula becomes satisfiable.
  In this case, the SAT solver might return the solution $\sol$ with $\sol(v_\text{bell pepper}) = \sol(v_\text{prawns}) = 1$ and $\sol(v) = 0$ for all other $v$.
  This tells us that in a dish containing bell pepper and prawns, every guest will find something they like.
\end{example}

\section{Incremental SAT Solving under Assumptions\label{sec:inc-sat}}

% Incremental SAT solving
Many applications of SAT solving---such as algorithms for solving maximum satisfiability~\autocite{handbook2-maxsat}---require solving a series of SAT problems that only differ slightly.
To be able to solve these sub problems more efficiently, most modern SAT solvers provide an incremental interface that allows for retaining learned information (in form of, e.g., learned clauses) from previous solver calls~\autocites{DBLP:journals/entcs/EenS03,handbook2-cdcl}.
Retaining this learned information allows CDCL solvers to determine satisfiability for subsequent calls faster in many cases.
In order for these learned clauses from previous queries to be valid in subsequent calls, it is only possible to \emph{add} clauses to the internal formula, not remove them.
Additionally, incremental SAT solvers support solving under assumptions.
The assumptions $\assumps$ are a set of literals that are treated as unit clauses, i.e., a solver call with internal formula $\formula$ and assumptions $\assumps$ either returns \sat{} and a solution $\sol \supset \assumps$, or \unsat{} and a subset $\core \subset \{\lnot l \mid l\in\assumps\}$ such that $\formula \land \bigwedge_{l \in \core} (\lnot l)$ is unsatisfiable.
The subset $\core$ is called an unsatisfiable \emph{core}~\autocite{handbook2-cdcl} and implied by $\formula$, meaning the set of assumptions it stems from cannot all be satisfied at the same time.

\begin{example}\label{ex:inc-sat}
  Recall the situation from \cref{ex:sat-modelling} after the change of Guest C liking prawns.
  Assume that there is no bell pepper at the store.
  To check if we can still make a pizza with only two toppings such that every Guest likes at least one of the toppings, we can use incremental SAT solving.
  Instead of adding the fact that bell pepper is not available as a new clause $(\lnot v_\text{bell pepper})$, we can set the assumptions to $\assumps=\{\lnot v_\text{bell pepper}\}$.
  Solving with these assumptions, the solver might return the solution $\sol$ with $\sol(v_\text{courgette}) = \sol(v_\text{prawns}) = 1$ and $\sol(v) = 0$ for all other $v$. \\
  If instead prawns are not available, we can set the assumptions to $\assumps=\{\lnot v_\text{prawns}\}$.
  In this case, the solver will return \unsat{} and the core $\core=\{v_\text{prawns}\}$.
  The interpretation of this core is that prawns need to be available so that we can make a pizza that satisfies all constraints.
\end{example}

\section{Encoding Cardinality Constraints as Totalizers\label{sec:card-const}}

\TODO{rewrite motivation, splitting it into applications with cardinality constraints (that we want to be able to represent) and algorithms, e.g., enforcing a bound on a objective}

% Cardinality constraints
Numerous applications and encodings based on propositional logic require so-called cardinality constraints.
These applications include algorithms for solving maximum satisfiability and the algorithmic approach presented in this thesis.
Informally speaking, cardinality constraints enforce a bound on how many literals in a given set can be assigned to true.
Formally, for a set $L$ of literals and a bound $b \in \mathbb{N}$, $\texttt{As-CNF}\left(\sum_{l \in L} l \circ b\right)$ denotes a CNF formula that encodes the linear (in)equality $\sum_{l \in L} l \circ b$, where $\circ \in \{< ,> ,\geq, \leq, =\}$.
Numerous methods of forming such CNF formulas are known (e.g.,~\autocites{DBLP:conf/cp/BailleuxB03,DBLP:conf/cp/Sinz05,DBLP:journals/jsat/EenS06}).

% Totalizer encoding
In this work we make use of the totalizer encoding.
Given a set $L$ of $n$ input literals and a bound $k\in\{1,\dots,n\}$, the (incremental) totalizer encoding~\autocites{DBLP:conf/cp/BailleuxB03,DBLP:conf/cp/MartinsJML14} produces a CNF formula $\tot(L, k)$ that defines a set $\{\ov{L}{1}, \ldots, \ov{L}{k+1}\} \subset \var(\tot(L,k))$ of \emph{output literals} that---informally speaking---count the number of literals in $L$ assigned to true by solutions to $\tot(L,k)$:
if $\sol$ is an assignment that satisfies $\tot(L,k)$ and $b < k$, then $\sol(\ov{L}{b}) = 1$ if and only if $\sum_{l \in L} \sol(l) < b$.
For applications where only either upper or lower bounding of the number of literals assigned to true is needed, the size of the encoded totalizer can be reduced by encoding implications in only one direction instead of both (i.e., $\ov{L}{b} \leftarrow (\sum_{l \in L} l < b)$ or $\ov{L}{b} \rightarrow \sum_{l \in L} l < b$, but not both).
The algorithmic approach presented in this paper only uses upper bounding cardinality constraints, therefore we employ the size-reduced version of the totalizer encoding.
The \emph{incremental} totalizer supports both increasing the bound $k$ and adding new input literals without having to rebuild the whole formula:
we have that $\tot(L, k) \subset \tot(L, k')$ and $\tot(L, k) \subset  \tot(L \cup L', k)$ hold for any bound $k' > k$ and set $L'$ of literals for which $L \cap L' =  \emptyset$. 
This is desirable when making incremental SAT calls where the bound or the set of input literals changes between calls, since it allows for making earlier calls to the SAT solver on a formula with fewer clauses and retaining information from these calls.
Extending the totalizer means adding clauses while reusing the ones that were previously added.

We use $\ove{L}{b}$ as a shorthand for the literal $\ov{L}{b+1}$;
furthermore, if the maximal bound $k$ of a totalizer $\tot(L,k)$ is clear from context or the bound is $k=|L|$, we omit it and simply write $\tot(L)$.
Note that there is only one satisfying assignment of the auxiliary and output variables of the totalizer encoding, given an assignment of its input variables.
As such we will leave them out from the solutions we describe in favour of brevity and clarity of examples. 

\section{Maximum Satisfiability\label{sec:max-sat}}

% MaxSAT
Maximum satisfiability (MaxSAT) is the optimization variant of the SAT decision problem.
It asks for a solution that satisfies as many of the given clauses as possible.
Most commonly and in this work, MaxSAT refers to the extension of \emph{weighted partial} MaxSAT, in which a set of \emph{hard} clauses $\hards$ and a set of \emph{soft} literals $\softs$ are given.
Soft clauses can be modelled by adding a new relaxation variable to the clause, treating it as a hard clause and the relaxation variable as a soft literal.
Each clause $l \in \softs$ is assigned a weight $w_l$ and a solution $\sol$ that satisfies $\hards$ while achieving minimum weight of assigned soft literals $\sum_{l \in \softs} \sol(l) \cdot w_l$ is optimal for the problem.
Since the \NP-complete SAT decision problem (recall \cref{sec:sat}) can be solved by MaxSAT (its optimization extension), MaxSAT is \NP-hard.

% MaxSAT as a modelling language
In the same way that SAT can be used as a declarative language to solve other decision problems, MaxSAT can be used to solve optimization problems.

% Example: MaxSAT modelling
\begin{example}\label{ex:maxsat-modelling}
  Recall \cref{ex:sat-modelling}.
  We can change the problem from the set covering decision problem to the optimization variant by asking for the \emph{smallest} cover rather than giving an upper bound on the size of the cover.
  This is done by removing the cardinality constraint from the encoding and using the encoded sets as hard clauses for a partial MaxSAT instances.
  In addition to that, the set of soft literals is assigned as the set of all variables $\softs = V$.
  Solving this MaxSAT instance for the initial situation, the MaxSAT solver will return that at least three of the soft literals need to be set, i.e., that at least three toppings must be chosen so that every guest likes something.
  An optimal solution $\sol$ that the solver might return for this has $\sol(v_\text{courgette}) = \sol(v_\text{chicken}) = \sol(v_\text{mushroom}) = 1$ and $\sol(v) = 0$ for all other $v$.
  After we modify the instance to account for guest C liking prawns, the MaxSAT solver might give us the optimal solution $\sol$ with $\sol(v_\text{bell pepper}) = \sol(v_\text{prawns}) = 1$ and $\sol(v) = 0$ for all other $v$.
  There is no valid cover with size smaller than two.
\end{example}

% MaxSAT solving algorithms
Many algorithms for solving MaxSAT have been proposed over the recent years.
Best scalability to large instances is achieved by algorithms that solve MaxSAT by solving a number of SAT instances with the help of an underlying SAT solver, implicit hitting set approaches, or solving MaxSAT via integer programming~\autocite{handbook2-maxsat}.
This thesis builds on the MaxSAT algorithms that solve a sequence of SAT problems;
these algorithms can be categorized as either solution-improving, bound-improving or core-guided algorithms.
Detailed descriptions for most of these algorithms can be found in~\autocite{handbook2-maxsat}.

% SAT-UNSAT and UNSAT-SAT
The two simplest algorithms we build on are solution-improving SAT-UNSAT search~\autocite{DBLP:journals/jsat/BerreP10} and bound-improving UNSAT-SAT search~\autocite{DBLP:conf/sat/FuM06}.
SAT-UNSAT (also known as Linear SAT-UNSAT (LSU)~\autocite{handbook2-maxsat}) search solves MaxSAT by starting from a known satisfiable solution for the hard clauses.
From this, a cardinality constraint is added to the SAT solver, enforcing that the next found solution achieves a better objective value than the last.
If such a solution is found, the cardinality constraint is tightened to the objective value of this new solution.
As soon as the SAT solver return \unsat{} for a call, the last found solution is known to be optimal.
Because this search procedure goes through a series of satisfiable calls first, terminating at the first unsatisfiable call, it is known as SAT-UNSAT search.
UNSAT-SAT search is similar, however the optimal value is found by lower-bounding instead of upper-bounding.
It starts by adding a tight cardinality constraint to the SAT solver---resulting in unsatisfiable queries---and slowly loosening the constraint until a first satisfiable query is reached.
The idea of UNSAT-SAT search was first proposed not as an algorithm, but as a formulation for subset satisfiable boolean formulas~\autocite{DBLP:journals/tcad/XuRS03}.
Later, it was rephrased as a MaxSAT algorithm~\autocite{DBLP:conf/sat/FuM06}.

% MSU3 and OLL
The other two algorithms we are building on in this thesis are MSU3~\autocite{DBLP:journals/corr/abs-0712-1097} and OLL~\autocite{DBLP:conf/cp/MorgadoDM14,DBLP:journals/jsat/IgnatievMM19}.
Both of these search procedures are core-guided, meaning they make use of unsatisfiable cores returned by the SAT solver to more efficiently steer the search.
Core-guided MaxSAT was first proposed with the algorithm now known as Fu-Malik~\autocite{DBLP:conf/sat/FuM06}.
The central insight behind core-guided MaxSAT is that an optimal solution must set at least one of the soft literals in every core.
In the MSU3 algorithm, the soft literals are assumed to false and cores are extracted over them.
When a soft literal appears in a core, this literal is removed from the assumptions and then added to a cardinality constraint allowing some soft literals to be true.
Every iteration increases the bound of the cardinality constraint by one.
With this, MSU3---as also every other core-guided algorithm---makes a series of unsatisfiable SAT queries, terminating at the first satisfiable one, which yields the optimal solution.
OLL, which was first proposed for the paradigm of answer set programming~\autocite{DBLP:conf/iclp/AndresKMS12} and later applied to MaxSAT~\autocite{DBLP:conf/cp/MorgadoDM14,DBLP:journals/jsat/IgnatievMM19}, differs from MSU3 in how the extracted cores are relaxed.
Instead of building one large cardinality constraint over all cores, it builds an individual cardinality constraint for each core.
Furthermore, these cardinality constraints are not treated as hard but as additional \emph{soft} clauses, meaning they can be relaxed in subsequent iterations of the algorithm.

% MaxSAT algorithms and cardinality constraints
All MaxSAT algorithms that solve a series of SAT queries need to build cardinality constraints over the soft literals of the MaxSAT instance and typically these only change slightly from query to query.
For this reason, incremental cardinality constraints such as the incremental totalizer (discussed in \cref{sec:card-const}) are employed in modern MaxSAT solvers to allow for incrementally using the underlying SAT solver without resetting it.
\chapter{Bi-Objective Optimization and Pareto Optimality\label{chap:biobjective-optimization}}

% Signposting
This chapter describes bi-objective optimization, starting with a definition of the problem and the more general case of multi-objective optimization.
We also define the notation for bi-objective optimization in the context of SAT that is used in this work.
In a second and last section, we discuss different approaches to solving bi-objective optimization problems.

\section{Multi-Objective Optimization Under Pareto Optimality\label{sec:multiopt}}

% Multiobjective optimization
Multi-objective optimization is the problem of optimizing $\nobj$ objective functions with the decision variable $x$, while $x$ is from a feasible set $\feasible$~\autocite{Ehrgott2005-1}.
W.l.o.g., we assume that all optimizations seek to minimize the objective function.
Formally a multi-objective optimization problem (MOOP) is the following:
\begin{equation}\label{eq:moop}
  \min (\generalobj_1(x),\dots,\generalobj_\nobj(x))\ \text{s.t.}\ x\in \feasible.
\end{equation}

% Pareto optimality
For multi-objective optimization problems, since the objectives might be in conflict with each other, no single optimal solution exists.
However, the notions of domination and pareto optimality can be used to define what a minimal point for multiple objective functions is.
\begin{definition}[Domination~\autocite{Ehrgott2005-2}]
  Given a MOOP as defined in \cref{eq:moop} and two solutions $x,x' \in \feasible$, $x$ dominates $x'$ (w.r.t.\ $\generalobj_1,\dots,\generalobj_\nobj$) if (i)~$\generalobj_i(x) \leq \generalobj_i(x')$ for $i=1,\dots,\nobj$, and (ii)~$\generalobj_i(x) < \generalobj_i(x')$ for some $i\in\{1,\dots,\nobj\}$.
  We represent $x$ dominating $x'$ as $x \prec x'$.
\end{definition}
\begin{definition}[Pareto optimality~\autocite{Ehrgott2005-2}]
  Given a MOOP as defined in \cref{eq:moop}, a solution $x \in \feasible$ is pareto-optimal (w.r.t.\ $\generalobj_1,\dots,\generalobj_\nobj$) iff there is no $x' \in \feasible$ such that $x' \prec x$, i.e., $x$ is not dominated by any other solution.
\end{definition}
When the objectives are clear from context, we will simply say that a solution $x$ is pareto-optimal.
Note that there can be multiple pareto-optimal solutions to a MOOP.
The set of all pareto-optimal solutions is called the pareto front (w.r.t.\ $\generalobj_1,\dots,\generalobj_\nobj$);
the tuple $(\generalobj_1(x),\dots,\generalobj_\nobj(x))$ for a pareto-optimal $x$---i.e., the image of $x$ in objective space---is a pareto point (which multiple solutions can correspond to).

\section{Bi-Objective Optimization in a SAT Context\label{sec:biopt}}

% Bi-objective optimization in a SAT context
The subset of all multi-objective optimization problems which this work provides a solving algorithm for, is \emph{bi}-objective optimization, where the number of objective functions $\nobj=2$.
Furthermore, we restrict the objective functions to be linear.
Bi-objective optimization problems can be solved by many approaches, but in this work we focus on SAT-based approaches.
We therefore formalize the problem in a context of propositional satisfiability.
An objective $\Obj$ is a multiset of literals, which allows for representing objective functions with non-unit coefficients.
The value $\Obj(\sol)$ of a truth assignment $\sol$ under $\Obj$ is $\Obj(\sol) = \sum_{l \in \Obj} \sol(l)$, i.e., the number of the literals in $\Obj$ that $\sol$ assigns to 1. 
Weighted objectives are represented by adding a literal multiple times.
Formalizing the objectives this way and encoding the feasible set $\feasible$ as a propositional formula $\formula$ is similar to MaxSAT, where $\formula$ resembles the hard clauses while the two objectives resemble two sets of (unit) soft clauses.
With our algorithm, we consider the task of computing a representative solution for each pareto point as well as the task of enumerating all solutions in the pareto front.

% Example: A bi-objective problem
\begin{figure}
  \begin{minipage}{0.36\textwidth}
    \footnotesize
    \begin{align*}
      \formula = &\{ \texttt{As-CNF}\left(\sum_ {x \in \Obj_\inc \cup \Obj_\dec} x \geq 3 \right), \\
        &(i_1 \lor i_2),  (i_2 \lor i_3), \\
        &(d_1 \lor d_2), (d_2 \lor d_3) \} \\ \\
      \Obj_\dec =&\{ d_1,d_2, d_3\}   \\ 
      \Obj_\inc =&\{ i_1,i_2, i_3\}  
    \end{align*}
  \end{minipage}
  \;
  \begin{minipage}{0.6\textwidth}
    \includegraphics{search-trace.pdf}
  \end{minipage}
  \caption{Left: An example formula $\formula$ and two objectives $\Obj_\inc$ and $\Obj_\dec$.
    Right: the solution space of $\formula$ w.r.t.\ $\Obj_\inc$ and $\Obj_\dec$.
    The solutions $\tau^o_1$ and $\tau^o_2$ (solid points) are pareto-optimal, while $\tau^c_i$ for $i=1,\ldots,4$ are not.\label{fig:search-trace}}
\end{figure}
\begin{example}\label{ex:main}
  An example formula $\formula$ and two objectives $\Obj_\inc$ and $\Obj_\dec$ are shown on the left side of \cref{fig:search-trace}. 
  The solution space is illustrated on the right.
  The two solid dots correspond to the two pareto points of $\formula$ w.r.t.\ $\Obj_\inc$ and $\Obj_\dec$. 
  Examples of pareto-optimal solutions corresponding to these points are $\sol^o_1 = \{d_1, d_3, i_2, \lnot d_2, \lnot i_1, \lnot i_3\}$ and $\sol^o_2 = \{i_1, i_3, d_2, \lnot i_2, \lnot d_1, \lnot d_3\}$.
  The solution $\sol^c_3 = \{d_1, d_2, d_3, i_2, \lnot i_1, \lnot i_3\}$ is dominated by $\sol^o_1$ ($\sol^o_1 \prec \sol^c_e$) because $\Obj_\inc(\sol^o_1) \leq \Obj_\inc(\sol^c_3)$ and $\Obj_\dec(\sol^o_1) < \Obj_\dec(\sol^c_3)$.
\end{example}

% Proposition: Ordered pareto front
An important property of pareto-optimal solutions to bi-objective problems is summarized by the next proposition.
\begin{proposition}[Adapted from~\autocite{DBLP:conf/aaai/HartertS14}] \label{prop:biobjective}
  Sorting the pareto-optimal solutions of $\formula$ w.r.t.\ increasing values of $\generalobj_1$ is equivalent to sorting them w.r.t.\ decreasing values of $\generalobj_2$ and vice-versa.
\end{proposition}

% Example: Ordered pareto front
\begin{example}
  Consider the formula $\formula$, the objectives $\Obj_\inc$ and $\Obj_\dec$ and the two pareto-optimal solutions $\sol^o_1$ and $\sol^o_2$ from \cref{fig:search-trace} and \cref{ex:main}.
  By the definition of pareto-optimality, lowering the value of one objective of a pareto-optimal solution has to increase the value of the other;
  we have $\Obj_\inc(\sol^o_1) = 1 < 2 = \Obj_\inc(\sol^o_2)$ and $\Obj_\dec(\sol^o_1) = 2 > 1 = \Obj_\dec(\sol^o_2)$.
\end{example}

\section{Approaches to Bi-Objective Optimization\label{sec:approaches}}

In this section, we give an overview of different approaches to solving bi-objective optimization problems.
The focus hereby lies on the first subsection, describing SAT-based approaches.
This is because the algorithm we are proposing in this work is SAT-based and therefore other SAT-based approaches are directly competing.
In the section thereafter, we survey exact approaches based on other declarative optimization paradigms, mainly constraint and mixed integer programming.
The last section gives a brief overview of methods to approximate the pareto front, rather than finding it exactly.
In addition to approaches solving bi-objective optimization under pareto optimality, we also discuss some approaches that make use of different optimality definitions.

\subsection{SAT-Based Approaches\label{sec:sat-based}}

% Signposting
We highlight three SAT-based approaches to bi-objective optimization:
enumeration of $P$-minimal solutions, Seesaw, and enumeration of pareto-minimal correction sets.

\subsubsection{$P$-minimal Solution Enumeration\label{sec:p-minimal}}

% P-minimal solution enumeration
The approach perhaps closest to ours is solving multi-objective constraint optimization problems by enumerating so-called $P$-minimal solutions~\autocite{DBLP:conf/cp/SohBTB17,DBLP:conf/ftp/KoshimuraNFH09}.
Since this approach is so close to ours, we are using it as one of the two approaches we empirically compare \algname{} to in \cref{chap:experiments}.
The $P$-minimal approach  corresponds to enumerating the solutions of $\formula^\text{W} = \formula \land \tot(\Obj_{\inc}) \land \tot(\Obj_{\dec})$ that are subset-minimal w.r.t.\ the set of outputs of the totalizers.
More precisely, if $P$ is the set of output literals of $\tot(\Obj_{\inc}) \land \tot(\Obj_{\dec})$, then the goal is to enumerate solutions $\sol$ such that no other solution $\sol$ has $\{ l \mid l \in P \land \sol(l) = 0\} \subsetneq \{ l \mid l \in P \land \sol(l) = 0\}$.
The procedure for enumerating such solutions (detailed in~\textcite{DBLP:conf/ftp/KoshimuraNFH09}) works by (i)~using a solver to obtain any solution $\sol$ of $\formula^\text{W}$, (ii)~iteratively minimizing the subset of variables of $P$ set to true by the solution, and, once a minimal solution $\sol_m$ has been found, (iii)~adding the clause $(\ov{\Obj_{\inc}}{k_1} \lor \ov{\Obj_{\dec}}{k_2})$ containing the output variables corresponding to the lowest index set to true by $\sol_m$.

% Example: P-minimal
\begin{example}\label{ex:pmin}
  Consider the formula $\formula$ and two objectives $\Obj_\inc$ and $\Obj_\dec$ from \cref{fig:search-trace}.
  $P$-minimal starts by building two totalizers $\tot(\Obj_\inc)$ and $\tot(\Obj_\dec)$ and invoking the SAT solver on $\formula^\text{W} = \formula \land \tot(\Obj_\inc) \land \tot(\Obj_\dec)$.
  The result is satisfiable, assume the first solution obtained is $\sol^c_1 = \{i_1, i_2, i_3, d_1, d_2, d_3\}$. 
  In order to minimize $\sol^c_1$, the clause $(\ov{\Obj_\inc}{3} \lor \ov{\Obj_\dec}{3})$ is added to the SAT solver, and the solver is invoked again under the assumptions $\{ \ove{\Obj_\inc}{3}, \ove{\Obj_\dec}{3} \}$.
  The added clause blocks $\sol^c_1$ and all solutions dominated by $\sol^c_1$ from the search space.
  Assume the next solution obtained is $\sol^c_5 = \{d_1, d_3, i_1, i_3, \lnot d_2, \lnot i_2\}$. 
  Again, a clause $(\ov{\Obj_\inc}{2} \lor \ov{\Obj_\dec}{2})$ is added, and the SAT solver is queried with assumptions $\{ \ove{\Obj_\inc}{2}, \ove{\Obj_\dec}{2} \}$.
  The result is SAT, assume the solution obtained is $\sol^o_2 = \{ i_1, i_3, d_2, \lnot i_2, \lnot d_2, \lnot d_3\}$. 
  $P$-minimal then adds the clause $(\ov{\Obj_\inc}{2} \lor \ov{\Obj_\dec}{1})$ and invokes the solver again under the assumptions $\{ \ove{\Obj_\inc}{2}, \ove{\Obj_\dec}{1} \}$.
  The result is UNSAT which proves that $\sol^o_2$ is pareto-optimal. 
  To find a next pareto-optimal solution, the solver is queried without any assumptions for a new solution to start the minimization process from.
\end{example}

% P-minimal not ordered
Note that $P$-minimal has no guarantee on the order that the solutions are enumerated in. 
Intuitively, when an intermediate solution $\sol$ is found, the following SAT solver call either provides another solution that dominates $\sol$, or proves that $\sol$ is pareto-optimal.  

% Extending P-minimal to enumerate the full pareto front
As presented in~\cite{DBLP:conf/cp/SohBTB17}, the $P$-minimal approach will only enumerate a single solution per pareto point.
In our implementation we extended $P$-minimal to the task of enumerating all solutions on the pareto front.
Specifically, we add a new relaxation variable $r$ to the clause added each iteration for use as an assumption to enumerate all solutions at that pareto point:
the next SAT solver query is done including the assumption $\lnot r$, if a dominating solution is found, the clause is hardened by adding $\lnot r$ as a unit clause.
If no dominating solution is found, all solutions corresponding to the just discovered pareto point can be enumerated when removing the assumption $\lnot r$ by blocking every found solution and querying the solver again until it returns UNSAT.
If the next solution found dominates the previous one, we harden the clause added at the previous iteration by adding $\lnot r$ as a unit clause.
Once all solutions for that pareto point are enumerated, the clause is hardened.

% Example: P-minimal for full pareto front
\begin{example}
  Consider the same invocation of $P$-minimal as in \cref{ex:pmin}.
  In order to enumerate all solutions in the pareto front, the clause added in the first iteration is $(\ov{\Obj_\inc}{3} \lor \ov{\Obj_\dec}{3} \lor r_1)$ and the solver is queried again with the assumptions $\{ \ove{\Obj_\inc}{3}, \ove{\Obj_\dec}{3}, \lnot r_1 \}$.
  Since the solver will return a dominating solution, the clause added is hardened by adding the unit clause $\lnot r$ to the solver.
  The second iteration is modified similarly as the first, adding the relaxation variable $r_2$.
  In the third iteration, the added clause is $(\ov{\Obj_\inc}{2} \lor \ov{\Obj_\dec}{1} \lor r_3)$ and the solver call with assumptions $\{ \ove{\Obj_\inc}{2}, \ove{\Obj_\dec}{1}, \lnot r_3 \}$ is unsatisfiable.
  Now, by iteratively querying the solver with the assumptions $\{ \ove{\Obj_\inc}{2}, \ove{\Obj_\dec}{1} \}$ and blocking all found solutions, the set of solutions corresponding to the pareto point $(2,1)$ are enumerated.
\end{example}

\subsubsection{Enumeration of Pareto-Minimal Correction Sets\label{sec:pareto-mcs}}

% Pareto MSCes
In~\textcite{DBLP:conf/ijcai/Terra-NevesLM18a,DBLP:conf/aaai/Terra-NevesLM18,DBLP:conf/ijcai/Terra-NevesLM18} an approach for computing pareto-optimal solutions via so-called pareto-minimal correction sets (paretoMCSes) was proposed.
A paretoMCS (w.r.t.\ two objectives $\Obj_1$ and $\Obj_2$) consists of two sets of literals $(M_1, M_2)$ such that (i)~$M_1 \subset \Obj_1$ and $M_2 \subset \Obj_2$, and (ii)~there is a pareto-optimal solution $\sol$ that sets $\sol(l) = 1$ for all $l \in M_1 \cup M_2$ and $\sol(l) = 0$ for all other $l \in (\Obj_1 \cup \Obj_2) \setminus (M_1 \cup M_2)$.
In~\textcite{DBLP:conf/ijcai/Terra-NevesLM18a}, the computation of pareto-optimal solutions is reduced into the computation of paretoMCSes.
The task of computing paretoMCSes is accomplished by enumerating all subsets $S \subset  (\Obj_1 \cup \Obj_2)$ for which (i)~$\formula \land \bigwedge_{l \in  (\Obj_1 \cup \Obj_2) \setminus S} (\lnot l)$ is satisfiable, and (ii)~$\formula \land \bigwedge_{l \in  (\Obj_1 \cup \Obj_2) \setminus S'} (\lnot l)$ is unsatisfiable for all $S' \subsetneq S$.
Let $\mathcal{S}$ be the collection of all such sets.
The computation of $\mathcal{S}$ corresponds to MCS enumeration to which numerous algorithms have been proposed~\autocite{DBLP:conf/lpar/BendikC20,DBLP:conf/hvc/MorgadoLM12,DBLP:conf/sat/PrevitiMJM17}.
The pareto-optimal solutions are obtained by extracting the solutions satisfying $\formula \land \bigwedge_{l \in  (\Obj_{\inc} \cup \Obj_\dec) \setminus S} (\lnot l)$ for all $S \in \mathcal{S}$ and removing the dominated ones~\autocite{DBLP:conf/ijcai/Terra-NevesLM18a}.
The paretoMCS approach to multi-objective optimization is approximative in that it can only guarantee that a solution is pareto-optimal once the full set $\mathcal{S}$ has been computed.
Since paretoMCS enumeration is approximative in this sense, we are not comparing the performance of \algname{} to this approach, since our algorithm has the algorithmic advantage that all found solutions are immediately known to be pareto-optimal.
% In contrast, every minimal solution found during the $P$-minimal approach of~\textcite{DBLP:conf/cp/SohBTB17} and every solution returned by the $\E$ subroutine of \cref{alg:base-algorithm} is immediately known to be pareto-optimal.

% Example: A MCS that is not pareto-optimal
\begin{example}\label{ex:MCS}
  Consider the formula $\formula$ and two objectives $\Obj_\inc$ and $\Obj_\dec$ from \cref{fig:search-trace}.
  The paretoMCS enumeration procedure will return the solution $\sol = \{d_1, d_3, i_1, i_3, \lnot d_2, \lnot i_2\}$ since no solution $\sol_s$ of $\formula$ has $\{x \in \Obj_\inc \cup \Obj_\dec \mid  \sol_s(x) = 1\} \subsetneq \{d_1, d_3, i_1, i_3\}$.
  The solution $\sol$ is not pareto-optimal, but only filtered out when a solution that dominates it is enumerated.
  However, there are no guarantees on when such a dominating solution is found. 
\end{example}

\subsubsection{Implicit Hitting Set Approach: Seesaw\label{sec:seesaw}}

% Implicit hitting set approach
Implicit hitting set approaches for solving combinatorial optimization problems were first proposed in \textcite{DBLP:journals/ior/Moreno-CentenoK13}.
The overarching idea is that an optimization problem is modelled as a set $\cores$ of so-called \emph{cores} which represent an undesirable or conflicting substructure of the problem.
(Note that these cores are not necessarily equal to a core in SAT solving as described in \cref{sec:inc-sat}.)
All cores in $\cores$ need to be resolved by selecting a smallest hitting set $\hs$ that contains at least one element from each core.
However, the set $\cores$ is not explicitly know;
instead, it is defined implicitly through a separation oracle that can be used to test if a given hitting set $\hs$ intersects with all cores.
If the hitting set does not intersect with all cores, the oracle returns a core that is not hit.

% MaxHS and Seesaw
In~\textcite{DBLP:conf/cp/DaviesB13,DBLP:conf/sat/DaviesB13,DBLP:conf/cp/DaviesB11,DBLP:conf/sat/BergBP20}, an implicit hitting set algorithm for solving MaxSAT was proposed and refined;
other successful applications can be found in~\textcite{DBLP:conf/cp/IgnatievPLM15,DBLP:conf/kr/SaikkoWJ16,DBLP:conf/cade/FazekasBB18,DBLP:conf/kr/SaikkoDAJ18}.
Recently, Seesaw~\autocite{DBLP:conf/cp/JanotaMSM21} was proposed as a generalized implicit hitting set framework for bi-objective optimization.
In contrast to our work, a main ingredient in Seesaw is the idea of treating one of the objectives as a black box.
This allows for---but also requires---problem-specific instantiations of the black box.

% Seesaw in a SAT context
While the original paper presents Seesaw in general terms, in our context the Seesaw algorithm computes pareto-optimal solutions of a formula $\formula$ by maintaining a collection $\cores$ of subsets of $\Obj_\inc$ that are called \emph{cores}.
Informally speaking, every solution $\sol$ that improves on $\Obj_\dec$ needs to assign at least one literal from each core to 1.
The algorithm works iteratively by computing a hitting set $\hs \subset \Obj_\inc$ (using an integer programming solver), i.e., a subset-minimal set of literals of $\Obj_\inc$ that intersects with each core in $\cores$, and then a solution $\sol$ that sets $\sol(o) = 1$ for each $o \in \hs$ and $\sol(o) = 0$ for each $o \in \Obj_\inc \setminus \hs$ and for which $\Obj_\dec(\sol)$ is the smallest possible value for all such solutions if one exists.
The iteration then extracts a new core that $\hs$ does not intersect with.
The pareto-optimal solutions of $\formula$ are identified by the size of the hitting set increasing.
More precisely, if the hitting set is found to increase from size $|\hs|$ to size $|\hs_2|$ with $|\hs_2|>|\hs|$, the solution $\sol$ found with a hitting set of size $|\hs|$ that has the smallest minimal value $\Obj_\dec(\sol)$ is pareto-optimal~\autocite{DBLP:conf/cp/JanotaMSM21}.

% Example: Seesaw
\begin{example}
  Consider the formula $\formula$ and two objectives $\Obj_\inc$ and $\Obj_\dec$ from \cref{fig:search-trace}. 
  Initially there are no cores, so $\cores = \emptyset$ and $\hs = \emptyset$.
  Since there is no $\sol$ that sets $\sol(u) = 0$ for each $u \in \Obj_\inc$, the iteration ends by extracting the core $\Obj_\inc$. 
  The intuition is, that any solution $\sol$ of $\formula$ sets at least one variable in $\Obj_\inc$ to 1.
  In the next iteration, a hitting set over $\cores = \{ \Obj_\inc \}$ is computed.
  There are a number of alternatives;
  assume $\hs = \{ i_1 \}$.
  Since there is no $\sol$ that set $\sol(i_2) = \sol(i_3) = 0$, the iteration ends with extracting the core $\{ i_2, i_3\}$.
  The same intuition as earlier holds for this core.
  Assume the next hitting set computed is $\hs = \{i_2\}$.
  Now there is a $\sol$ that set $\sol(i_1) = \sol(i_3) = 0$;
  one that minimizes $\Obj_\dec(\sol)$ is $\sol^o_1 = \{d_1, d_3, i_2, \lnot d_2, \lnot i_1, \lnot i_3 \}$.
  The iteration ends with extracting the core $\core = \{i_1, i_3\}$.
  Now the intuition is that, since $\sol^o_1$ minimizes $\Obj_\dec$ over solutions that assign $\sol(u) = 0$ for every $u \notin \hs$, every solution that obtains a lower value of $\Obj_\dec$ assigns at least one literal of $\core$ to 1 as well. 
  Now we have $\cores = \{ \Obj_\inc, \{i_2, i_3\}, \{i_1, i_3\}\}$;
  the only smallest hitting set is $\hs = \{i_3\}$.
  There is no $\sol$ that sets $\sol(i_1) = \sol(i_2) = 0$ so a new core $\{i_1, i_2\}$ is extracted. 
  Next, one possible hitting set is $\hs = \{i_1, i_3\}$.
  Since the size of the hitting set grew from 1 to 2, the algorithm concludes that $\sol^o_1$ is pareto-optimal. 
  The algorithm continues in this manner, finding the pareto-optimal $\sol^o_2$ in the process.
  After computing the hitting set consisting of all literals in $\Obj_\inc$, the core extracted is $\emptyset$ at which point the algorithm terminates. 
\end{example}

% Refined core extraction strategy
Note that the core-extraction strategy that only computes $\Obj_\inc \setminus \hs$ as the new core detailed in the example corresponds to what is called the weakest possible strategy in~\textcite{DBLP:conf/cp/JanotaMSM21}.
Seesaw is only feasible in practice when using a stronger core-extraction strategy---like the improved version detailed in the original Seesaw paper---since Seesaw otherwise reduces to enumerating all subsets of $\Obj_\inc$ as hitting sets~\autocite{DBLP:conf/cp/JanotaMSM21}.
Also note that, in contrast to \algname{} and $P$-minimal, extending Seesaw as it is presented in~\textcite{DBLP:conf/cp/JanotaMSM21} to support the enumeration of all pareto-optimal solutions seems non-trivial.
For a non-formal intuition note that, while Seesaw is guaranteed to find at least one solution obtaining the objective values of each pareto-optimal point, the non-deterministic hitting set computation might steer the algorithm past other solutions that obtain the same values.

\subsubsection{SAT-Based Lexicographic Optimization\label{sec:lex-opt}}

% (SAT-based) lexicographic optimization
In contrast to bi-objective optimization under pareto optimality, the objectives can also be lexicographically minimized~\autocite{Ehrgott2005-1}.
For two objectives, given a feasible set $\feasible$ and two objectives $\generalobj_1$ and $\generalobj_2$, a solution $x$ dominates another solution $x^d$ in the lexicographic sense if (a)~$\generalobj_1(x) < \generalobj_1(x^d)$, or (b)~$\generalobj_1(x) = \generalobj_1(x^d)$ and $\generalobj_2(x) < \generalobj_2(x^d)$.
Informally speaking, in contrast to pareto-optimality, lexicographic optimization imposes an explicit preference over the objectives and asks to compute a solution that minimizes $\Obj_1$ using $\Obj_2$ as a tie-breaker.
There is also earlier work on SAT-based lexicographic optimization~\autocite{DBLP:journals/ors/EhrgottG00,DBLP:conf/ijcai/ArgelichLS09,DBLP:journals/amai/Marques-SilvaAGL11}. 

% Lexicographic optimization and MaxSAT
Lexicographic optimization is closely related to the so-called multi-level optimization problem.
In particular, both can be cast as a single objective weighted optimization problem and solved with a MaxSAT solver~\autocite{DBLP:conf/ijcai/ArgelichLS09,DBLP:journals/amai/Marques-SilvaAGL11}.
In fact, many modern MaxSAT solvers exploit  multilevel properties of input instances in order to improve search efficiency~\autocite{DBLP:conf/vmcai/PaxianRB21,DBLP:conf/cp/AnsoteguiBGL12}.

\subsection{Other Declarative Optimization Paradigms\label{sec:other-approaches}}

% CP-based approaches
Beyond SAT-based approaches, multi-objective optimization has been studied in other declarative optimization paradigms.
An early algorithm that can be used in constraint programming is a lexicographic method first presented in~\textcite{Wassenhove1980}.
\Textcite{DBLP:conf/ecai/Gavanelli02} proposed a branch-and-bound-based algorithm that outperforms that presented in~\textcite{Wassenhove1980}.
This filtering algorithm was improved by the pareto constraint presented in~\textcite{DBLP:conf/cp/SchausH13,DBLP:conf/aaai/HartertS14}
The resulting search algorithm is similar to paretoMCSes in that it maintains a set $\mathcal{T}$ of solutions that do not dominate each other.
When a new solution is found, any solution it dominates is removed from $\mathcal{T}$.

% MIP-based approaches
Multi-objective optimization has also been studied in the context of mixed integer programming and zero-one-programming~\autocite{Ehrgott2005-6,Rasmussen1986,DBLP:journals/eor/AlvesC07}.
There are different algorithmic approaches in this field:
based on the Simplex algorithm~\autocite{Ehrgott2005-7,DBLP:journals/mp/EvansS73}, branch-and-bound approaches~\autocite{Adelgren2021,DBLP:journals/siamjo/SantisENR20} and based on reducing it to a different mixed integer programming problem~\autocite{DBLP:journals/jota/Sun17,DBLP:journals/ol/LuMS20,Soland1979}.

% Why we don't compare to these approaches
Our focus in this work is to develop a MaxSAT-based approach for solving bi-objective optimization problems, especially suited for problems naturally represented in propositional logic.
As such, we only compare to approaches solving the same problem (presented in \cref{sec:sat-based}) in our empirical evaluation.

\subsection{Probabilistic and Meta-Heuristic Approaches\label{sec:approximative}}

% Approximative vs. exact methods
Other than the exact algorithms presented so far, there are also probabilistic and mate-heuristic search algorithms for multi-objective optimization problems~\autocite{Saini2021}.
These algorithms are \emph{not} guaranteed to return the exact pareto front, but they will return good solutions that might be sufficiently close approximations.
Probabilistic and meta-heuristic algorithms are mainly used for problems that are too large to solve them with exact methods under given research constraints.
Comparing exact with approximative approaches does not necessarily give much insight, however for the sake of completeness we are giving a short overview of these approximative methods as well.

% Simulated annealing
The first category of approximative algorithms is the family of simulated annealing algorithms~\autocite{Saini2021}.
These probabilistic and meta-heuristic algorithms model the annealing process that is used to increase the quality of crystalline structure in metals.
It was first proposed in~\textcite{DBLP:journals/science/KirkpatrickGV83} and extended to multiple objectives in~\textcite{DBLP:journals/tec/BandyopadhyaySMD08}.
An example of an algorithm building on multi-objective simulated annealing can be found in~\textcite{DBLP:journals/isci/SenguptaS18}.

% Evolutionary algorithms
A second big group of algorithms are evolutionary algorithms~\autocite{Saini2021,Dasgupta2013}.
These meta-heuristic algorithms hold a set of candidate solutions referred to as \emph{population}, which is then developed over multiple iterations to approximate the pareto front;
this process simulates evolution in nature, hence the name.
Algorithms for this category were presented for example in~\textcite{DBLP:journals/jgo/StornP97,DBLP:journals/tec/DebAPM02}.
\chapter{The \algname{} Algorithm\label{chap:approach}}

% Signposting
In this chapter, we detail \algname{}, the MaxSAT-based approach to bi-objective optimization developed in this work together with its variants.
\Cref{sec:algorithm} gives an overview of the algorithmic framework, while \cref{sec:variants} presents five specific instantiations of one of the subroutines, based on established MaxSAT algorithms.
To conclude the chapter, in \cref{sec:refinements} we discuss some refinements to \algname{}.

% BiOptSat pseudo-code
\begin{algorithm}[t]
  \caption{\algname{}: MaxSAT-based  bi-objective optimization} %\TODO{Alternatively ``enumeration of'', but then it sounds like full enumeration again.}}
  \label{alg:base-algorithm}
  \textbf{Input}: A formula $\formula$, two objectives $\Obj_\inc$ and $\Obj_\dec$.\\
  \textbf{Output}: Either a single representative for each Pareto point of $\formula$ or the full Pareto front.

  \begin{algorithmic}[1]
    \STATE $\texttt{InitSATsolver}(F)$ \label{l:init-solv} 
    \STATE $(\res, \sol^r) \gets \satsolver(\emptyset)$ \quad\COMMENT{Invokes the SAT solver on the formula.}  \label{l:sols} 
    \IF{$\res=\unsat$}
      \STATE \textbf{return} ``no solutions''
    \ENDIF
    \STATE $b_\dec \gets \infty, b_\inc \gets -1$ \label{l:bounds}
    \WHILE{$\res = \sat$} \label{l:loopstart}
      \STATE $(b_\inc, \sol^r) \gets \Min(b_\dec , b_\inc, \Obj_\inc(\sol^r))$  \quad\COMMENT{Maintains $\tot(\Obj_\inc)$ (or similar)}\label{l:minim1}
      \STATE $(b_\dec, \sol^r) \gets  \Simpr(b_\inc , \Obj_\dec(\sol^r))$  \quad\COMMENT{Builds $\tot(\Obj_\dec)$}\label{l:minim2}
      \STATE \textbf{yield} $\sol^r$  \quad\COMMENT{Optionally:  \textbf{yield} $\E(b_\dec, b_\inc)$} \label{ln:stage3} 
      \STATE $(\res, \sol^r) \gets \satsolver(\{\ov{\Obj_\dec}{b_\dec}\})$\label{l:endL}
    \ENDWHILE
  \end{algorithmic}
\end{algorithm}

\TODO{Make sure to differentiate between $\sol^r$ from the pseudocode and other $\sol$s.}

\section{Overview of the Algorithm\label{sec:algorithm}}

% The framework and lexicographic method
\Cref{alg:base-algorithm} details the \algname{} framework for computing the Pareto-optimal solutions of a CNF formula $\formula$ w.r.t.\ two objectives $\Obj_\inc$ and $\Obj_\dec$.
\algname{} is an instantiation of the general lexicographic optimization method~\autocite{survey} instantiated with a SAT solver.
To find a Pareto-optimal solution, the lexicographic method for multi-objective optimization creates multiple iterative single-objective optimization problems minimizing the objectives in order, where the latter calls are under the additional constraint to not worsen the previously minimized objectives.
Once a first Pareto-optimal solution is found, by adding a constraint that one of the objectives needs to be improved.
The lexicographic method will enumerate all Pareto-optimal solutions in monotonically increasing order of the first objective.
Following \cref{prop:biobjective}, for bi-objective optimization, this means that the solutions are in decreasing order for the second objective.
For this reason, we call objective $\Obj_\inc$ \emph{increasing} and $\Obj_\dec$ \emph{decreasing}.

% Single SAT solver and high-level overview
In \algname{}, the lexicographic method is instantiated with a single SAT solver that all subroutines make use of.
This single solver instantiation is invoked incrementally and preserved (i.e., not reset) during the whole search. 
\algname{} maintains the bounds $b_\inc$ and $b_\dec$ on the two objectives $\Obj_\inc$ and $\Obj_\dec$, respectively.
In each iteration, the value of $b_\inc$ is set to the smallest value for which there is a still-undiscovered Pareto-optimal solution $\sol^o$ for which $\Obj_\inc(\sol^o) = b_\inc$ by the $\Min$ procedure.
The value of $b_\dec$ is then set to $\Obj_\dec(\sol^o)$ by the $\Simpr$ procedure.
In the default configuration shown in \cref{alg:base-algorithm}, \algname{} solves the task of finding a single representative per Pareto point.
In case one wishes to enumerate all Pareto-optimal solutions, the \E{} procedure then enumerates all Pareto-optimal solutions $\sol^o$ for which $\Obj_\inc(\sol^o) = b_\inc$ and $\Obj_\dec(\sol^o) = b_\dec$.
We note that finding a single Pareto-optimal solution can simply be done by terminating the algorithm as soon as the first Pareto-optimal solution is returned.

% Initializing the algorithm
In detail, given a formula $\formula$ and two objectives $\Obj_\inc$ and $\Obj_\dec$, the search of \algname{} in \cref{alg:base-algorithm} starts by initializing a SAT solver with all clauses in $\formula$ on \cref{l:init-solv}.
Satisfiability (i.e., the existence of any Pareto-optimal solutions) is checked by invoking the SAT solver on its internal formula without assumptions via the $\satsolver(\emptyset)$ function (\cref{l:sols}).
Here, $\satsolver(\assumps)$ denotes an incremental invocation of the SAT solver initialized on \cref{l:init-solv}, with the set of assumptions $\assumps$.
It has three return parameters, $\res,\sol,\core$ where the first is either \sat{} or \unsat{}, indicating if the internal formula is satisfiable with the given assumptions.
If $\res=\sat$, $\sol$ is populated by a satisfying assignment and $\core$ is not modified;
if $\res=\unsat$, $\core$ is populated with an unsatisfiable core and $\sol$ is not modified.
In case the returned core of a specific call is not used, we will omit it as a return parameter.
On \cref{l:sols}, if $\res$ is \unsat{}, the formula has no Pareto-optimal solutions and the algorithm terminates.
Otherwise, $\sol$ is an assignment that satisfies the formula.
Before the main enumeration procedure starts, the bounds $b_\inc$ and $b_\dec$ on $\Obj_\inc$ and $\Obj_\dec$ are set to $-1$ and $\infty$, respectively.
This is because those values are known to be smaller, respectively greater, than they will take in the first iteration.

% Main loop: minimization of increasing objective
The main search loop (\crefrange{l:loopstart}{l:endL}) iterates as long as there are Pareto-optimal solutions of $\formula$ that have not been enumerated yet. 
This is the case if there is a solution $\sol$ for which $\Obj_\dec(\sol) < b_\dec$, checked by invoking the SAT solver under the assumption $\ov{\Obj_\dec}{b_\dec}$ on \cref{l:endL}.
In the beginning of each main loop iteration, the procedure $\Min$ is employed to minimize the increasing objective, i.e., compute the smallest value $b_\inc$ for which there is a solution $\sol$ for which $\Obj_\dec(\sol) < b_\dec$ and $\Obj_\inc(\sol) = b_\inc$  (\cref{l:minim1}). 
The parameters of the $\Min$ procedure are the bound $b_\dec$ that the decreasing objective of the found solution needs to be below, $b_\inc$ as a known lower and $\Obj_\inc(\tau^r)$ as a known upper bound on the minimum increasing objective value.
We assume that $\Min$ maintains a way to enforce that $\Obj_\inc(\sol) < k$, e.g., through a totalizer $\tot(\Obj_\inc)$, and that \algname{} and all of its subroutines have access to a set of assumptions for enforcing this bound for any $k$.
Details of different instantiations of the \Min{} subroutine are discussed in \cref{sec:variants}.

% Main loop: minimization of decreasing objective
Next, the algorithm employs \emph{solution-improving search}~\autocites{handbook2-maxsat,DBLP:journals/jsat/BerreP10,DBLP:journals/jsat/EenS06} to minimize the decreasing objective, i.e., to compute the smallest $b_\dec$ for which there is a solution $\sol$ for which $\Obj_\dec(\sol) = b_\dec$ and $\Obj_\inc(\sol) = b_\inc$ (\cref{l:minim2}).
The totalizer $\tot(\Obj_\dec, \Obj_\dec(\sol^r))$ is built the first time this subroutine is invoked.
Building the totalizer at this point allows for only building it up to bound $\Obj_\dec(\sol^r)$, since all Pareto-optimal solutions are known to have at most that value for $\Obj_\dec$.
Solution-improving search works by---starting from the known upper bound $k=\Obj_\dec(\sol^r)$---iteratively invoking the SAT solver under the assumptions $\{\ov{\Obj_\dec}{k}, \ove{\Obj_\inc}{b_\inc}\}$ for decreasing values of $k$ until the solver reports \unsat{}.
As soon as unsatisfiability is reached, $\Simpr$ returns $b_\dec=k+1$ and the last satisfying solution $\sol^r$ for which $\Obj_\dec(\sol^r) = b_\dec$ and $\Obj_\inc(\sol^r) = b_\inc$.
At this point, we know that there is no solution of $\formula$ that dominates $\sol^r$, so $\sol^r$ is returned as Pareto-optimal on \cref{ln:stage3}.
If one wants to enumerate all solutions $\sol^o$ that correspond to the Pareto point $(b_\inc,b_\dec)$, the \E{} procedure repeatedly invokes the SAT solver with the assumptions $\{\ove{\Obj_\dec}{b_\dec}, \ove{\Obj_\inc}{b_\inc}\}$ and blocks each found solution until no more solutions are found.

% Main algorithm example
\begin{example}\label{ex:main-iteration}
  Invoke \algname{} on the formula $\formula$ and objectives $\Obj_\inc$, $\Obj_\dec$ detailed in \cref{fig:search-trace}. 
  The search starts by invoking a SAT solver on $\formula$.
  This call returns a feasible solution, say $\sol^c_1 = \solcone$ for which $\Obj_\inc(\sol^c_1) = \Obj_\dec(\sol^c_1) = 3$. 
  The first iteration of the main search loop starts with a call to \Min{}.
  This returns $b_\inc = 1$ and, e.g., the solution $\sol^c_3 = \solcthree$ for which $\Obj_\inc(\sol^c_3) = 1$ and $\Obj_\dec(\sol^c_3) = 3$.
  \algname{} then proceeds to the \Simpr{} subroutine that initializes a totalizer $\tot(\Obj_\dec, 3)$.
  The first call to the SAT solver is made with the assumptions $\assumps = \{ \ove{\Obj_\inc}{1}, \ov{\Obj_\dec}{4} \}$.
  The result is satisfiable;
  say that the solver returns the solution $\sol^o_1 = \soloone$.
  Then, the solver is invoked with the assumptions $\assumps =  \{ \ove{\Obj_\inc}{1}, \ov{\Obj_\dec}{3} \}$.
  The result is unsatisfiable, so the procedure returns the Pareto-optimal $\sol^o_1$ and $b_\dec = \Obj_\dec(\sol^o_1) = 3$.
  Now optionally, the procedure \E{} can be used to enumerate all other solutions corresponding to the Pareto point $(\Obj_\inc(\sol^o_1),\Obj_\dec(\sol^o_1))$.
  At the end of the iteration, the SAT solver is queried with the assumption $\{ \ov{\Obj_\dec}{2} \}$.
  As the result is \sat{} and the solver returns, e.g., the solution $\sol^c_4 = \solcfour$,
  the algorithm starts a new iteration. \\
  The next iteration of \algname{} proceeds similarly to the first.
  The procedure \Min{} returns $b_\inc = 2$ and, e.g., the solution $\sol^o_2 = \solotwo$.
  \Simpr{} cannot improve on the decreasing objective, so the solution $\sol^o_2$ is proven to be Pareto-optimal.
  At the end of the iteration, on \cref{l:endL} the SAT solver is invoked with the assumption $\{\ov{\Obj_\dec}{2}\}$ which returns \sat{} and, e.g., the solution $\sol^o_3 = \solothree$. \\
  The last iteration starts by calling \Min{} which returns $b_\inc = 3$ and, e.g., again the solution $\sol^o_3$.
  \Simpr{} again cannot improve on the decreasing objective, so $\sol^o_3$ is also Pareto-optimal.
  Lastly, the SAT solver is queried with the assumption $\{\ov{\Obj_\dec}{1}\}$.
  The solver returns unsatisfiable, terminating the algorithm. 
\end{example}

\section{Variants for Minimizing the Increasing Objective\label{sec:variants}}

% Signposting
We consider five different instantiations of the \Min{} procedure for minimizing the increasing objective.
The first four (\satunsat{}, \unsatsat{}, \msu{} and \oll{}) are inspired by existing MaxSAT algorithms while the last (\msh{}) switches between two MaxSAT-like algorithms to combine their advantages.

\TODO{Point out which of the \Min{} parameters the instatiations are using.}

\subsection{\satunsat{}\label{sec:sat-unsat}}

% SAT-UNSAT pseudo-code
\begin{algorithm}[t]
  \caption{\satunsat{} instantiation of \Min{}}\label{alg:sat-unsat}
  \textbf{Input}: Last bound $b_\dec$ on $\Obj_\dec$ and known upper bound $k$ on minimum of $\Obj_\inc$. \\
  \textbf{Output}: Solution $\sol^r$ and smallest $k=\Obj_\inc(\sol^r)$ so that $\Obj_\dec(\sol^r)<b_\dec$.

  \begin{algorithmic}[1]
    \STATE build or extend $\tot(\Obj_\inc,k)$ if necessary \label{ln:su-tot}
    \STATE $(\res,\sol^r) \gets \satsolver(\{\ov{\Obj_\dec}{b_\dec}, \ov{\Obj_\inc}{k}\})$
    \WHILE{$\res = \sat$}
      \STATE $k \gets \Obj_\inc(\sol^r)$
      \STATE $(\res,\sol^r) \gets \satsolver(\{\ov{\Obj_\dec}{b_\dec}, \ov{\Obj_\inc}{k}\})$ \label{ln:su-query}
    \ENDWHILE
    \STATE \textbf{return} $(k,\sol^r)$ \label{ln:su-ret}
  \end{algorithmic}
\end{algorithm}

% Solution-improving search and input parameters
\satunsat{} is a variant of solution-improving search~\autocites{handbook2-maxsat,DBLP:journals/jsat/BerreP10,DBLP:journals/jsat/EenS06} that is also used for minimizing $\Obj_\dec$. 
As inputs, the procedure only gets the last bound $b_\dec$ on $\Obj_\dec$ and the upper bound $k=\Obj_\inc(\sol^r)$ on the minimum of the increasing objective known from the last SAT solver call.
The known lower bound is not needed for this variant.
Since the last call is made on \cref{l:endL} with the assumption $\ov{\Obj_\dec}{b_\dec}$, the solution $\sol^r$ will have $\Obj_\dec(\sol^r) < b_\dec$. 

% Details of SAT-UNSAT
How \satunsat{} progresses is outlined in~\cref{alg:sat-unsat}.
The procedure maintains the totalizer $\tot(\Obj_\inc)$ and begins on \cref{ln:su-tot} by checking, if the current upper bound on that totalizer is at least $k$, extending it if not. 
Then the SAT solver is iteratively invoked with the assumptions $\{\ov{\Obj_\dec}{b_\dec}, \ov{\Obj_\inc}{k}\}$ for decreasing values of $k$ (\cref{ln:su-query}).
The procedure terminates when the result is \unsat{}, after which on \cref{ln:su-ret}, the value of $k$ and the solution obtained during the final satisfiable call are returned as $b_\inc$ and $\sol^r$.  

% Example: SAT-UNSAT
\begin{example}\label{ex:satunsat}
  Consider the invocation of \algname{} detailed in \cref{ex:main-iteration}. 
  We detail the invocation of \Min{} instantiated as \satunsat{}. 
  The full progression of the search of \algname{} with \Min{} instantiated as \satunsat{} is illustrated in \cref{fig:search-trace}.
  In the first iteration, \satunsat{} is invoked with $b_\dec = \infty$ and $\Obj_\inc(\sol^c_1) = 4$.
  At this point, the totalizer over $\Obj_\inc$ has not been built, so the procedure starts by adding $\tot(\Obj_\inc, 3)$ to the solver.
  The first call to the SAT solver is made with the assumptions $\{\ov{\Obj_\inc}{4}\}$, since $b_\dec = \infty$ and therefore no assumption constraining $\Obj_\dec$ is needed.
  The result is satisfiable, the solver returns, e.g., the solution $\sol^c_2 = \solctwo$. 
  In the next iteration, the set of assumptions is $\{\ov{\Obj_\inc}{2}\}$.
  The result is again satisfiable, returning, e.g., the solution $\sol^c_3 = \solcthree$.
  The SAT solver is then invoked with the assumptions $\{\ov{\Obj_\inc}{1}\}$.
  Now the result is \unsat{} so the procedure terminates and returns $b_\inc = 1$ and $\sol^c_3$. \\
  In the second iteration of \algname{}, \satunsat{} is invoked with $b_\dec = 3$ and $\Obj_\inc(\sol^c_4) = 3$.
  The first call to the SAT solver is made with the assumptions $\{ \ov{\Obj_\dec}{3}, \ov{\Obj_\inc}{3} \}$.
  The result is \sat{} and the solver returns, e.g., the solution $\sol^o_2 = \solotwo$.
  \satunsat{} invokes the SAT solver again with the assumptions $\{ \ov{\Obj_\dec}{3}, \ov{\Obj_\inc}{2} \}$.
  The result is \unsat{}, so the procedure returns $b_\inc = 2$ and $\sol^o_2$. \\
  In the third (and last) iteration of \algname{}, \satunsat{} is invoked with $b_\dec = 2$ and $\Obj_\inc(\sol^o_3) = 3$.
  The SAT solver is queried with the assumptions $\{ \ov{\Obj_\dec}{2}, \ov{\Obj_\inc}{3} \}$ and returns \unsat{}.
  \satunsat{} therefore returns $b_\inc = 3$ and $\sol^o_3$.
\end{example}

\subsection{\unsatsat{}\label{sec:unsat-sat}}

% UNSAT-SAT pseudo-code
\begin{algorithm}[t]
  \caption{\unsatsat{} instantiation of \Min{}}\label{alg:unsat-sat}
  \textbf{Input}: Last bound $b_\dec$ on $\Obj_\dec$ and last bound $b_\inc$ on $\Obj_\inc$. \\
  \textbf{Output}: Solution $\sol^r$ and smallest $k=\Obj_\inc(\sol^r)$ so that $\Obj_\dec(\sol^r)<b_\dec$.

  \begin{algorithmic}[1]
    \STATE $k \gets b_\inc$ \label{ln:us-boundinit}
    \STATE build or extend $\tot(\Act,k+1)$
    \STATE $(\res,\sol^r) \gets \satsolver(\{\ov{\Obj_\dec}{b_\dec}, \ove{\Obj_\inc}{k+1}\})$
    \WHILE{$\res = \unsat$}
      \STATE $k \gets k+1$
      \STATE extend $\tot(\Obj_\inc,k+1)$ \label{ln:us-tot}
      \STATE $(\res,\sol^r) \gets \satsolver(\{\ov{\Obj_\dec}{b_\dec}, \ove{\Obj_\inc}{k+1}\})$ \label{ln:us-query}
    \ENDWHILE
    \STATE \textbf{return} $(k+1,\sol^r)$ \label{ln:us-ret}
  \end{algorithmic}
\end{algorithm}

% UNSAT-SAT and input parameters
\unsatsat{} takes a similar approach to \satunsat{} search but searches for the smallest value by lower-bounding instead of upper-bounding~\autocite{}.
As input parameters it receives the last bound $b_\dec$ on $\Obj_\dec$ and the last bound $b_\inc$ on $\Obj_\inc$ as a lower bound on the sought-after smallest value.
The upper bound on the smallest value is not needed for this variant.
\unsatsat{} also maintains a totalizer $\tot(\Obj_\inc)$.

% Details of UNSAT-SAT
The \unsatsat{} instantiation of \Min{} proceeds as illustrated in \cref{alg:unsat-sat}.
On \cref{ln:us-boundinit}, the bound $k$ is set to the known lower bound $b_\inc$ and the solver is then iteratively queried on \cref{ln:us-query} under the assumptions $\{ \ove{\Obj_\inc}{k+1}, \ov{\Obj_\dec}{b_\dec} \}$.
If the solver returns unsatisfiable, the bound $k$ is increased by $1$ and the solver is queried again.
The search ends once the solver returns satisfiable and in this case, the solution, and the bound are returned on \cref{ln:us-ret}.
Since the bound of this lower bounding search procedure will only monotonically increase, it is enough if the totalizer $\tot(\Obj_\inc)$ is at every step built up to the bound $k+1$ (\cref{ln:us-tot}) and extended to the next bound in the next iteration.
This way, the SAT solver is always loaded with a minimum number of clauses.

% Example: UNSAT-SAT
\begin{example}
  Consider the invocation of \algname{} detailed in \cref{ex:main-iteration}. 
  Here we detail the invocation of \Min{} instantiated as \unsatsat{}.
  In the first iteration, \unsatsat{} is invoked with $b_\dec = \infty$ and $b_\inc=-1$.
  At this point, the totalizer over $\Obj_\inc$ has not been built, so the procedure starts by initializing $\tot(\Obj_\inc, 0)$ and invokes the SAT solver with the assumptions $\{\ove{\Obj_\inc}{0}\}$.
  The result is \unsat{}, so the totalizer is extended to $\tot(\Obj_\inc, 1)$ and the SAT solver invoked with the assumptions $\{ \ove{\Obj_\inc}{1}\}$.
  The result is \sat{} and the procedure returns, e.g., $b_\inc = 1$ and $\sol^c_3 = \solcthree$. \\
  In the second iteration of \algname{}, \unsatsat{} is invoked with $b_\dec = 3$ and $b_\inc = 1$.
  The totalizer is extended to $\tot(\Obj_\inc, 2)$ and the solver is invoked with the assumptions $\{\ove{\Obj_\inc}{2}, \ov{\Obj_\dec}{3}\}$.
  The result is \sat{}, and the routine returns, e.g., $b_\inc = 2$ and $\sol^o_2 = \solotwo$. \\
  In the last iteration of \algname{}, \unsatsat{} is invoked with $b_\dec = 2$ and $b_\inc = 2$.
  The totalizer is extended to $\tot(\Obj_\inc, 3)$ and the solver is invoked with the assumptions $\{\ove{\Obj_\inc}{3}, \ov{\Obj_\dec}{2}\}$.
  The result is \sat{}, and \unsatsat{} returns, e.g., $b_\inc = 3$ and $\sol^o_3 = \solothree$.
\end{example}

\subsection{\msu{}\label{sec:msu}}

% MSU3 pseudo-code
\begin{algorithm}[t]
  \caption{\msu{} instantiation of \Min{}}\label{alg:msu}
  \textbf{Input}: Last bound $b_\dec$ on $\Obj_\dec$ and last bound $b_\inc$ on $\Obj_\inc$. \\
  \textbf{Output}: Solution $\sol^r$ and smallest $k=\Obj_\inc(\sol^r)$ so that $\Obj_\dec(\sol^r)<b_\dec$.

  \begin{algorithmic}[1]
    \STATE $k \gets \max\{b_\inc,0\}$
    \STATE $(\res,\sol^r,\core) \gets \satsolver(\{\ov{\Obj_\dec}{b_\dec}, \ove{\Obj_\inc}{k}\} \cup \{\lnot l \mid l \in \Obj_\inc \setminus \Act\})$ \label{ln:msu-firstquery}
    \WHILE{$\res = \unsat$}
      \STATE $k \gets k+1$ \label{ln:msu-inc}
      \STATE $\core \gets \core \setminus \{\lnot\ov{\Obj_\dec}{b_\dec}, \lnot\ove{\Obj_\inc}{k}\}$
      \STATE $\Act \gets \Act \cup \core$
      \STATE build or extend $\tot(\Act,k)$ \label{ln:msu-tot}
      \STATE $(\res,\sol^r,\core) \gets \satsolver(\{\ov{\Obj_\dec}{b_\dec}, \ove{\Obj_\inc}{k}\} \cup \{\lnot l \mid l \in \Obj_\inc \setminus \Act\})$ \label{ln:msu-mainquery}
    \ENDWHILE
    \STATE \textbf{return} $(k,\sol^r)$
  \end{algorithmic}
\end{algorithm}

% MSU3 and input parameters
\msu{} implements a core-guided approach inspired from the MSU3 MaxSAT algorithm~\autocites{DBLP:journals/corr/abs-0712-1097}.
The input parameters of this subroutine are the last bound $b_\dec$ on $\Obj_\dec$ and the last bound $b_\inc$ on $\Obj_inc$ as a lower bound on the sought-after smallest value.
The upper bound on the smallest value is not needed for this variant.

% Details of MSU3
The \msu{} instantiation does not maintain a totalizer $\tot(\Obj_\inc)$, but a set $\Act \subset \Obj_\inc$ of \emph{active} objective literals and a totalizer $\tot(\Act)$ built over them. 
Initially, $\Act = \emptyset$, i.e., all literals of $\Obj_\inc$ are inactive.
Informally speaking, an inactive literal $l \in \Obj_\inc \setminus \Act$ is assumed to the value $0$ in every invocation of the SAT solver until it is returned as part of a core.
\Cref{alg:msu} illustrates the search performed by \msu{}.
The algorithm starts from the value $k=b_\inc$ computed in the previous iteration and invokes the SAT solver with the assumptions $\assumps = \{\ove{\Act}{k}, \ov{\Obj_\dec}{b_\dec}  \} \cup \{ \lnot l \mid l \in \Obj_\inc \setminus \Act\}$ on \cref{ln:msu-firstquery}.
If the result is unsatisfiable, the SAT solver returns a core $\core \subset \{\lnot l \mid l \in \assumps\}$.
Next, the bound $k$ is increased by one, the inactive literals in $\core$ are added to $\Act$ and the totalizer $\tot(\Act)$ is extended (\crefrange{ln:msu-inc}{ln:msu-tot}).
The procedure continues until the SAT solver returns satisfiable, and a solution $\sol^r$ which has $\Obj_\inc(\sol^r) = k$ and $\Obj_\dec(\sol^r) < b_\dec$ is found.
At that point the value $k$ is the minimum value $\Obj_\inc(\sol)$ for any solution $\sol$ subject to $\Obj_\dec(\sol) < b_\dec$.
This is because the value of $k$ is increased monotonically, and the solver returned unsatisfiable in the second-to-last iteration. 

% Enforcing an upper bound on O_I in other places and proof of correctness
For enforcing $\ove{\Obj_\inc}{k}$ when employing \msu{}, consider an invocation of $\msu(b_\dec, b_\inc)$ made during \algname{} and assume it returns the tuple $(b_\inc, \sol^r)$. 
In the next call to \Simpr{}, the number of literals in $\Obj_\inc$ set to $1$ needs to be restricted to at most $b_\inc$. 
Since the totalizer maintained by \msu{} only has $\Act \subset \Obj_\inc$ as inputs, we do not have access to an output literal of form  $\ove{\Obj_\inc}{b_\inc}$.
Instead, we use  the assumptions $\{ \ove{\Act}{b_\inc}\} \cup \{ \lnot l \mid l \in \Obj_\inc \setminus \Act \}$, i.e., restrict the number of literals in $\Act$ set to $1$ to $b_\inc$ and assume the value of each inactive literal $l \in \Obj_\inc \setminus \Act$ to $0$. 
The following observation proves that doing so can be done without removing any Pareto-optimal solutions from the search. 
\begin{observation}\label{obs:sound}
  Let $\sol^o$ be a Pareto-optimal solution of $\formula$ for which $\Obj_\inc(\sol^o) = b_\inc$.
  Then $\sol^o(l) = 0$ for all $l \in \Obj_\inc \setminus \Act$. 
\end{observation}
\begin{proof}(Sketch)
  Since, $b_\inc$ was returned by \msu{}, we know that there exists a Pareto-optimal $\sol^o$ for which $\Obj_\inc(\sol^o) = b_\inc$ and $\Obj_\dec(\sol^o) < b_\dec$.
  By the properties of cores, we also know that \emph{any} solution $\sol^s$ of $\formula$ for which $\Obj_\dec(\sol^s) < b_\dec$ assigns at least $b_\inc$ literals in $\Act$ to $1$.
  Thus, any $\sol^n$ that assigns $\sol^n(l) = 1$ for an inactive literal $l \in \Obj_\inc \setminus \Act$ will have $\Obj_\inc(\sol^n) > b_\inc$.
\end{proof}

% Example: MSU3
\begin{example}\label{ex:msu}
  Consider the invocation of \algname{} detailed in \cref{ex:main-iteration}. 
  Here we detail the invocations of \Min{} instantiated as \msu{}. 
  In the first iteration of \algname{}, \msu{} is invoked with $b_\dec =\infty$ and $b_\inc = -1$.
  Initially, the set $\Act = \emptyset$ of active literals is empty, so the first call to the SAT solver is made with the assumptions $\assumps =  \{ \lnot i_1, \allowbreak \lnot i_2, \allowbreak \lnot i_3, \allowbreak \lnot i_4\}$.
  The result is unsatisfiable and the solver returns, e.g., $\core = \{i_1, i_2\}$. 
  The literals in $\core$ are marked as active and the totalizer $\tot(\Act, 2)$ is initialized.
  The SAT solver is then invoked with the assumptions $\assumps = \{ \lnot i_3, \lnot i_4, \ove{\Act}{1}\}$. 
  The result is satisfiable so the procedure returns, e.g., $b_\inc = 1$ and the solution $\sol^c_3 = \solcthree$. \\
  In the next iteration of \algname{}, \msu{} is invoked with $b_\dec = 3$ and $b_\inc = 1$.
  The set $\Act = \{i_1, i_2\}$ is kept from the previous iterations, so the first call to the SAT solver is made with the assumptions $\assumps = \{ \ov{\Obj_\dec}{3}, \allowbreak \ove{\Act}{1}, \allowbreak \lnot i_3, \allowbreak \lnot i_4 \}$.
  The result is unsatisfiable;
  assume the core returned by the solver is $\core = \{ \lnot \ov{\Obj_\dec}{3}, \allowbreak \lnot \ove{\Act}{1}, \allowbreak i_3, \allowbreak i_4 \}$.
  The totalizer outputs $\lnot \ov{\Obj_\dec}{3}$ and $\lnot \ove{\Act}{1}$ are discarded, $i_3$ and $i_4$ are added to the active literals, and the totalizer is extended to $\tot(\Act, 3)$.
  The SAT solver is queried again with the assumptions $\assumps = \{\ov{\Obj_\dec}{2}, \ove{\Act}{2}\}$;
  the result is \sat{} and the returned solution, e.g., $\sol^o_2 = \solotwo$.
  \msu{} returns $b_\inc = 2$ and $\sol^o_2$. \\
  In the last iteration, \msu{} is invoked with $b_\dec = 2$ and $b_\inc = 2$.
  The SAT solver is queried with the assumptions $\assumps = \{\ov{\Obj_\dec}{2}, \ove{\Act}{2}\}$.
  The result is \unsat{};
  assume the core is $\core = \{ \lnot \ov{\Obj_\dec}{2}, \lnot \ove{\Act}{2} \}$.
  Both totalizer outputs are discarded, and the totalizer is extended to $\tot(\Act, 4)$.
  The solver is queried again with the assumptions $\assumps = \{\ov{\Obj_\dec}{2}, \ove{\Act}{3}\}$.
  The result is \sat{};
  assume the returned solution is $\sol^o_3 = \solothree$.
  \msu{} returns $b_\inc = 3$ and $\sol^o_3$.
\end{example}

\subsection{\oll{}\label{sec:oll}}

% High-level OLL
As briefly mentioned in \cref{sec:max-sat}, the OLL MaxSAT algorithm~\autocite{DBLP:conf/cp/MorgadoDM14} builds a cardinality constraint for each core extracted from the formula, treating the totalizer output as new soft clauses.
The same holds true for the \oll{} instantiation of \Min{}.
In each iteration, the assumptions given to the SAT solver consist of (i)~the inactive literals of $\Obj_\inc$, (ii)~the outputs of previously built totalizers corresponding to the lowest number of input literals that should be assigned to $1$ in any possible satisfying assignment and (iii)~the bound $\ov{\Obj_\dec}{b_\dec}$.
The procedure terminates when the SAT solver returns a solution $\sol$.
Similarly to \msu{}, the assumptions for enforcing a bound on $\Obj_\inc$ in the other subroutines of \cref{alg:base-algorithm} need to be adapted when using \oll{} by assuming the inactive literals of $l \in \Obj_{\inc} \setminus \Act$ to $0$.
Additionally, a set of assumptions over the totalizer outputs needs to be included.

\TODO{Probably extend}

\subsection{\msh{}\label{sec:hybrid}}

% MSHyrbid pseudo-code
\begin{algorithm}[t]
  \caption{\msh{} instantiation of \Min{}}\label{alg:msh}
  \textbf{Input}: Last bound $b_\dec$ on $\Obj_\dec$, known upper and lower bounds $k$ and $b_\inc$ on min.\ of $\Obj_\inc$. \\
  \textbf{Output}: Solution $\sol^r$ and smallest $k=\Obj_\inc(\sol^r)$ so that $\Obj_\dec(\sol^r)<b_\dec$.

  \begin{algorithmic}[1]
    \IF{$|\Act| < \thr \cdot |\Obj_\inc|$}
      \STATE $(b_\inc,\sol^r) \gets \msu{}(b_\dec, b_\inc)$ \COMMENT{Immediately terminates once $|\Act| < \thr \cdot |\Obj_\inc|$ is reached} \label{ln:msh-msu}
    \ENDIF
    \IF{$|\Act| \ge \thr \cdot |\Obj_\inc|$}
      \STATE fully build or extend $\tot(\Obj_\inc, k)$ if necessary \label{ln:msh-tot}
      \STATE $(b_\inc,\sol^r) \gets \satunsat{}(b_\dec, k)$ \label{ln:msh-su}
    \ENDIF
    \STATE \textbf{return} $(b_\inc,\sol^r)$ \label{ln:msh-ret}
  \end{algorithmic}
\end{algorithm}

% MSHybrid intuition
The final variant proposed in this work, \msh{}, is a hybrid between \msu{} and \satunsat{} with the following intuition:
if \msu{}  reaches the stage where all literals of the objective are active, its search will become equivalent to \unsatsat{}, meaning it is a lower bounding search where the bound on the totalizer $\tot(\Obj_\inc)$ is increased by one every iteration until the SAT query is satisfiable.
However, \satunsat{} search may be a significantly better approach compared to \unsatsat{}.
If this is the case, \msu{} might have an advantage over \satunsat{} as long as not all literals are active, but as soon as all literals are active, it looses its advantage.
Furthermore, if a problem instance has literals in $\Obj_\inc$ that are not constrained by $\formula$, these literals will never appear in any core making \msu{} behave like \unsatsat{} even before the totalizer is fully built.

% Details and parameters
With this intuition, we propose \msh{} as a hybrid variant that starts with \msu{} search and switches over to \satunsat{} as soon as a certain percentage of the literals in $\Obj_\inc$ have been added to the totalizer $\tot(\Act)$.
The subroutine that takes the last bound $b_\dec$ on $\Obj_\dec$, the known upper $k=\Obj_\inc(\sol^r)$ and lower bounds $b_\inc$ on the minimum of $\Obj_\inc$ as parameters is outlined in \cref{alg:msh}.
Additionally, \msh{} has a configuration $\thr$ that defines at what percentage of the literals in $\Obj_\inc$ being active it switches from \msu{} to \satunsat{}.
Initially, \msh{} will execute \msu{} on \cref{ln:msh-msu}.
If \msu{} finds the minimum without the condition $|\Act| < \thr \cdot |\Obj_\inc|$ being met, the found minimum $b_\inc$ and corresponding solution $\sol^r$ will be returned on \cref{ln:msh-ret}.
In case during the execution of \msu{} $|\Act| < \thr \cdot |\Obj_\inc|$ is met, \msu{} will immediately be terminated.
On \cref{ln:msh-tot}, $\tot(\Obj_\inc,k)$ will then be fully built from the existing $\tot(\Act)$ and \satunsat{} invoked on \cref{ln:msh-su}.
From now on, every call to \msh{} will call \satunsat{} on \cref{ln:msh-su}.
With this, the advantages of both \msu{} and \satunsat{} can in the best case be combined.
A similar approach of combining core-guided search is known in incomplete MaxSAT solving as core-boosted linear search~\autocite{DBLP:conf/cpaior/BergDS19}.

% Example: MSHybrid
\begin{example}
  Consider the invocation of \algname{} detailed in \cref{ex:main-iteration}.
  We detail the invocations of \Min{} instantiated as \msh{}.
  Since \msh{} starts out as \msu{}, the first invocation follows the description in \cref{ex:msu}. \\
  Assume \msh{} is configured to switch as soon as 70\% of the literals in $\Obj_\inc$ are active ($\thr = 0.7$).
  Since after the first iteration of \algname{} we have $\Act = \{i_1, i_2\}$, the second invocation of \msh{} also starts as \msu{} since less than 70\% of the literals in $\Obj_\inc$ are active.
  As soon as $i_3$ and $i_4$ become active, with the first core in the second invocation of \msu{}, the \msu{} subroutine is terminated since the threshold for switching to \satunsat{} is reached.
  Because all literals in $\Obj_\inc$ are already active in this example and therefore included in $\tot(\Act)$, the totalizer does not need to be extended but \satunsat{} can directly be invoked as in the second iteration outlined in \cref{ex:satunsat}. \\
  In the third iteration of \algname{}, \msh{} will directly invoke \satunsat{}, which proceeds as described in the third iteration in \cref{ex:satunsat}.
\end{example}

\section{Refinements to \algname{}\label{sec:refinements}}

\subsection{Lazily Building $\tot(\Obj_\dec)$}

% Lazily building totalizer for decreasing objective
Assume \algname{} is invoked on a formula $\formula$ and a pair of overlapping objectives $\Obj_\inc$ and $\Obj_\dec$ for which $\Obj_\inc \cap \Obj_\dec \neq \emptyset$ with \Min{} instantiated as \msu{} or \oll{}.
Let $\Act$ be the set of active literals of $\Obj_\inc$ as maintained by \Min{}.
Lazy building of $\tot(\Obj_\dec)$ refers to only having $(\Obj_\dec \setminus \Obj_\inc) \cup  (\Act \cap \Obj_\dec)$ as input to the totalizer (incrementally extending the totalizer as the set $\Act$ grows), and assuming the value of each literal $l \in (\Obj_\dec \cap \Obj_\inc) \setminus \Act$ to $0$ in each SAT call made during invocations of \Simpr{}.
The soundness of doing so follows by an argument very similar to the one we made in \cref{obs:sound}.
Essentially, the properties of cores imply that the Pareto-optimal solutions $\sol^o$ of $\formula$ for which $\Obj_\inc(\sol^o) = b_\inc$ assign $\sol^o(l) = 0$ for all  $l \in (\Obj_\dec \cap \Obj_\inc) \setminus \Act$. 

% Why are we doing this?
The idea behind this refinement is to build the totalizer only over literals that can be assigned to 1.
In this, unnecessary clauses are removed from the working formula of the SAT solver.

% Adapting assumptions
Lazy building of $\tot(\Obj_\dec)$ requires a minor adaption to the termination criterion of \cref{alg:base-algorithm}.
More specifically, as the totalizer maintained by \Simpr{} might not have all literals of $\Obj_\dec$ as inputs, the algorithm does not have a (straight-forward) way of checking if there exists a solution $\sol$ for which $\Obj_\dec(\sol) < b_\dec$.
However, the lack of further Pareto-optimal solutions is instead detected in the next call to \Min{} by the SAT solver returning an empty core, or more precisely, a subset of assumptions that does not contain any inactive literals nor outputs of $\tot(\Obj_\inc)$.

\subsection{Blocking of Dominated Solutions}

% Blocking solutions dominated by candidates
Every time in the search procedure that a candidate solution $\sol$ with objective values $b_\inc = \Obj_\inc(\sol)$ and $b_\dec = \Obj_\dec(\sol)$ is found, the definition of a Pareto-optimal point leads to the conclusion that all solutions $\sol^d$ with $\Obj_\inc(\sol^d) > b_\inc$ and $\Obj_\dec(\sol^d) > b_\dec$ cannot be Pareto-optimal.
These points can all be blocked by adding the clause $\{ \ove{\Obj_\inc}{b_\inc}, \ove{\Obj_\dec}{b_\dec} \}$ to the solver.
Adding this refinement might help prune the search space quicker and therefore speed up finding Pareto-optimal solutions.

\subsection{Domain-Specific Solution Blocking}

% Naive and domain-secific blocking
If multiple representatives of the same Pareto point are of interest, the procedure \E{} needs to block all obtained solutions in order to enumerate all solutions corresponding to the same Pareto point. 
Naively, a found solution $\sol$ can be blocked with the clause $\{ \lnot v \mid v \in \sol(v) \cap \formula^\text{orig} \}$, where $\formula^\text{orig}$ is the original formula of the instance without any clauses added by the algorithm.
An improved by still generic way of blocking a solution is to build a blocking clause over all variables that the SAT solver decisions for. \TODO{is this correct and enough?}
In later sections we also give examples of how domain-specific knowledge can be used in order to derive stronger clauses that block not only a specific solution obtained, but also other, symmetric solutions.

\subsection{Bound Hardening}

% Hardening bounds for O_D and why we don't do it for O_I
Because the values for $\Obj_\dec(\sol^r)$ monotonically decrease while the algorithm progresses, this information can be passed on to the SAT solver for it to be able to draw logical conclusions from it.
This is done by the literal $\ove{\Obj_\dec}{k}$ as a unit clause, as soon as the algorithm has verified that all remaining Pareto-optimal solutions $\sol^o$ have $\Obj_\dec(\sol^o)\le k$.
Something similar could be done for $\Obj_\inc$, since it is known that the corresponding objective values will monotonically increase.
However, as mentioned in \cref{sec:card-const}, the way \algname{} uses totalizers, only upper bounds can be enforced with them.
For this reason, bound hardening is only done for the decreasing objective.

\subsection{Refinements to Core-Guided Variants}

Our implementation of the variants \algname{} with \msu{} or \oll{} make use of refinements commonly used in core-guided MaxSAT solving.
More specifically, we employ core minimization~\autocite{DBLP:journals/jsat/IgnatievMM19} (either exact or heuristic) and core-exhaustion~\autocites{DBLP:journals/jsat/IgnatievMM19,DBLP:conf/cp/AnsoteguiBGL13}.
Additionally, we consider a disjoint core extraction phase~\autocite{DBLP:conf/cp/DaviesB11}.
Given a core $\core$ returned by the SAT solver, heuristic core minimization refers to reinvoking the SAT solver with $\{\lnot l \mid l \in \core\}$ as the assumptions hoping that the solver returns a smaller set of assumptions.
Exact core minimization refers to iteratively finding a minimal unsatisfiable subset by attempting to remove each assumption separately.
Core exhaustion is an OLL-specific technique that seeks to improve the lower bound of each totalizer being added.
A disjoint core phase refers to iteratively invoking the SAT solver in order to extract several disjoint sets of objective literals to add to the totalizer (when using \msu{}) or build new totalizers over (when using \oll{}).
\TODO{expand}
\chapter{Experiments\label{chap:experiments}}

\TODO{Rewrite when all included experiments are known.}
% Experiment setup and some signposting
We implemented  all variants and refinements of \algname{} described in \cref{chap:approach} in C++.
Our implementations of MSU3 and OLL were inspired by their implementations in Open-WBO~\autocite{DBLP:conf/sat/MartinsML14}, the other variants were implemented from scratch.
We used CaDiCaL~\autocite{BiereFazekasFleuryHeisinger-SAT-Competition-2020-solvers} as the internal SAT solver.
Additionally, we implemented $P$-minimal solution enumeration and Seesaw (see \cref{sec:sat-based}) as competing approaches, since no reference implementations were available.
As the hitting set solver for Seesaw, CPLEX 20.10 was used.
We evaluate the relative runtime performance of the \algname{} variants against the two competing approaches, as well as the impact of the specific refinements (recall \cref{sec:refinements}; employed as applicable, by default with heuristic core minimization) to \algname{} on their runtime performance.
As a parametric detail, in its default \msh{} is configured to switch between \msu{} and \satunsat{} once 70\% of the literals in $\Obj_\inc$ have been added to $\tot(\Act)$.
All experiments were run on 2.60-GHz Intel Xeon E5-2670 machines with 64-GB RAM in RHEL under a 1.5-hour per-instance time and 16-GB memory limit.

\section{Benchmarks\label{sec:benchmarks}}

% Signposting
We make use of two different benchmarking problems to empirically evaluate the performance of \algname{} and the competing approaches.
These two problems, learning interpretable decision rules from data and the bi-objective set covering problem, are described in the following two sections.

\subsection{Learning Interpretable Decision Rules\label{sec:lidr}}

% What is MLIC
Recently, a variety of SAT and MaxSAT-based approaches have been developed for learning interpretable classifiers from data~\autocite{DBLP:conf/ijcai/Ignatiev0NS21,DBLP:conf/cp/MaliotovM18,DBLP:conf/ijcai/NarodytskaIPM18,DBLP:conf/ijcai/Hu0HH20,DBLP:journals/corr/abs-2010-09919,DBLP:conf/cp/YuISB20,DBLP:conf/cade/IgnatievPNM18}.
The two objectives of minimizing size (the smaller, the more interpretable) and classification error (when there is no perfect classifier, as typical for real-world data) are conflicting, hence giving naturally rise to bi-objective optimization problems.
Here we consider learning of interpretable decision rules as a representative benchmark domain from this line of work, building on the encoding presented in~\textcite{DBLP:conf/cp/MaliotovM18}.
In short, here a decision rule is a binary classifier in the form of a CNF formula over boolean features.
The result of the formula evaluated on the features of a data sample is the classification assigned by the classifier to this sample.
In~\textcite{DBLP:conf/cp/MaliotovM18} a linear combination of the two objectives, using a parameter $\lambda\geq 0$, was proposed in order to directly apply a MaxSAT solver to find decision rules under a pre-fixed value for $\lambda$.
While this allows for finding a pareto-optimal decision rule under a specific value of $\lambda$, MaxSAT solving multiple times under different choices of $\lambda$ does not guarantee finding a representative pareto-optimal decision rule for each pareto point~\autocite{survey}.
In contrast, here we address directly the problem of computing all pareto-optimal solutions w.r.t.\ the two objectives.
\bigskip

% Example: Decision rule
\begin{minipage}{.75\textwidth}
  \begin{example}\label{ex:dr}
    Consider the sample dataset shown on the right with features $x_1$ and $x_2$, class $y$ and three samples.
    Two exemplary decision rules are the following: $r_1 = f_1$, $r_2 = f_1 \land f_2$.
    $r_1$ has size $1$ and classification error $1$ while $r_2$ has size $2$ and classification error $0$.
    Both $r_1$ and $r_2$ are pareto-optimal w.r.t\ size and classification error.
  \end{example}
\end{minipage}
\;
\begin{minipage}{.2\textwidth}
  \begin{center}
    \begin{tabular}{cc@{\hspace{2em}}c}
      \toprule
      $x_1$ & $x_2$ & $y$ \\
      \midrule
      1 & 1 & 1 \\
      0 & 1 & 0 \\
      1 & 0 & 0 \\
      \bottomrule
    \end{tabular}
  \end{center}
\end{minipage}
\bigskip

% The MLIC encoding and how we use it
For a gives set of $\nsamp$ data samples over $\nfeat$ features, the encoding uses two sets of variables:
$\selector_l^j$ for $l=1,\dots,\nclauses$, $j=1,\dots,\nfeat$ and $\noise_i$ for $i=1,\dots,\nsamp$ for a specific number $\nclauses$ of clauses in the decision rules to be learned, with the interpretation that $\selector_l^j=1$ iff the $j$th feature is included in the $l$th clause of the decision rule, and $\noise_i=1$ if the $i$th data sample is misclassified.
We represent the sample with index $i$ with a boolean class $y_i$ and the boolean features $x_i^j$ where $j=1,\dots,\nfeat$.
With this, the encoding is $\lnot \noise_i \rightarrow (y_i \leftrightarrow \bigwedge_{l=1}^\nclauses \bigvee_{j=1}^\nfeat (x_i^j \land \selector_l^j))$.
We use this encoding, literals $\selector_l^j$ as $\Obj_\inc$ and literals $\noise_i$ as $\Obj_\dec$.
This corresponds to finding pareto-optimal solutions w.r.t.\ the size of the decision rule as the total number of literals and its classification error.
We also consider swapping the objectives, so that the classification error is the increasing objective.

% Symmetry breaking clauses
Since decision rules in CNF contain many symmetric solutions obtained by changing the order of clauses, we add additional clauses to the encoding to break these symmetries.
The idea behind the symmetry breaking is that the bit-strings $\sol(\selector_l^1)\sol(\selector_l^2)\dots\sol(\selector_l^\nfeat)$ are forced to be in lexicographic ordering.
In addition to the $\selector$ variables, we introduce variables $\equals_l^j$ for $j=1,\dots,\nfeat$ and $l=2,\dots,\nclauses$ that represent whether the bit-strings of the clauses with index $(l-1)$ and $l$ are equal for the first $j$ bits.
The semantics of this representation are encoded as follows:
$\equals_l^1 \leftrightarrow (\selector_{l-1}^1 \leftrightarrow \selector_l^1)$ and $\equals_l^j \leftrightarrow (\equals_l^{j-1} \land (\selector_{l-1}^f \leftrightarrow \selector_l^j))$ for $j=2,\dots,\nfeat$.
The lexicographic ordering is then enforced by adding the constraints $\lnot \equals_l^1 \rightarrow (\selector_{l-1}^1 \land \lnot \selector_{l-1}^1)$ and $(\equals_l^{j-1} \land \lnot \equals_l^j) \rightarrow (\selector_{l-1}^j \land \lnot \selector_l^j)$ for $j=2,\dots,\nfeat$, enforcing that the bit with the smallest index in which the clauses differ should be $1$ in the clause with index $(l-1)$ and $0$ in the clause with index $l$.

% Example: Symmetric decision rules
\begin{example}
  Consider the rule $r_2 = f_1 \land f_2$ for the data in \cref{ex:dr}.
  It consists of two clauses, $C_1 = f_1$ and $C_2 = f_2$.
  In a solution $\sol$ to the encoding, $C_1$ will be represented as the bit-string $\sol(\selector_l^1)\sol(\selector_l^2)=10$ and $C_2$ as $\sol(\selector_l^1)\sol(\selector_l^2)=01$.
  Without symmetry breaking, either $\sol_1 = \{\selector_1^1, \lnot\selector_1^2, \lnot\selector_2^1, \selector_2^2\}$ or $\sol_2 = \{\lnot\selector_1^1, \selector_1^2, \selector_2^1, \lnot\selector_2^2\}$ would be valid solutions, even though they both map to $r_2$.
  The symmetry breaking clauses enforce that the bit-string representing $C_1$ precedes the bit-string representing $C_2$, therefore only $\sol_1$ is a feasible solution.
\end{example}

% Benchmark datasets
As the basis of our benchmark instances, we used 24 standard UCI~\autocite{UciMlr} and Kaggle ({\small\url{https://www.kaggle.com}}) benchmark datasets, including ones used in~\textcite{DBLP:conf/cp/MaliotovM18}; see \cref{appendix:datasets} for details.
We randomly and independently sampled subsets of $\nsamp\in\{50,100,1000,5000,10000\}$ data samples from the datasets, four of each size (when applicable), resulting in a total of 372 datasets.
The datasets were discretized as in~\textcite{DBLP:conf/cp/MaliotovM18}:
categorical features are one-hot encoded, continuous features discretized by comparing to a collection of thresholds.
All experiments on these datasets were run with the encoding from~\textcite{DBLP:conf/cp/MaliotovM18} configured to learn CNF decision rules consisting of two clauses.

% domain-specific blocking
When enumerating multiple solutions corresponding to the same pareto point, the blocking clauses for \algname{} (as well as the $P$-minimal approach compared to in the experiments) can be strengthened to find solutions mapping to distinct rules:
blocking over the variables $s_l^j$ is sufficient and blocks multiple symmetric solutions that only differ in the assignment to auxiliary variables.
Further, making use of the algorithm-specific fact that \algname{} is guaranteed to enumerate pareto-optimal solutions in order of increasing size, for \algname{} it is sufficient to block a solution over all $s_l^j$ that are assigned to false.

% Seesaw instantiation for MLIC
We instantiated Seesaw for learning interpretable decision rules by using misclassifications as the objective over which cores are extracted and a hitting set $\hs$ is found over these cores.
In the second step, the number of literals in the smallest rule misclassifying the samples in $\hs$ or a subset of it is found.
This function is implemented as a solution-improving search with a SAT solver.
This instantiation was chosen because finding the smallest rule misclassifying $\hs$ is an anti-monotone function and the refined version of core extraction presented in~\textcite{DBLP:conf/cp/JanotaMSM21} can therefore be used, making Seesaw feasible in the first place.

\subsection{Bi-Objective Set Covering}

% Bi-objective set covering and its encoding
In the set covering problem over sets $\sets$, a subset $\cover$ of the elements $\{1,\dots,\nelems\}$ needs to be chosen such that (i)~$\cover$ covers all sets in $\sets$, i.e., $\cover\cap S\neq\emptyset,\,\forall S\in\sets$, and (ii)~$\cover$ is minimal regarding some objective.
The bi-objective variant assigns each element $\element$ two different cost values $\cost^\element_1$ and $\cost^\element_2$ where the two different objectives for the cover $\cover$ are to minimize $c^\cover_1 = \sum_{\element\in\cover}\cost^\element_1$ and $c^\cover_2 = \sum_{\element\in\cover}\cost^\element_2$.
When encoding set covering into propositional logic, every set $S\in\sets$ forms one clause in the encoding, i.e., the clauses are $\{l_\element \mid \element\in S\}$ with $l_\element$ being a literal representing if element $\element$ is in $\cover$.
Furthermore, the integer values for the cost associated with element $\element$ can be represented by adding $l_\element$ to the objective set $\cost^\element$ times.
Note that multi-objective set covering was also used originally for evaluating $P$-minimal~\autocite{DBLP:conf/cp/SohBTB17}.

% Example: Set covering
\begin{example}
  Consider the sets $\sets = \{ \{a, b\}, \{b, d\} \}$ and costs $c^a_1 = c^b_1 = c^d_1 = c^a_2 = c^d_2 = 1$ and $c^b_2 = 5$.
  The two covers $\cover_1 = \{ b \}$ and $\cover_2 = \{ a, b \}$ are pareto-optimal with $\cover_1$ having costs $c^{\cover_1}_1 = 1$ and $c^{\cover_1}_2 = 5$ and $\cover_2$ costs $c^{\cover_2}_1 = 2$ and $c^{\cover_2}_2 = 2$.
\end{example}

% Benchmark instances
We generated two types of  bi-objective set covering problem instances:
(i)~using a fixed probability $\elemprob$ for an element appearing in a set (\scep{}), and (ii)~using fixed set cardinality $\setcard$, with elements in a set chosen uniformly at random without replacement (\scsc{}).
We generated both types of instances using combinations of the following parameters:
number of elements $\nelems\in\{100,150,200\}$, number of sets $\nsets\in\{20,40,60,80\}$, element probability $\elemprob\in\{0.1,0.2\}$ and set cardinality $\setcard\in\{5,10\}$.
For each combination, we generated five instances, leading to 120 instances of each type.
The integer cost values $\cost$ for the two objectives were chosen uniformly at random from the range $\cost\in[1,100]$.

% Domain-specific blocking
The blocking clauses used in \algname{} for enumerating all pareto-optimal solutions can be strengthened also for set covering:
due to the fact that \algname{} is guaranteed to enumerate the pareto-optimal solutions so that one of the objectives will monotonically decrease, it is enough to block in \algname{} the solution over all $l_e$ that are assigned to true.

% Seesaw not feasible
Since both objectives in the bi-objective set covering problem are monotone over the chosen cover, the refined core extraction strategy for Seesaw from~\textcite{DBLP:conf/cp/JanotaMSM21} cannot be used.
Seesaw is therefore infeasible for this problem since it would enumerate every possible solution.

\section{Results\label{sec:results}}

\TODO{Signposting}

\subsection{Finding a Single Representative Solution per Pareto Point}

\begin{figure}
    \centering
    \includegraphics{mlic-cactus-single.pdf}
    \caption{Runtime comparison of $P$-minimal and variants of \algname{} for learning interpretable decision rules;
      enumeration of a single representative solution per pareto point.
      Plot on the right is zoomed in on the more interesting part of the left plot.
    }\label{fig:mlic-cactus-single}
\end{figure}

\begin{figure}
    \centering
    \includegraphics{setcover-cactus-single.pdf}
    \caption{Runtime comparison of $P$-minimal and variants of \algname{} for bi-objective set covering problem;
      enumeration of a single representative solution per pareto point.
    }\label{fig:setcover-cactus-single}
\end{figure}

\begin{table}
  \centering
  \caption{Solved instances by approach and benchmark family;
    enumeration of a single representative per pareto point.
  }\label{tab:nsolved-single}
  \begin{tabular}{@{}lrrrrrr@{}}
    \toprule
    Instance Type & \satunsat{} & \unsatsat{} & \msu{} & \oll{} & \msh{} & $P$-minimal \\
    \midrule
    Decision Rules & 222 & 222 & 222 & 222 & \textbf{223} & 219 \\
    \scep{} & 76 & 71 & 70 & 58 & \textbf{82} & 71 \\
    \scsc{} & 35 & 30 & 36 & 34 & \textbf{40} & 38 \\
    \bottomrule
  \end{tabular}
\end{table}

\begin{figure}
  \centering
  \includegraphics{all-datasets-msh-scatter-log-single.pdf}
  \caption{Runtime comparison between $P$-minimal and \algname{} in the \msh{} variant;
    enumeration of a single representatipe per pareto point.
  }\label{fig:msh-scatter}
\end{figure}

\subsection{Enumerating All Pareto-Optimal Solutions}

\begin{figure}
  \centering
  \includegraphics{mlic-cactus.pdf}
  \caption{Runtime comparison of  $P$-minimal and variants of \algname{} for learning interpretable decision rules;
    enumeration of all pareto-optimal solutions.
    Plot on the right is zoomed in on the more interesting part of the left plot.
  }\label{fig:mlic-cactus-multi}
\end{figure}

\begin{figure}
  \centering
  \includegraphics{setcover-cactus.pdf}
  \caption{Runtime comparison of $P$-minimal and variants of \algname{} for bi-objective set covering problem;
    enumeration of all pareto-optimal solutions.
  }\label{fig:setcover-cactus-multi}
\end{figure}

\begin{table}
  \centering
  \caption{Solved instances by approach and benchmark family;
    enumeration of all pareto-optimal solutions.
  }\label{tab:nsolved-multi}
  \begin{tabular}{@{}lrrrrrr@{}}
    \toprule
    Instance Type & \satunsat{} & \unsatsat{} & \msu{} & \oll{} & \msh{} & $P$-minimal \\
    \midrule
    Decision Rules & \textbf{215} & \textbf{215} & \textbf{215} & 212 & \textbf{215} & 213 \\
    \scep{} & 73 & 71 & 70 & 58 & \textbf{80} & 68 \\
    \scsc{} & 29 & 26 & 36 & 34 & \textbf{40} & 26 \\
    \bottomrule
  \end{tabular}
\end{table}

\begin{figure}
  \centering
  \includegraphics{all-datasets-single-multi-scatter-log.pdf}
  \caption{Runtime comparison between enumerating a single representative vs.\ all solutions per pareto point with \msh{}.}\label{fig:single-multi}
\end{figure}

\subsection{Impact of Refinements}

\begin{figure}
    \centering
    \includegraphics{all-datasets-chosen-refinements.pdf}
    \caption{Instance runtime comparisons for the two refinements lazily building the totalizer for the decreasing objective (left) and exact core minimization (right).
      %        Marker shapes and colours represent different instance types and instances with less than 30 seconds runtime are cropped out.}
      } \label{fig:refinements}
\end{figure}


\TODO{======= Old text from here on =======}


We start with a comparison of the runtime performance of different variants of \algname{}, $P$-minimal and (for LIDR) Seesaw.
For LIDR, \Cref{fig:mlic-cactus} shows the number of instances solved ($x$-axis) for different per-instance time limits ($y$-axis) for the task of computing a single representative solution for each pareto point.
The best-performing approach is the \algname{} variant \msh{}, solving 223 instances, while $P$-minimal solves 219 instances.
%With that, it outperforms $P$-minimal by four solved instances.
All  variants of \algname{} outperform $P$-minimal to some extent.
Seesaw is very clearly outperformed by all other approaches, solving only 123 instances within the resource constraints.
\Cref{fig:setcover-cactus} shows a similar comparison for the two variants of bi-objective set covering.
Here again \msh{} is the best-performing variant of \algname{}, considerably outperforming $P$-minimal:
$P$-minimal solves 71 (resp.~38) fixed element probability (resp., fixed set cardinality) instances, whereas \msh{} solves 82 (resp.~40) instances.
Similar plots for the task of enumerating all solutions on the pareto front are provided in Appendix~\ref{sup:plots} \TODO{include}.

%Next, we compare \algname{} with $P$-minimal on  bi-objective set covering; see
%\Cref{fig:setcover-cactus} for a runtime comparison for
% fixed element probability instances (left) and fixed set cardinality instances (right).
% $P$-minimal  solves 71 (resp., 38) fixed element probability (resp., fixed set cardinality) instances,  while the best-performing \algname{} variant (\msh{}) solves
% 82 (resp., 40).

The numbers of solved instances are summarized in \cref{tab:nsolved}, both for  enumerating a single representative solution per pareto point and for enumerating \emph{all} pareto-optimal solutions.
\msh{} is the best-performing \algname{} variant overall, outperforming $P$-minimal in all cases.
The performance difference is greater when enumerating all pareto-optimal solutions.
For more details, \cref{fig:msh-scatter} (left) shows a per-instance runtime comparison between \msh{} and $P$-minimal.
%In this plot, the dotted diagonal lines mark equal runtimes as well as double runtime in either direction.
We note that $P$-minimal did not uniquely solve any instance.
In general, \msh{} was outperformed by $P$-minimal on only 37 instances while \msh{} solves 308 instances in less time.
\Cref{fig:msh-scatter} (right) shows a runtime comparison between enumerating a single representative solution per pareto point and enumerating all pareto-optimal solutions with \msh{}.
Overall, the approach scales well also for the latter task, although there understandably is an overhead when the number of solutions required to be enumerated grows significantly;
this is the case for LIDR where some instances have more than $10,000$ solutions per pareto point.
This is in contrast to the set covering instances, which tend to have only a single (or few) solutions per pareto point.
%More details on enumerating all representative solutions with all approaches can be found in \cref{sup:plots}.
%  Since the set covering instances are weighted, and therefore it is significantly less likely that two solutions lead to the same objective values, these instances typically only have a single solution corresponding to every pareto point.
%Therefore, no significant difference in runtimes is visible for enumerating a single compared to all representative solutions.

Finally, we evaluated the impact of the proposed refinements on the runtime efficiency of the best-performing approach, \msh{}.
%We consider two refinements in detail: lazily building $\tot(\Obj_\dec)$ and exact core minimization.
\Cref{fig:refinements} shows the impact of lazily building $\tot(\Obj_\dec)$ (left) and exact vs heuristic core minimization (right).
Lazily building $\tot(\Obj_\dec)$ has no evident impact on LIDR, as expected (the literals from $\Obj_\dec$ do not appear in $\Obj_\inc$ and $\tot(\Obj_\dec)$ can therefore not be lazily built).
%For the set covering instances generated with fixed element probabilities, we see a slight negative effect for medium runtimes, but we have two instances that could only be solved when lazily building $\tot(\Obj_\dec)$.
For fixed set cardinality set covering, however, we see a strong positive effect.
%This is most likely due to the fact that the instances generated with fixed set cardinalities on average have more elements that don't appear in any of the sets and can therefore be left out of $\tot(\Obj_\dec)$.
Heuristic core minimization appears to have a positive effect  on LIDR as well as on harder set covering instances, although the difference to exact minimization is smaller than that of lazily building $\tot(\Obj_\dec)$.
%For set covering, exact core minimization has a slightly positive effect for instances that are solved in less time, for instances with long runtimes however, the effect is slightly negative.
%
%Other than those refinements, we also evaluated adding a
%The third refinement, i.e., disjoint phase to \msu{} which we found to have no lead to an improvement in performance.
%Data on this claim can be found in \cref{sup:plots}.
\chapter{Conclusions\label{chap:conclusion}}

% Recap of contribution
In this thesis, we presented \algname{}, an algorithm for exact bi-objective optimization under Pareto optimality.
The structured search procedure of \algname{} builds on algorithms for maximum satisfiability (MaxSAT) and makes incremental use of a solver for propositional satisfiability (SAT).
It can solve three different tasks for \NP-hard bi-objective optimization problems encoded in propositional logic:
finding a single Pareto-optimal solution, finding one representative solution for each Pareto point, and enumerating all Pareto-optimal solutions.

% Results of BiOptSat variants comparing to competitors
We presented four variants of \algname{} that are based on algorithms proposed for MaxSAT (\satunsat{}, \unsatsat{}, \msu{}, and \oll{}), as well as \msh{}, a hybrid between \msu{} and \satunsat{}.
The main difference between the \algname{} variants and their MaxSAT inspirations is that an additional constraint over the second objective needs to be enforced during the optimization.
An open source implementation of all five variants and two of the three previously proposed SAT-based approaches is provided.
The previously-proposed approaches that we are comparing to are $P$-minimal~\autocite{DBLP:conf/cp/SohBTB17}, ParetoMCS~\autocite{DBLP:conf/ijcai/Terra-NevesLM18a}, and Seesaw~\autocite{DBLP:conf/cp/JanotaMSM21}.
We evaluated the five variants of \algname{}, comparing them to the three competitors, on two benchmark domains: learning interpretable decision rules~\autocite{DBLP:conf/cp/MaliotovM18} and bi-objective set covering.
Additionally, we evaluate the algorithms for the two tasks of finding one representative solution for each Pareto point, and of enumerating all Pareto-optimal solutions.
In the empirical evaluation we found that the \msh{} variant did not only outperform all other four variants of \algname{} but also the three competitors.
The advantage of \algname{} over its competitors is slightly more pronounced when enumerating all Pareto-optimal solutions.
The good performance of \algname{} is in part due to of the incremental use of the SAT solver, but---since $P$-minimal also makes fully incremental use of a SAT solver---more important for the good efficiency is the structured nature of the search of \algname{}.
\msh{} in particular achieves improved performance by combining advantages from the MSU3 and the SAT-UNSAT MaxSAT algorithms.

% Results of refinements
Going beyond evaluating variants of \algname{}, we also evaluated refinements to the best-performing variant, \msh{}.
The refinements we found to be most impactful were lazily building the totalizer for the decreasing objective when the two objectives share literals, and heuristic core minimization.
Other refinements that did not show a significant impact on performance are blocking of solutions dominated by candidates found during the search and adding a disjoint core-extraction phase.

% Future work
Going beyond the work presented in this thesis, it is likely that \algname{} could be further improved and evaluated.
Evaluating the performance of \algname{} as well for the task of finding a single Pareto-optimal solution is interesting future work.
Furthermore, the way weighted objectives are handled in \algname{} at the moment is fairly naive.
As shown, the performance of \algname{} on weighted instances is already competitive, however, applying more sophisticated ways of handling weights promises even better performance.
Lastly, a better understanding for what objective should be chosen as increasing to achieve the best performance remains an interesting open question.

%%%%%%%%%%%%%%%%%%%%%%%%%%%%%%%%%%%%%%%%%%%%%%%%%%%%%%%%%
%\cleardoublepage                          %fixes the position of bibliography in bookmarks
%\phantomsection
\addcontentsline{toc}{chapter}{\bibname}  % This lines adds the bibliography to the ToC
\printbibliography

%%%%%%%%%%%%%%%%%%%%%%%%%%%%%%%%%%%%%%%%%%%%%%%%%%%%%%%%%
\backmatter
\begin{appendices}

%% A sample Appendix
\appendix{Datasets Used for Decision Rule Learning\label{appendix:datasets}}

\Cref{tab:datasets} summarizes the datasets used in the empirical evaluations, including their origin and statistics, as well as the sizes of CNF formulas obtained from them with the encoding from~\textcite{DBLP:conf/cp/MaliotovM18}.
The original files were downloaded from the UCI Machine Learning Repository~\autocite{UciMlr} and from Kaggle ({\small\url{https://www.kaggle.com}}).
We randomly and independently sampled subsets of $\nsamp\in\{50,100,1000,5000,10000\}$ data samples from the datasets, four of each size (when applicable), resulting in a         
total of 372 datasets, and discretized the data as in~\autocite{DBLP:conf/cp/MaliotovM18}:
categorical features are one-hot encoded, continuous features discretized by comparing to a collection of thresholds. 

In addition to the name and the source of the datasets, the table shows the number of data samples as well as the number of features before and after discretization.
The last two columns give some statistics about the formulas generated with the encoding~\autocite{DBLP:conf/cp/MaliotovM18} for two clauses based on the full datasets.
We report both the number of clauses and the number of variables in these formulas.

For the decision rule instances, the instance that took the longest time to solve that did not time out for the \msh{} variant was a subset of 100 samples of the Connect 4 dataset.
The formula of this dataset has 678 variables and 4152 clauses.
The largest instance in terms of the number of samples that our algorithm was able to find a representative for every Pareto-point for was a subset of the Travel Insurance dataset with 10000 samples.
When looking at the number of features, the largest solvable dataset was a subset of the Twitter dataset with 50 samples and 1511 discretized features.

\begin{sidewaystable}
    \centering
    \caption{The datasets used in the decision rule experiments and some summary statistics about them and the encoded formulas created from them.}\label{tab:datasets}
    {\small
    \begin{tabular}{@{}llS[table-format=6.0]S[table-format=3.0]S[table-format=4.0]S[table-format=5.2]S[table-format=3.3]@{}}
        \toprule
        Dataset & Source & {\# samples} & {\# features} & {\# disc. feat.} & {\# clauses ($10^3$)} & {\# vars ($10^3$)} \\
        \midrule
        Adult                                & UCI    &  32561 &  14 &  144 &   635    &   98.1 \\
        Bank Marketing                       & UCI    &  45211 &  16 &   88 &  1329    &  136 \\
        Banknote Authentication              & UCI    &    372 &   4 &   16 &     6.67 &    4.16 \\
        Connect 4                            & UCI    &  67557 &  42 &  126 &  2052    &  203 \\
        Default of Credit Card Clients       & UCI    &  30000 &  23 &  110 &   878    &   90.3 \\
        Dota 2 Games Results                 & UCI    &  92650 & 115 &  345 & 11164    &  279 \\
        FIFA 2018 Man of the Match           & Kaggle &    128 &  26 &  106 &     3.00 &    0.708 \\
        Heart Disease                        & Kaggle &    303 &  13 &   31 &     3.72 &    1.00 \\
        Indian Liver Patient Dataset         & UCI    &    583 &  10 &   14 &     6.67 &    1.79 \\
        Ionosphere                           & UCI    &    351 &  33 &  144 &     9.90 &    1.49 \\
        Iris                                 & UCI    &    150 &   4 &   11 &     1.08 &    0.483 \\
        MAGIC Gamma Telescope                & UCI    &  19020 &  10 &   79 &   273    &   57.3 \\
        Medical Hospital Readmissions        & Kaggle &  25000 &  64 &  125 &  1641    &   75.4 \\
        Mushroom                             & UCI    &   8124 &  22 &  115 &   190    &   24.7 \\
        Parkinsons                           & UCI    &    195 &  22 &   51 &     2.81 &    0.738 \\
        Pima Indians Diabetes                & Kaggle &    768 &   8 &   30 &     7.25 &    2.39 \\
        Skin Segmentation                    & UCI    & 245057 &   3 &  119 &   745    &  736 \\
        Tic-Tac-Toe Endgame                  & UCI    &    958 &   9 &   27 &     7.75 &    2.96 \\
        Buzz in Social Media (Toms Hardware) & UCI    &  28179 &  96 &  910 &  3712    &   87.3 \\
        Buzz in Social Media (Twitter)       & UCI    &  49999 &  77 & 1511 &  5406    &  155 \\
        Blood Transfusion Service Center     & UCI    &    748 &   4 &    6 &     4.39 &    2.26 \\
        Travel Insurance                     & Kaggle &  63326 &  10 &  211 &  1188    &  191 \\
        Wisconsin Diagnostic Breast Cancer   & UCI    &    569 &  30 &   88 &    20.7  &    1.97 \\
        Rain in Australia                    & Kaggle & 107696 &  16 &  141 &  2952    &  339 \\
        \bottomrule
    \end{tabular}
    }
\end{sidewaystable}

\end{appendices}
%%%%%%%%%%%%%%%%%%%%%%%%%%%%%%%%%%%%%%%%%%%%%%%%%%%%%%%%%

\end{document}
