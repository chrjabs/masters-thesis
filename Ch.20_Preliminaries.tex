\chapter{Preliminaries\label{chap:preliminaries}}

\TODO{Signposting}

\section{Satisfiability Solving\label{sec:sat}}

For a Boolean variable $x$ there are two literals, the positive $x$ and the negative $\lnot x$. 
A clause $C$ is a set of (disjunction over) literals and a CNF formula $\formula$ is a set of (conjunction over) 
clauses. A truth assignment $\tau$  maps boolean variables to $1$ (true) or $0$ (false). The semantics of truth assignments 
are extended to a clause $C$ and a formula $\formula$ in the standard way: $\tau(C) = \max\{ \tau(l) \mid l \in C\}$ and $\tau(\formula) = \min\{\tau(C) \mid C \in \formula\}$.
When convenient, we view assignments $\tau$ over a set $X$ of variables as sets of literals $\tau = \{ x \mid x \in X,  \tau(x) = 1\} \cup \{ \lnot x \mid x \in X, \tau(x) = 0\}$.
An assignment $\tau$ for which $\tau(\formula) = 1$ is a solution to $\formula$. A formula $\formula$ is satisfiable if it has solutions, otherwise it is unsatisfiable. 
In this work, wlog we assume that all formulas are satisfiable.
The set of variables and literals appearing in $\formula$ are $\var(\formula)$ and $\lit(\formula)$, respectively.  
For a set $L$ of literals and a bound $k \in \mathbb{N}$, $\texttt{As-CNF}\left(\sum_{l \in L} l \geq k\right)$ denotes a CNF formula that encodes
the linear inequality $\sum_{l \in L} l \geq k$. %, where $\circ \in \{ ,< ,> ,\geq, \leq, =\}$. %Numerous methods of forming such CNF formulas are known~\cite{DBLP:conf/cp/BailleuxB03}.

\subsection{Incremental SAT Solving under Assumptions\label{sec:inc-sat}}

\TODO{Rewrite way more generally}

When the underlying CNF formula $\formula$ is clear from context, the call $\satsolver(\mathcal{A})$ invokes a SAT solver on the formula
under the assumptions specified by the set $\mathcal{A}$ of literals. The call either returns ``satisfiable'' (SAT) and a solution $\tau \supset \mathcal{A}$, or
``unsatisfiable'' (UNSAT) and a subset $\mathcal{A}_s \subset \{\lnot l \mid l\in\mathcal{A}\}$ such that $\formula \land \bigwedge_{l \in \mathcal{A}_s} (\lnot l)$ is unsatisfiable, i.e., an
unsatisfiable core of $\formula$.

\subsection{Maximum Satisfiability\label{sec:max-sat}}

\subsection{Encoding Cardinality Constraints\label{sec:card-const}}

Given a set $L$ of $n$ input literals and a bound $k=1, \ldots, n$, the (incremental) totalizer~\autocite{DBLP:conf/cp/BailleuxB03,DBLP:conf/cp/MartinsJML14} encoding produces a CNF formula $\tot(L, k)$ that defines a set $\{\ov{L}{1}, \ldots, \ov{L}{k}\} \subset \var(\tot(L))$ of \emph{output literals} that---informally speaking---count the number of literals in $L$ assigned to true by solutions to
 $\tot(L)$: If $\tau$ is an assignment that satisfies $\tot(L)$, then $\tau(\ov{L}{b}) = 1$ if $\sum_{l \in L} \tau(l) < b$.
The incremental totalizer supports both increasing the bound $k$ and adding new input literals without having to rebuild the whole formula: we have that 
$\tot(L, k) \subset \tot(L, k')$ and $\tot(L, k) \subset  \tot(L \cup L', k)$ hold for any bound $k' > k$ and set $L'$ of literals for which $L \cap L' =  \emptyset$. 
We use $\ove{L}{k}$ as a shorthand for the literal $\ov{L}{k+1}$.
We note that the assignments of the auxiliary variables of the totalizer encoding are functionally defined 
by the assignment of the input and output variables. As such we will leave them out from the solutions we describe in favour of brevity and clarity of examples. 

\section{Bi-Objective Optimization\label{sec:biopt}}

\TODO{General bi-opt setting}

An objective $\Obj$ is a multiset of literals, which allows for representing objective functions with non-unit coefficients.
The value $\Obj(\tau)$ of a truth assignment $\tau$ under $\Obj$
is  $\Obj(\tau) = \sum_{l \in \Obj} \tau(l)$, i.e., the number of the literals in $\Obj$ that $\tau$ assigns to $1$. 
Weighted objectives can be represented by adding a literal multiple times. %\TODO{Is this enough to highlight that we can solve the weighted case?}
% Note that formulating the optimization task in terms of minimizing the number of literals set to true by a truth assignment is equivalent to maximum satisfiability (MaxSAT). \TODO{somethign missing, at least a CNF?}

 %It also makes our algorithm applicable for solving bi-objective constrained optimization problems~\cite{DBLP:conf/aaai/HartertS14,DBLP:conf/cp/SohBTB17,DBLP:conf/sat/Terra-NevesLM17} by encoding  them into CNF.

Given a CNF formula $\formula$, two objectives $\Obj_1, \Obj_2 \subset \lit(\formula)$ and solutions $\tau_1, \tau_2$ to $\formula$, we say that $\tau_1$
dominates  $\tau_2$ if (i)~$\Obj_i(\tau_1) \leq \Obj_i(\tau_2)$ for $i=1,2$, and (ii)~either
$\Obj_1(\tau) < \Obj_1(\tau_2)$  or $\Obj_2(\tau) < \Obj_2(\tau_2)$.
A solution $\tau$ is pareto-optimal if no other solution dominates it. The pareto front of $\formula$ wrt $\Obj_1, \Obj_2$ consists of all solutions of
 $\formula$ that are pareto-optimal wrt $\Obj_1$ and $\Obj_2$. 
When the objectives are clear from context, we will simply say that a solution $\tau$ is a pareto-optimal solution of $\formula$. 
The pair $(\Obj_1(\tau),\Obj_2(\tau))$ of a pareto-optimal $\tau$ is a pareto point (of $\formula$ wrt $\Obj_1$ and $\Obj_2$).
Note that there may be multiple solutions that correspond to the same pareto point.
We consider the task of computing a representative solution for each pareto point as well as the task of enumerating all solutions in the pareto front.

\begin{figure}
  \begin{minipage}{0.36\textwidth}
  \footnotesize
  \begin{align*}
  \formula = &\{ \texttt{As-CNF}\left(\sum_ {x \in \Obj_\inc \cup \Obj_\dec} x \geq 3 \right), \\
  			&(i_1 \lor i_2),  (i_2 \lor i_3), \\
		 &(d_1 \lor d_2), (d_2 \lor d_3) \} \\ \\
  \Obj_\dec =&\{ d_1,d_2, d_3\}   \\ 
  \Obj_\inc =&\{ i_1,i_2, i_3\}  
  \end{align*}
  \end{minipage}
  \;
  \begin{minipage}{0.6\textwidth}
    \TODO{transfer plots}
  \end{minipage}
  \caption{Left: An example formula $\formula$ and two objectives $\Obj_\inc$ and $\Obj_\dec$. Right: the solution space of 
  $\formula$ wrt $\Obj_\inc$ and $\Obj_\dec$. The solutions $\tau^o_1$ and $\tau^o_2$ (solid points) are pareto-optimal, 
  while $\tau^c_i$ for $i=1,\ldots,4$ are  not.\label{fig:search-trace}}
\end{figure}

\begin{example}\label{ex:main}
An example formula $\formula$ and two objectives $\Obj_\inc$ and $\Obj_\dec$ are shown on the left side of \cref{fig:search-trace}. 
The solution space is illustrated on the right.
The two solid dots correspond to the two pareto points of $\formula$ wrt $\Obj_\inc$ and $\Obj_\dec$. 
Examples of pareto-optimal solutions corresponding to these points are $\tau^o_1 = \{d_1, d_3, i_2, \lnot d_2, \lnot i_1, \lnot i_3\}$ and 
$\tau^o_2 = \{i_1, i_3, d_2, \lnot i_2, \lnot d_1, \lnot d_3\}$.
\end{example}

An important property 
of pareto-optimal solutions to bi-objective problems is summarized by the next proposition.

\begin{proposition}[Adapted from~\autocite{DBLP:conf/aaai/HartertS14}]
\label{prop:biobjective}
Sorting the pareto-optimal solutions of $\formula$
wrt increasing values of $\Obj_1$ is equivalent to sorting them wrt decreasing values of 
$\Obj_2$ and vice-versa. %\TODO{Globally: be careful about loosely using the term ``equivalent'', maybe not here, but it really depends on the context if that
%  has an unambiguos interpretation}
\end{proposition}

\begin{example}
Consider the formula $\formula$, the objectives $\Obj_\inc$ and $\Obj_\dec$ and the two pareto-optimal solutions $\tau^o_1$ and $\tau^o_2$ from \cref{fig:search-trace} and \cref{ex:main}.
By the definition of pareto-optimality, lowering the value of one objective of a pareto-optimal solution has to increase the value of the other;
we have  $\Obj_\inc(\tau^o_1) = 1 < 2 = \Obj_\inc(\tau^o_2)$ and $\Obj_\dec(\tau^o_1) = 2 > 1 = \Obj_\dec(\tau^o_2)$.
\end{example}

\section{Approaches to Bi-Objective Optimization\label{sec:approaches}}

\TODO{signposting}

\subsection{SAT-Based Approaches\label{sec:sat-based}}

\subsubsection{$P$-minimal Solution Enumeration\label{sec:p-minimal}}

The approach perhaps closest to ours is solving multi-objective constraint optimization  problems by enumerating so-called
$P$-minimal solutions~\autocite{DBLP:conf/cp/SohBTB17,DBLP:conf/ftp/KoshimuraNFH09}.
We were unable to obtain an implementation of the approach from the authors. For a fair comparison with \algname{}, we hence
reimplemented the approach similarly as \algname{}.
In more detail,
the $P$-minimal approach  corresponds to enumerating the solutions of $\formula^\text{W} = \formula \land \tot(\Obj_{\inc}) \land \tot(\Obj_{\dec})$ that are subset-minimal
wrt the set of outputs of the totalizers.
More precisely, if $P$ is the set of output literals of $\tot(\Obj_{\inc}) \land \tot(\Obj_{\dec})$, then the goal is to enumerate solutions $\tau_m$ such that
no other solution $\tau$ has $\{ b \mid b \in P \land \tau(b) = 0\} \subsetneq \{ b \mid b \in P \land \tau_m(b) = 0\}$.
The procedure for enumerating such solutions (detailed in~\textcite{DBLP:conf/ftp/KoshimuraNFH09}) works by  (i)~using a solver to obtain any solution $\tau$ of $\formula^\text{W}$, (ii)~iteratively minimizing the subset of variables of $P$ set to true by the solution, and, once a minimal solution $\tau_m$ has been found, (iii)~adding the clause $(\ov{\Obj_{\inc}}{k_1} \lor \ov{\Obj_{\dec}}{k_2})$ containing the output variables corresponding to the lowest index set to true by $\tau_m$.

\begin{example}
  Consider the formula $\formula$ and two objectives $\Obj_\inc$ and $\Obj_\dec$ from \cref{fig:search-trace}. $P$-minimal starts by building two totalizers 
$\tot(\Obj_\inc)$ and $\tot(\Obj_\dec)$ and invoking the SAT solver on $\formula^\text{W} = \formula \land \tot(\Obj_\inc) \land \tot(\Obj_\dec)$. The result is satisfiable, assume the first solution obtained is 
$\tau^c_1 = \{i_1, i_2, i_3, d_1, d_2, d_3\}$. 
In order to minimize $\tau^c_1$, the clause $(\ov{\Obj_\inc}{3} \lor \ov{\Obj_\dec}{3})$ is added to the SAT solver, and the solver is invoked again under the assumptions $\{ \ove{\Obj_\inc}{3}, \ove{\Obj_\dec}{3} \}$.
The added clause blocks $\tau^c_1$ and all solutions dominated by $\tau^c_1$ from the search space. Assume the next solution obtained is $\tau^c_5 = \{d_1, d_3, i_1, i_3, \lnot d_2, \lnot i_2\}$. 
Again, a clause $(\ov{\Obj_\inc}{2} \lor \ov{\Obj_\dec}{2})$ is added and the SAT solver is queried with assumptions $\{ \ove{\Obj_\inc}{2}, \ove{\Obj_\dec}{2} \}$.
The result is SAT, assume the solution obtained is $\tau^o_2 = \{ i_1, i_3, d_2, \lnot i_2, \lnot d_2, \lnot d_3\}$. 
$P$-minimal then adds the clause $(\ov{\Obj_\inc}{2} \lor \ov{\Obj_\dec}{1})$ and invokes the solver again under the assumptions $\{ \ove{\Obj_\inc}{2}, \ove{\Obj_\dec}{1} \}$.
The result is UNSAT which proves that $\tau^o_2$ is pareto-optimal. 
To find a next pareto-optimal solution, the solver is queried without any assumptions for a new solution to start the minimization process from.
\end{example}

Note that $P$-minimal has no guarantee on the order that the solutions are enumerated in. 
Intuitively, when an intermediate
solution $\tau$ is found, the following SAT solver call either provides another solution that dominates $\tau$, or proves that $\tau$ is pareto-optimal.  

%As presented in~\cite{DBLP:conf/cp/SohBTB17}, $P$-minimal will only enumerate a single solution per pareto point.
In our implementation we extended $P$-minimal to the task of enumerating all solutions on the pareto front.
Specifically, we  add a new relaxation variable $r$ to the clause added each iteration for use as an assumption
to enumerate all solutions at that pareto point in a standard way.
If the next solution found dominates the previous one,
we harden the clause added at the previous iteration by adding $\lnot r$ as a unit clause.
Also,
once all solutions for that pareto point are enumerated, the clause is hardened.

\subsubsection{Enumeration of Pareto-Minimal Correction Sets\label{sec:pareto-mcs}}

In~\textcite{DBLP:conf/ijcai/Terra-NevesLM18a,DBLP:conf/aaai/Terra-NevesLM18,DBLP:conf/ijcai/Terra-NevesLM18} an approach for computing pareto-optimal solutions via so-called pareto-minimal correction sets (paretoMCSes) was proposed.
In terms of our notation, the approach works  
%A paretoMCS  consists  of two sets of literals $(M_1, M_2)$  such that  (i)~$M_1 \subset \Obj_1$ and $M_2 \subset \Obj_2$, and (ii)~there is
%a pareto-optimal solution $\tau$ that sets $\tau(o) = 1$ for all $o \in M_1 \cup M_2$ and $\tau(o) = 0$ for all other $o \in (\Obj_1 \cup \Obj_2) \setminus (M_1 \cup M_2)$.
%In~\cite{DBLP:conf/ijcai/Terra-NevesLM18a}, the computation of pareto-optimal solutions is reduced into the computation of paretoMCSes.
%The  task computing paretoMCSes is  accomplished in~\cite{DBLP:conf/ijcai/Terra-NevesLM18a}
by enumerating all subsets $S \subset  (\Obj_{\inc} \cup \Obj_{\dec})$ for which (i) $\formula \land \bigwedge_{l \in  (\Obj_{\inc} \cup \Obj_\dec) \setminus S} (\lnot l)$ is satisfiable and 
(ii) $\formula \land \bigwedge_{l \in  (\Obj_{\inc} \cup \Obj_\dec) \setminus S'} (\lnot l)$ is unsatisfiable for all $S' \subsetneq S$.
%(i)~there is a \emph{corresponding solution} $\tau^S$ that sets $\tau^S(o) = 1$ for all $o \in S$ and $\tau(S) = 0$ for all $o \in  (\Obj_1 \cup \Obj_2) \setminus S$, and (ii)~no such solution exists for any $S' \subsetneq S$.
Let $\mathcal{S}$ be the collection of all such sets.
The computation of $\mathcal{S}$ corresponds to MCS enumeration to which numerous algorithms have been proposed~\autocite{DBLP:conf/lpar/BendikC20,DBLP:conf/hvc/MorgadoLM12,DBLP:conf/sat/PrevitiMJM17}.
The pareto-optimal solutions are obtained by extracting the solutions satisfying $\formula \land \bigwedge_{l \in  (\Obj_{\inc} \cup \Obj_\dec) \setminus S} (\lnot l)$ for all $S \in \mathcal{S}$ and removing the dominated ones~\cite{DBLP:conf/ijcai/Terra-NevesLM18a}.
The paretoMCS approach to multi-objective optimization is approximative in that it can only
guarantee that a solution is pareto-optimal once the full set $\mathcal{S}$ has been computed.
In contrast, every minimal solution found during the $P$-minimal approach of~\textcite{DBLP:conf/cp/SohBTB17} and
every solution returned by the $\E$ subroutine of \cref{alg:base-algorithm} is immediately known to be pareto-optimal.
\begin{example}\label{ex:MCS}
Consider the formula $\formula$ and two objectives $\Obj_\inc$ and $\Obj_\dec$ from Example~\ref{ex:main}. The paretoMCS enumeration procedure will return the solution 
$\tau = \{d_1, d_3, i_1, i_3\}$ since no solution $\tau_s$ of $\formula$ has 
$\{x \in \Obj_\inc \cup \Obj_\dec \mid  \tau_s(x) = 1\} \subsetneq \{d_1, d_3, i_1, i_3\}$. The solution $\tau$ is not pareto-optimal, but only filtered out in the end when all solutions in 
$\mathcal{S}$ have been enumerated.
\end{example}
The fact that the solution $\tau$ is not pareto-optimal can only be discovered when a solution that dominates it is enumerated. 
However, there are no guarantees on when such a dominating solution is found. 

\subsubsection{Implicit Hitting Set: Seesaw\label{sec:seesaw}}

Seesaw~\autocite{DBLP:conf/cp/JanotaMSM21} was recently proposed as a framework for bi-objective optimization as a
generalization of the so-called implicit hitting set approach~\autocite{DBLP:conf/cp/DaviesB13,DBLP:conf/cp/IgnatievPLM15,DBLP:conf/kr/SaikkoWJ16,DBLP:conf/cade/FazekasBB18,DBLP:conf/kr/SaikkoDAJ18}. In contrast to our work, a main ingredient in Seesaw
is the idea of treating one of the objectives as a black box. This allows for---but also requires---problem-specific instantiations
of the black box; no generic Seesaw implementation applicable generally to bi-objective optimization is available.
That said, to enable a comparison with (an instantiation of) Seesaw, we instantiated the approach for the LIDR problem.
(For bi-objective set covering, both objectives are monotone over the chosen cover; instantiating Seesaw is not feasible because the
refined core extraction method from~\textcite{DBLP:conf/cp/JanotaMSM21} cannot be used, resulting in enumerating all possible solutions.)

While the original paper presents Seesaw in general terms, in our context the Seesaw algorithm computes pareto-optimal solutions of a
formula $\formula$ by maintaining a collection $\cores$ of subsets of $\Obj_\inc$ that are called \emph{cores}.
Informally speaking, every solution $\tau$ that improves on $\Obj_\dec$ needs to assign at least one literal from each core to $1$.
The algorithm works iteratively by computing a hitting set $\hs \subset \Obj_\inc$ (using an integer programming solver,
in our case CPLEX 20.10), i.e., a subset-minimal set of literals of $\Obj_\inc$ that intersects with each core in $\mathcal{C}$, and then
a solution $\tau$ that sets $\tau(o) = 1$ for each $o \in \hs$ and $\tau(o) = 0$ for each $o \in \Obj_\inc \setminus \hs$ and for which
$\Obj_\dec(\tau)$ is the smallest possible value for all such solutions if one exists. The iteration then extracts a new core that $\hs$ does not intersect with.
The pareto-optimal solutions of $\formula$ are identified by the size of the hitting set increasing.
More precisely, if the hitting set is found to increase from size $|\hs|$ to size $|\hs_2|$ with $|\hs_2|>|\hs|$, the solution $\tau$ found with a hitting set of size $|\hs|$ that has the smallest minimal value $\Obj_\dec(\tau)$ is pareto-optimal~\autocite{DBLP:conf/cp/JanotaMSM21}.

We instantiated Seesaw for LIDR
by using misclassifications as the objective over which cores are extracted and a hitting set $\hs$ is found with the help of CPLEX over these cores.
In the second step, the number of literals in the smallest rule misclassifying the examples in $\hs$ or a subset of it is found.
This function is implemented as a solution-improving search  in CaDiCaL.
This instantiation was chosen because finding the smallest rule misclassifying $\hs$ is an anti-monotone function and the refined version of core extraction presented in~\textcite{DBLP:conf/cp/JanotaMSM21} can therefore be used, making Seesaw feasible in the first place.

\begin{example}
Consider the formula $\formula$ and two objectives $\Obj_\inc$ and $\Obj_\dec$ from \cref{ex:main}. 
Initially there are no cores, so $\cores = \emptyset$. As such the first hitting set $\hs = \emptyset$ will also be empty.
Since there are no solutions $\tau$ that set $\tau(u) = 0$ for each $u \in \Obj_\inc$, the iteration ends by extracting $\Obj_\inc$ as a core. 
The intuition here is that at this point, we know that any solution $\tau$ of $\formula$ sets at least one variable in $\Obj_\inc$ to $1$.

In the next iteration, a hitting set over $\cores = \{ \Obj_\inc \}$ is computed. There are a number of alternatives for such hitting sets, assume $\hs = \{ i_1 \}$.
Since there are no solutions $\tau$ of $\formula$ that set $\tau(i_2) = \tau(i_3) = 0$, the iteration ends with extracting $\{ i_2, i_3\}$ as a new core.
At this point we know that any solution sets at least one variable of $\Obj_\inc$ and one from $\{i_2, i_3\}$ to $1$.

Assume the next hitting set computed is $\hs = \{i_2\}$. Now there actually are solutions $\tau$ of $\formula$ that set $\tau(i_1) = \tau(i_3) = 0$, one that minimizes 
$\Obj_\dec(\tau)$ is $\tau^o_1 = \{d_1, d_3, i_2, \lnot d_2, \lnot i_1, \lnot i_3 \}$. The iteration ends with extracting the core
$\kappa = \{i_1, i_3\}$. Now the intuition is that, since $\tau^o_1$ minimizes $\Obj_\dec$ over solutions that assign $\tau(u) = 0$ for every $u \notin \hs$, every solution that obtains a lower value of $\Obj_\dec$
assigns at least one literal of $\kappa$ to $1$ as well. 

In the next iteration, the set $\cores$ of cores is $\cores = \{ \Obj_\inc, \{i_2, i_3\}, \{i_1, i_3\}\}$. The only subset minimal hitting set $\hs$ of $\cores$ is $\hs = \{i_3\}$.
There are no solutions $\tau$ that set $\tau(i_1) = \tau(i_2) = 0$ so a new core $\{i_1, i_2\}$ is extracted. 
In the next iteration, one possible hitting set is $\hs = \{i_1, i_3\}$. Since the size of the hitting set grew from $1$ to $2$, the algorithm concludes that $\tau^o_1$ is pareto-optimal. 
In this iteration, the pareto-optimal solution $\tau^o_2 = \{d_2, i_1, i_3, \lnot d_1, \lnot d_3, \lnot i_2 \}$ is obtained 
as the solution that minimizes $\Obj_\dec$ over all solutions $\tau$ that set $\tau(i_2) = 0$.  

The algorithm continues in this manner, computing the hitting sets, $\{i_1, i_2\}$, $\{i_2, i_3\}$ and $\{i_1, i_2, i_3\}$. 
After computing the hitting set consisting of all literals in $\Obj_\inc$, the core extracted is $\emptyset$ at which point the algorithm terminates. 
\end{example}

%Note that the core-extraction strategy that only computes $\Obj_\inc \setminus \hs$ as the new core detailed in the example corresponds to what is called the weakest possible strategy in~\cite{DBLP:conf/cp/JanotaMSM21}.
%For the comparison
%we implemented also
%the improvements to SeeSaw proposed in~\cite{DBLP:conf/cp/JanotaMSM21} as the original authors  mention that the weakest strategy essentially reduces to enumerating
%all subsets of $\Obj_\inc$, thus being infeasible in practice. 
Note that, in contrast to $\algname$ and $P$-minimal, extending Seesaw as it is presented in~\textcite{DBLP:conf/cp/JanotaMSM21} to support the enumeration of all pareto-optimal solutions seems non-trivial. For a non-formal intuition note that, while Seesaw is guaranteed to find at least one solution obtaining the objective values of each pareto-optimal point, the non-deterministic 
hitting set computation might steer the algorithm past other solutions that obtain the same values.

\subsubsection{SAT-Based Lexicographic Optimization\label{sec:lex-opt}}

\subsection{Other Declarative Optimization Paradigms\label{sec:other-approaches}}

\subsection{Approximative Approaches\label{sec:approximative}}
