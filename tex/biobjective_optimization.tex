\chapter{Pareto Optimality and Bi-Objective Optimization\label{chap:biobjective-optimization}}

% Signposting
This chapter describes bi-objective optimization, starting with a definition of the problem and the more general case of multi-objective optimization.
We also define the notation for bi-objective optimization in the context of SAT that is used in this work.
In terms of related work, we give an overview of other notions of optimality for multiple objectives and different approaches to solving bi-objective optimization problems.

\section{Pareto-Optimal Multi-Objective Optimization\label{sec:multiopt}}

% Multiobjective optimization
Multi-objective optimization deals with optimizing $\nobj$ given objective functions $\generalobj_i: \feasible \rightarrow \mathbb{R}^+$ where $i=1,\dots,\nobj$, with the solution $\decvar$, while $\decvar$ is from a feasible set $\feasible$~\autocite{Ehrgott2005-1}.
Since maximization objective functions can be converted to minimization by negating them, we assume that all objective functions are to be \emph{minimized}.
Formally, a multi-objective optimization problem (MOOP) is of the form:
\begin{equation}\label{eq:moop}
  \min (\generalobj_1(x),\dots,\generalobj_\nobj(x)),\ \text{subject to}\ x\in \feasible.
\end{equation}
The space that the feasible set $\feasible$ is a subset of is called the \emph{decision space}.
Every point in the decision space is mapped to a point in \emph{objective space} by the objective functions.

% Pareto optimality
For multi-objective optimization problems, since the objectives might be in conflict with each other (i.e., optimality cannot be reached for all of them at the same time) no single optimal objective function value exists.
One natural way to define optimality for multiple objectives is Pareto optimality, based on dominated solutions.
\begin{definition}[Dominated Solutions~\autocite{Ehrgott2005-2}]
  Given a MOOP as defined in \cref{eq:moop} and two solutions $x,x' \in \feasible$, $x$ dominates $x'$ (w.r.t.\ $\generalobj_1,\dots,\generalobj_\nobj$) if (i)~$\generalobj_i(x) \leq \generalobj_i(x')$ for all $i=1,\dots,\nobj$, and (ii)~$\generalobj_i(x) < \generalobj_i(x')$ for some $i\in\{1,\dots,\nobj\}$.
  We represent $x$ dominating $x'$ by $x \prec x'$.
\end{definition}
\begin{definition}[Pareto Optimality~\autocite{Ehrgott2005-2}]
  Given a MOOP as defined in \cref{eq:moop}, a solution $x \in \feasible$ is Pareto-optimal (w.r.t.\ $\generalobj_1,\dots,\generalobj_\nobj$) if and only if there is no $x' \in \feasible$ such that $x' \prec x$, i.e., $x$ is not dominated by any other solution.
\end{definition}
When the objectives are clear from context, we will simply say that a solution $x$ is Pareto-optimal.
Note that there can be multiple Pareto-optimal solutions to a MOOP.
The set of all Pareto-optimal solutions is called the Pareto front (w.r.t.\ $\generalobj_1,\dots,\generalobj_\nobj$);
the tuple $(\generalobj_1(x),\dots,\generalobj_\nobj(x))$ for a Pareto-optimal $x$---i.e., the image of $x$ in objective space---is a Pareto point (also called non-dominated point in literature~\autocite{Ehrgott2005-2}).
Multiple Pareto-optimal solutions can correspond to the same Pareto point.

Multi-objective optimization under Pareto optimality gives rise to three different tasks:
finding (i)~a single Pareto-optimal solution (e.g.,~\autocite{Ehrgott2005-3}), (ii)~a representative solution for every Pareto point (e.g.,~\autocite{DBLP:conf/cp/SohBTB17,DBLP:conf/cp/JanotaMSM21}), and (iii)~all Pareto-optimal solutions (e.g.,~\autocite{Isermann1979enumerationallefficient}).
The algorithmic approach proposed in this thesis can be used for solving all three of these tasks, however, we focus on the latter two.

\section{Bi-Objective Optimization in a SAT Context\label{sec:biopt}}

% Bi-objective optimization in a SAT context
In this work, we focus on \emph{bi}-objective optimization, where the number of objective functions is $\nobj=2$ and the objective functions are linear.
With linear objective functions, a vast range of real-world problems can be modelled (see integer linear programming~\autocite{ChenEtAl2010-modelling}).
Furthermore, they can be encoded into SAT, as done in MaxSAT.
We formalize linear bi-objective optimization in the context of propositional satisfiability in the following way:
An objective $\Obj$ is a multiset of literals, which allows for representing objective functions with non-unit coefficients.
The value $\Obj(\sol)$ of a truth assignment $\sol$ under $\Obj$ is $\Obj(\sol) = \sum_{l \in \Obj} \sol(l)$, i.e., the number of the literals in $\Obj$ that $\sol$ assigns to 1. 
Weighted objectives with integer coefficients are represented by adding a literal multiple times.
Formalizing the objectives this way and encoding the feasible set $\feasible$ as a propositional formula $\formula$ is similar to MaxSAT.
$\formula$ corresponds to the hard clauses while the two objectives correspond to two sets of soft literals (whereas MaxSAT only has a single set of soft literals).

% Example: A bi-objective problem
\begin{figure}
  \begin{minipage}{0.377\textwidth}
    \small
    \begin{align*}
      \formula = &\bigg\{ \texttt{As-CNF}\left(\sum_ {l \in \Obj_\inc \cup \Obj_\dec} l \geq 4 \right), \\
        &(i_1 \lor i_2), (i_2 \lor i_3), (i_2 \lor i_4) \\
        &(d_1 \lor d_2), (d_2 \lor d_3), (d_2 \lor d_4) \bigg\}, \\
      \Obj_\inc =&\{ i_1, i_2, i_3, i_4 \}, \\
      \Obj_\dec =&\{ d_1, d_2, d_3, d_4 \} 
    \end{align*}
  \end{minipage}
  \;
  \begin{minipage}{0.605\textwidth}
    \includegraphics{biobj-inst.pdf}
  \end{minipage}
  \caption{Left: An example formula $\formula$ and two objectives $\Obj_\inc$ and $\Obj_\dec$.
    Right: the feasible region of $\formula$ in the objective space defined by $\Obj_\inc$ and $\Obj_\dec$.
    The solutions $\tau^o_1$ and $\tau^o_2$ (solid points) are Pareto-optimal, while $\tau^c_i$ for $i=1,\ldots,4$ are not.\label{fig:biobj-inst}}
\end{figure}

\begin{example}\label{ex:main}
  An example formula $\formula$ and two objectives $\Obj_\inc$ and $\Obj_\dec$ are shown on the left side of \cref{fig:biobj-inst}. 
  The solution space is illustrated on the right side.
  The three solid dots correspond to the three Pareto points of $\formula$ w.r.t.\ $\Obj_\inc$ and $\Obj_\dec$. 
  Examples of Pareto-optimal solutions corresponding to these points are $\sol^o_1 = \soloone$, $\sol^o_2 = \solotwo$ and $\sol^o_3 = \solothree$.
  The solution $\sol^c_3 = \solcthree$ is dominated by $\sol^o_1$ ($\sol^o_1 \prec \sol^c_3$) because $\Obj_\inc(\sol^o_1) \leq \Obj_\inc(\sol^c_3)$ and $\Obj_\dec(\sol^o_1) < \Obj_\dec(\sol^c_3)$.
\end{example}

% Proposition: Ordered Pareto front
An important property of Pareto-optimal solutions to bi-objective problems is summarized by the next observation.
\begin{observation}[Adapted from~\autocite{DBLP:conf/aaai/HartertS14}] \label{obs:biobjective}
  Sorting the Pareto-optimal solutions of a bi-objective optimization problem under the objectives $\generalobj_1$ and $\generalobj_2$ w.r.t.\ increasing values of $\generalobj_1$ amounts to sorting them w.r.t.\ decreasing values of $\generalobj_2$ and vice-versa.
\end{observation}

% Example: Ordered Pareto front
\begin{example}
  Consider the formula $\formula$, the objectives $\Obj_\inc$ and $\Obj_\dec$ and the three Pareto-optimal solutions $\sol^o_1$, $\sol^o_2$ and $\sol^o_3$ from \cref{fig:biobj-inst} and \cref{ex:main}.
  By \cref{obs:biobjective}, lowering the value of one objective of a Pareto-optimal solution has to increase the value of the other;
  we have $\Obj_\inc(\sol^o_1) = 1 < \Obj_\inc(\sol^o_2) = 2 < \Obj_\inc(\sol^o_3) = 3$ and $\Obj_\dec(\sol^o_1) = 3 > \Obj_\dec(\sol^o_2) = 2 > \Obj_\dec(\sol^o_3) = 1$.
\end{example}

\section{On Other Notions of Optimality\label{sec:other-notions}}

% Signposting
As mentioned in \cref{sec:multiopt}, Pareto optimality is only one notion of optimality for multiple objectives.
There are two other important notions of optimality for bi-objective optimization that narrow down the set of solutions that are considered optimal:
lexicographic optimization and lexicographic max-ordering optimization~\autocite{Ehrgott2005-5}.
These notions of optimality can be seen as a way of specifying in advance which Pareto-optimal solutions are of interest.
The solutions considered optimal are a subset of all Pareto-optimal solutions~\autocite{Ehrgott2005-5} and every algorithm finding all Pareto-optimal solutions will therefore also find all solutions optimal under these other notions.
For this reason, finding an optimal solution under these other notions of optimality can be considered easier than finding a representative solution for every Pareto point or enumerating all Pareto-optimal solutions.
Both lexicographic optimization and lexicographic max-ordering optimization are applicable for any number of objectives, however, here we describe them in the context of bi-objective optimization.

% Lexicographic optimization
In lexicographic optimization~\autocite{Ehrgott2005-5}, a preference over the objectives is enforced, considering only one of the ``end points'' of the Pareto front---i.e., a Pareto-optimal solution with the smallest value for the objective chosen as primary---optimal.
Formally, given a feasible set $\feasible$ and two objectives $\generalobj_1$ and $\generalobj_2$, a solution $x$ dominates another solution $x'$ in the lexicographic sense if (a)~$\generalobj_1(x) < \generalobj_1(x')$, or (b)~$\generalobj_1(x) = \generalobj_1(x')$ and $\generalobj_2(x) < \generalobj_2(x')$.
Intuitively, we can also say lexicographic optimization asks to compute a solution that minimizes $\generalobj_1$ using $\generalobj_2$ as a tie-breaker.
The comparison criterion can also be seen as lexicographically comparing the string of objective values of two solutions, hence the name of the notion of optimality.

% Example: lexicographic optimization from a perspective of Pareto optimality
\begin{example}
  Consider again the formula $\formula$ and the objectives $\Obj_\inc$ and $\Obj_\dec$ from \cref{fig:biobj-inst}.
  Assume the objective $\Obj_\inc$ is chosen as the objective with higher priority.
  In this case, all solutions corresponding to the Pareto point $(3,1)$ (e.g., $\sol^o_1 = \soloone$) are lexicographically optimal.
\end{example}

% Lexicographic optimization with the weighted sum method
Lexicographic optimization can be cast into an optimization problem with a single objective with the help of the weighted sum method~\autocite{Ehrgott2005-3}.
This is a common approach to solving lexicographic optimization, since solving algorithms for optimization problems with one objective are widely available (e.g., in MaxSAT~\autocite{handbook2-maxsat} or integer linear programming~\autocite{ChenEtAl2010-intro}).

% Leximax optimization
Lexicographic max-ordering (leximax) optimization~\autocite{Ehrgott2005-5} is closely related to lexicographic optimization.
The only difference is that for leximax optimization, the objective values are sorted in descending order before comparing them lexicographically.
This leads to the Pareto points with the smallest maximum objective value being considered optimal.
Let $\generalobj_\text{max}(x) = \max\{\generalobj_1(x), \generalobj_2(x)\}$ and $\generalobj_\text{min}(x) = \min\{\generalobj_1(x), \generalobj_2(x)\}$.
Formally, a solution $x$ dominates another solution $x'$ in the leximax sense if (a)~$\generalobj_\text{max}(x) < \generalobj_\text{max}(x')$, or (b)~$\generalobj_\text{max}(x) = \generalobj_\text{max}(x')$ and $\generalobj_\text{min}(x) < \generalobj_\text{min}(x')$.
Informally speaking, this notion of optimality seeks to keep all objective values low by minimizing the maximum value first.
All leximax-optimal solutions are contained in the set of Pareto-optimal solutions, however, they might correspond to different Pareto points.

% Example: lexicographic max-ordering optimization from a perspective of Pareto optimality
\begin{example}
  Consider again the formula $\formula$ and the objectives $\Obj_\inc$ and $\Obj_\dec$ from \cref{fig:biobj-inst}.
  The solution $\sol^o_2 = \solotwo$ is leximax-optimal, since it has the smallest maximum objective value.
\end{example}

\section{Earlier Approaches to Bi-Objective Optimization\label{sec:approaches}}

% Signposting
In this section, we give an overview of different approaches to solving bi-objective optimization problems.
The focus hereby lies on \cref{sec:sat-based}, describing SAT-based approaches.
In the section thereafter, we survey exact approaches based on other declarative optimization paradigms, mainly constraint and mixed integer programming.
\Cref{sec:approximative} gives a brief overview of methods to approximate the Pareto front.
In addition to approaches to solving bi-objective optimization under Pareto optimality, we also discuss some approaches that make use of different optimality definitions.

\subsection{SAT-Based Approaches\label{sec:sat-based}}

% Signposting
We highlight three SAT-based approaches to bi-objective optimization under Pareto optimality:
enumeration of $P$-minimal solutions, enumeration of Pareto-minimal correction sets, and Seesaw.
Furthermore, we touch on SAT-based optimization under lexicographic optimality.

\subsubsection{$P$-Minimal Solution Enumeration\label{sec:p-minimal}}

% P-minimal solution enumeration
The approach perhaps closest to ours is solving multi-objective constraint optimization problems by enumerating so-called $P$-minimal solutions~\autocites{DBLP:conf/cp/SohBTB17,DBLP:conf/ftp/KoshimuraNFH09}.
Originally, enumeration of $P$-minimal solutions was proposed to be used with the order encoding~\autocite{DBLP:conf/ictai/TamuraBS13} for converting constraint satisfaction problems to propositional formulas~\autocite{DBLP:conf/cp/SohBTB17}.
Instead of the order encoding, we employ the totalizer encoding to achieve the same result of a set of output literals that encode the objective values.
This approach corresponds to enumerating the solutions of $\formula^\text{W} = \formula \land \tot(\Obj_{\inc}) \land \tot(\Obj_{\dec})$ that are subset-minimal w.r.t.\ the set of outputs of the totalizers.
More precisely, if $P$ is the set of output literals of $\tot(\Obj_{\inc}) \land \tot(\Obj_{\dec})$, then the goal is to enumerate solutions $\sol^m$ such that no other solution $\sol$ has $\{ l \mid l \in P, \sol(l) = 0\} \subsetneq \{ l \mid l \in P, \sol^m(l) = 0\}$.
The procedure for enumerating such solutions~\autocite{DBLP:conf/ftp/KoshimuraNFH09} works by (i)~using a solver to obtain any solution $\sol$ of $\formula^\text{W}$, (ii)~iteratively minimizing the subset of variables of $P$ set to true by the solution, and, once a minimal solution $\sol_m$ has been found, (iii)~adding the clause $(\ov{\Obj_{\inc}}{k_1} \lor \ov{\Obj_{\dec}}{k_2})$ containing the output variables corresponding to the lowest index set to true by $\sol_m$.
We refer to the algorithm for enumerating $P$-minimal solutions as ``$P$-minimal'' for short.

% Example: P-minimal
\begin{example}\label{ex:pmin}
  Consider the formula $\formula$ and two objectives $\Obj_\inc$ and $\Obj_\dec$ from \cref{fig:biobj-inst}.
  $P$-minimal starts by building two totalizers $\tot(\Obj_\inc)$ and $\tot(\Obj_\dec)$ and invoking the SAT solver on $\formula^\text{W} = \formula \land \tot(\Obj_\inc) \land \tot(\Obj_\dec)$.
  The result is satisfiable;
  assume the first solution obtained is $\sol^c_1 = \solcone$.%
  \footnote{Note that the auxiliary and output variables of the totalizer encoding are implied by the input variables. For this reason we omit them when giving the assignment of a solution.} 
  In order to minimize $\sol^c_1$, the clause $(\ov{\Obj_\inc}{4} \lor \ov{\Obj_\dec}{4})$ is added to the SAT solver, and the solver is invoked again under the assumptions $\{ \ove{\Obj_\inc}{4}, \ove{\Obj_\dec}{4} \}$.
  The added clause blocks $\sol^c_1$ and all solutions dominated by $\sol^c_1$ from the search space.
  Assume the next solution obtained is $\sol^c_5 = \solmcstrap$. 
  Again, a clause $(\ov{\Obj_\inc}{3} \lor \ov{\Obj_\dec}{3})$ is added, and the SAT solver is queried with the assumptions $\{ \ove{\Obj_\inc}{3}, \ove{\Obj_\dec}{3} \}$.
  The result is \sat{};
  assume the solution obtained is $\sol^o_2 = \soloone$. 
  $P$-minimal then adds the clause $(\ov{\Obj_\inc}{2} \lor \ov{\Obj_\dec}{2})$ and invokes the solver again under the assumptions $\{ \ove{\Obj_\inc}{2}, \ove{\Obj_\dec}{2} \}$.
  The result is \unsat{} which proves that $\sol^o_2$ is Pareto-optimal. 
  To find a next Pareto-optimal solution, the solver is queried without any assumptions for a new solution to start the minimization process from.
\end{example}

% Extending P-minimal to enumerate the full Pareto front
As presented in~\cite{DBLP:conf/cp/SohBTB17}, the $P$-minimal approach will only enumerate one representative solution per Pareto point.
In our implementation we extended $P$-minimal to the task of enumerating all solutions on the Pareto front.
Specifically, we add a new relaxation variable $r$ to the clause added at each iteration for use as an assumption to enumerate all solutions at that Pareto point:
the next SAT solver query is done including the assumption $\lnot r$, if a dominating solution is found, the clause is made permanent, i.e., hardening it, by adding $\lnot r$ as a unit clause.
If no dominating solution is found, all solutions corresponding to the just discovered Pareto point can be enumerated when removing the assumption $\lnot r$---effectively removing the clause that $r$ appears in---by blocking every found solution and querying the solver again until it returns \unsat{}.
Once all solutions for that Pareto point are enumerated, the clause is hardened by adding $\lnot r$ as a unit clause.
If the next solution found dominates the previous one, we also harden the clause added in the previous iteration.

% Example: P-minimal for full Pareto front
\begin{example}
  Consider the same invocation of $P$-minimal as in \cref{ex:pmin}.
  In order to enumerate all solutions in the Pareto front, the clause added in the first iteration is $(\ov{\Obj_\inc}{4} \lor \ov{\Obj_\dec}{4} \lor r_1)$ and the solver is queried again with the assumptions $\{ \ove{\Obj_\inc}{4}, \ove{\Obj_\dec}{4}, \lnot r_1 \}$.
  Since the solver will return a dominating solution, the clause added is hardened by adding the unit clause $\lnot r_1$ to the solver.
  The second iteration is modified similarly as the first, adding the relaxation variable $r_2$.
  In the third iteration, the added clause is $(\ov{\Obj_\inc}{2} \lor \ov{\Obj_\dec}{2} \lor r_3)$ and the solver call with assumptions $\{ \ove{\Obj_\inc}{2}, \ove{\Obj_\dec}{2}, \lnot r_3 \}$ is unsatisfiable.
  Now, by iteratively querying the solver with the assumptions $\{ \ove{\Obj_\inc}{2}, \ove{\Obj_\dec}{2} \}$ and blocking all found solutions, the set of solutions corresponding to the Pareto point $(2,2)$ are enumerated.
\end{example}

\subsubsection{Enumeration of Pareto-Minimal Correction Sets\label{sec:pareto-mcs}}

% Pareto MSCes
An approach for computing Pareto-optimal solutions via so-called Pareto-minimal correction sets (ParetoMCSes) has been previously proposed~\autocite{DBLP:conf/ijcai/Terra-NevesLM18a,DBLP:conf/aaai/Terra-NevesLM18,DBLP:conf/ijcai/Terra-NevesLM18}.
A ParetoMCS (w.r.t.\ two objectives $\Obj_1$ and $\Obj_2$) consists of two sets of literals $(M_1, M_2)$ such that (i)~$M_1 \subset \Obj_1$ and $M_2 \subset \Obj_2$, and (ii)~there is a Pareto-optimal solution $\sol$ that sets $\sol(l) = 1$ for all $l \in M_1 \cup M_2$ and $\sol(l) = 0$ for all other $l \in (\Obj_1 \cup \Obj_2) \setminus (M_1 \cup M_2)$.
Computing Pareto-optimal solutions can be reduced to the computations of ParetoMCSes~\autocite{DBLP:conf/ijcai/Terra-NevesLM18a}.
The task of computing ParetoMCSes is accomplished by enumerating all subsets $T \subset  (\Obj_1 \cup \Obj_2)$ for which (i)~$\formula \land \bigwedge_{l \in  (\Obj_1 \cup \Obj_2) \setminus T} (\lnot l)$ is satisfiable, and (ii)~$\formula \land \bigwedge_{l \in  (\Obj_1 \cup \Obj_2) \setminus T'} (\lnot l)$ is unsatisfiable for all $T' \subsetneq T$.
Let $\mathcal{T}$ be the collection of all such sets.
The Pareto-optimal solutions are obtained by extracting the solutions satisfying $\formula \land \bigwedge_{l \in  (\Obj_{\inc} \cup \Obj_\dec) \setminus T} (\lnot l)$ for all $T \in \mathcal{T}$ and removing the dominated ones~\autocite{DBLP:conf/ijcai/Terra-NevesLM18a}.
The computation of $\mathcal{T}$ corresponds to MCS enumeration to which numerous algorithms have been proposed~\autocites{DBLP:conf/lpar/BendikC20,DBLP:conf/hvc/MorgadoLM12,DBLP:conf/sat/PrevitiMJM17}.
The ParetoMCS approach to multi-objective optimization is approximative in that it can only guarantee that a solution is Pareto-optimal once the full set $\mathcal{T}$ has been computed.

% Example: A MCS that is not Pareto-optimal
\begin{example}\label{ex:MCS}
  Consider the formula $\formula$ and two objectives $\Obj_\inc$ and $\Obj_\dec$ from \cref{fig:biobj-inst}.
  The ParetoMCS enumeration procedure will return the solution $\sol = \solmcstrap$ since no solution $\sol'$ of $\formula$ has $\{l \in \Obj_\inc \cup \Obj_\dec \mid  \sol'(l) = 1\} \subsetneq \{i_1, i_3, i_4, d_1, d_3, d_4\}$.
  The solution $\sol$ is not Pareto-optimal, but only filtered out when a solution that dominates it is enumerated.
  However, there are no guarantees on when such a dominating solution is found. 
\end{example}

% Naming
We will be referring to the algorithm of enumerating Pareto-optimal solutions via enumerating ParetoMCSes as ``ParetoMCS'' for short.

\subsubsection{Implicit Hitting Set Approach: Seesaw\label{sec:seesaw}}

% Implicit hitting set approach
Implicit hitting set approaches for solving combinatorial optimization problems were first proposed by~\textcite{DBLP:journals/ior/Moreno-CentenoK13}.
The overarching idea is that an optimization problem is modelled as a set $\cores$ of so-called \emph{cores} which represent an undesirable or conflicting substructure of the problem.
Note that these cores are not necessarily equal to a core in SAT solving as described in \cref{sec:inc-sat};
for this reason, we will be referring to them as \emph{optimization cores} from now on.
For multiple objectives, an optimization core is only equivalent to a core if the decreasing objective is constrained to not be worse than the last value.
The implicit hitting set approach has been successfully applied to MaxSAT~\autocite{DBLP:conf/cp/DaviesB13,DBLP:conf/sat/DaviesB13,DBLP:conf/cp/DaviesB11,DBLP:conf/sat/BergBP20} and other SAT-related applications~\autocite{DBLP:conf/ecai/IgnatievMM16,DBLP:conf/cp/IgnatievPLM15,DBLP:conf/kr/SaikkoWJ16}.
Recently, Seesaw~\autocite{DBLP:conf/cp/JanotaMSM21} was proposed as a generalized implicit hitting set framework for bi-objective optimization.
In contrast to our work, in Seesaw one of the objectives is treated as a black box.
This allows for---but also requires---problem-specific instantiations of the black box.
SAT and MaxSAT solvers can be used in both the hitting set extraction and the black box objective, but Seesaw is not necessarily SAT-based.

% Seesaw in a SAT context
While the original paper presents Seesaw in general terms, in our context the Seesaw algorithm computes Pareto-optimal solutions of a formula $\formula$ by maintaining a collection $\cores$ of optimization cores that are subsets of $\Obj_\inc$.
Informally speaking, in the bi-objective setting, every solution $\sol$ that improves on $\Obj_\dec$ needs to assign at least one literal from each core to 1.
The algorithm works iteratively by computing a hitting set $\hs \subset \Obj_\inc$ (using an integer programming solver), i.e., a subset-minimal set of literals of $\Obj_\inc$ that intersects with each optimization core in $\cores$.
Next, a solution $\sol$ is computed, so that $\sol(l) = 1$ for each $l \in \hs$, $\sol(l) = 0$ for each $l \in \Obj_\inc \setminus \hs$, and $\Obj_\dec(\sol)$ is the smallest possible value for all such solutions if one exists.
This is the step which employs a SAT solver in our instantiations of the algorithm.
Seesaw then extracts a new core that $\hs$ does not intersect with.
The Pareto-optimal solutions of $\formula$ are identified by the size of the hitting set increasing.
More precisely, if the hitting set is found to increase from size $|\hs|$ to size $|\hs_2|$ with $|\hs_2|>|\hs|$, the solution $\sol$ found with a hitting set of size $|\hs|$ that has the smallest minimum value $\Obj_\dec(\sol)$ is Pareto-optimal~\autocite{DBLP:conf/cp/JanotaMSM21}.

% Example: Seesaw
\begin{example}
  Consider the formula $\formula$ and two objectives $\Obj_\inc$ and $\Obj_\dec$ from \cref{fig:biobj-inst}. 
  Initially, there are no optimization cores, so $\cores = \emptyset$ and $\hs = \emptyset$.
  Since there is no $\sol$ that sets $\sol(l) = 0$ for each $l \in \Obj_\inc$, the iteration ends by extracting the optimization core $\Obj_\inc$. 
  The intuition is, that any solution $\sol$ of $\formula$ sets at least one variable in $\Obj_\inc$ to 1.
  In the next iteration, a minimum hitting set over $\cores = \{ \Obj_\inc \}$ is computed.
  There are a number of alternatives;
  assume $\hs = \{ i_1 \}$.
  Since there is no $\sol$ that sets $\sol(i_2) = \sol(i_3) = \sol(i_4) = 0$, the iteration ends with extracting the optimization core $\{ i_2, i_3, i_4\}$.
  The same intuition as earlier holds for this core.
  Assume the next hitting set computed is $\hs = \{i_2\}$.
  Now there is a $\sol$ that sets $\sol(i_1) = \sol(i_3) = \sol(i_4) = 0$;
  one that also minimizes $\Obj_\dec(\sol)$ is $\sol^o_1 = \soloone$.
  The iteration ends with extracting the optimization core $\core = \{i_1, i_3, i_4\}$.
  Now the intuition is that, since $\sol^o_1$ minimizes $\Obj_\dec$ over solutions that assign $\sol(l) = 0$ for every $l \notin \hs$, every solution that obtains a lower value of $\Obj_\dec$ assigns at least one literal of $\core$ to 1. 
  Assume next we get $\hs = \{i_3\}$ for which no corresponding solution exists;
  the optimization core $\{i_1, i_2, i_4\}$ is added to $\cores$.
  Now we have $\cores = \{ \Obj_\inc, \{i_2, i_3, i_4\}, \{i_1, i_3, i_4\}, \{i_1, i_2, i_4\}\}$;
  the only minimum hitting set is $\hs = \{ i_4 \}$.
  There is no $\sol$ that sets $\sol(i_1) = \sol(i_2) = \sol(i_3) = 0$ so a new optimization core $\{i_1, i_2, i_3\}$ is extracted. 
  Next, one possible hitting set is $\hs = \{i_1, i_3\}$.
  Since the size of the hitting set grew from 1 to 2, the algorithm concludes that $\sol^o_1$ is Pareto-optimal. 
  The algorithm continues in this manner, finding the Pareto-optimal $\sol^o_2$ and $\sol^o_3$ in the process.
  After computing the hitting set consisting of all literals in $\Obj_\inc$, the core extracted is $\emptyset$ at which point the algorithm terminates. 
\end{example}

% Core extraction strategies for Seesaw
Note that the optimization core extraction strategy that only computes $\Obj_\inc \setminus \hs$ as the new optimization core detailed in the example corresponds to what is called the weakest possible strategy by~\textcite{DBLP:conf/cp/JanotaMSM21}.
Seesaw is only feasible in practice when using a stronger optimization core extraction strategy since Seesaw otherwise reduces to enumerating all subsets of $\Obj_\inc$ as hitting sets~\autocite{DBLP:conf/cp/JanotaMSM21}.
One such stronger strategy for extracting optimization cores that is generally applicable if the oracle function is anti-monotone was presented in the original paper~\autocite{DBLP:conf/cp/JanotaMSM21}.
When using a SAT-based instantiation of the black box objective in Seesaw, it is also possible to use cores extracted by the SAT solver as the optimization cores for Seesaw.
More details on this are given with the concrete instantiations of Seesaw that we use in \cref{sec:competing}.

% Extending Seesaw to full pareto front enumeration
Also note that, in contrast to \algname{} and $P$-minimal, extending Seesaw as it was originally presented~\autocite{DBLP:conf/cp/JanotaMSM21} to support the enumeration of all Pareto-optimal solutions seems non-trivial.
For a non-formal intuition note that, while Seesaw is guaranteed to find at least one solution obtaining the objective values of each Pareto point, the non-deterministic hitting set computation might steer the algorithm past other solutions that obtain the same values.

\subsubsection{SAT-Based Lexicographic Optimization\label{sec:lex-opt}}

% (SAT-based) lexicographic optimization
There is also earlier work on SAT-based lexicographic optimization (recall \cref{sec:other-notions})~\autocites{DBLP:journals/ors/EhrgottG00,DBLP:conf/ijcai/ArgelichLS09,DBLP:journals/amai/Marques-SilvaAGL11}. 
% Lexicographic optimization and MaxSAT
Lexicographic optimization is closely related to the so-called multi-level optimization problem.
In particular, both can be cast as a single objective weighted optimization problem and solved with a MaxSAT solver~\autocites{DBLP:conf/ijcai/ArgelichLS09,DBLP:journals/amai/Marques-SilvaAGL11}.
In fact, many modern MaxSAT solvers exploit multilevel properties of input instances in order to improve search efficiency~\autocites{DBLP:conf/vmcai/PaxianRB21,DBLP:conf/cp/AnsoteguiBGL12}.

\subsection{Other Declarative Optimization Paradigms\label{sec:other-approaches}}

% CP-based approaches
Beyond SAT-based approaches, multi-objective optimization has been studied in other declarative optimization paradigms.
An early algorithm that can be used in constraint programming is based on the lexicographic method~\autocite{Wassenhove1980}.
A branch-and-bound-based algorithm that outperforms the previous algorithm was presented later~\autocite{DBLP:conf/ecai/Gavanelli02}.
This improved filtering algorithm was improved again by the Pareto constraint~\autocite{DBLP:conf/cp/SchausH13,DBLP:conf/aaai/HartertS14}.
The resulting search algorithm is similar to ParetoMCS in that it maintains a set $\mathcal{T}$ of solutions that do not dominate each other.
When a new solution is found, any solution it dominates is removed from $\mathcal{T}$.

% CP-based leximax optimization
Other than approaches for finding all Pareto-optimal solutions, there have also been constraint programming algorithms proposed for finding leximax-optimal solutions.
\Textcite{DBLP:journals/ai/BouveretL09} give five algorithms for this problem.
There is one branch-and-bound-based algorithm and another algorithm based on adding constraints to encode the sorted objective value vector, minimizing multiple times over that.

% MIP-based approaches
Multi-objective optimization has also been studied in the context of linear programming, mixed integer programming and zero-one-programming~\autocites{Ehrgott2005-6,Rasmussen1986,DBLP:journals/eor/AlvesC07}.
There are different algorithmic approaches in this field, some based on the Simplex algorithm~\autocites{Ehrgott2005-7,DBLP:journals/mp/EvansS73}, others on branch-and-bound~\autocites{Adelgren2021,DBLP:journals/siamjo/SantisENR20} while a last category build on reducing the problem of finding a Pareto-optimal solution to single-objective mixed integer programming~\autocites{DBLP:journals/jota/Sun17,DBLP:journals/ol/LuMS20,Soland1979}.

\subsection{Probabilistic and Meta-Heuristic Approaches\label{sec:approximative}}

% Approximative vs. exact methods
Other than the exact algorithms presented so far, there are also probabilistic and meta-heuristic search algorithms for multi-objective optimization problems~\autocite{Saini2021}.
These algorithms are \emph{not} guaranteed to return the exact Pareto front, but they will return approximately optimal solutions.
Depending on the application, such an approximate solution might be sufficient.
Probabilistic and meta-heuristic algorithms are mainly used for problems that are too large to solve them with exact methods under given resource constraints.
Comparing exact with inexact approaches does not necessarily give much insight, however for the sake of completeness we are giving a short overview of these inexact methods as well.

% Simulated annealing
The first category of inexact algorithms are simulated annealing algorithms~\autocite{Saini2021}.
These probabilistic and meta-heuristic algorithms model the annealing process that is used to increase the quality of crystalline structure in metals.
It was first proposed for a single objective~\autocite{DBLP:journals/science/KirkpatrickGV83} and extended to multiple later~\autocite{DBLP:journals/tec/BandyopadhyaySMD08}.
An example of an algorithm building on multi-objective simulated annealing can be found in~\textcite{DBLP:journals/isci/SenguptaS18}.

% Evolutionary algorithms
A second big group of algorithms are evolutionary algorithms~\autocites{Saini2021,DBLP:books/daglib/0087893}.
These meta-heuristic algorithms hold a set of candidate solutions referred to as a \emph{population}, which is then developed over multiple iterations to approximate the Pareto front;
this process simulates evolution in nature, hence the name.
Many algorithms for this category have been presented (e.g.,~\autocites{DBLP:journals/jgo/StornP97,DBLP:journals/tec/DebAPM02}).