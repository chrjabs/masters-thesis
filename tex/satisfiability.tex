\chapter{Propositional Satisfiability\label{chap:satisfiability}}

% Signposting
In this chapter, we provide an overview of propositional satisfiability (SAT).
After a general introduction to the topic, we discuss techniques that our algorithmic approach to bi-objective optimization builds on: incremental SAT solving, cardinality constraints and maximum satisfiability (MaxSAT).

\section{Propositional Satisfiability\label{sec:sat}}

% Propositional logic and SAT
For a Boolean variable $v$ there are two literals, the positive $v$ and the negative $\lnot v$. 
A clause $C$ is a set of (i.e., disjunction over) literals, and a CNF formula $\formula$ is a set of (i.e., conjunction over) clauses~\autocite{handbook2-cnf}.
Any formula that is not in CNF can be converted to an equivalent CNF formula of linear size with the Tseitin encoding~\autocites{handbook2-cnf,Tseitin1983ComplexityDerivationPropositional}.
The set of variables and literals appearing in $\formula$ are $\var(\formula)$ and $\lit(\formula)$, respectively.  
A truth assignment $\sol$ maps Boolean variables to 1 (true) or 0 (false).
The semantics of truth assignments are extended to a negated variable $\lnot v$, a clause $C$ and a formula $\formula$ in the standard way:
$\sol(\lnot v) = 1 - \sol(v)$, $\sol(C) = \max\{ \sol(l) \mid l \in C\}$, and $\sol(\formula) = \min\{\sol(C) \mid C \in \formula\}$.
When convenient, we view an assignment $\sol$ over a set $\var(\formula)$ of variables as the set of literals $\sol = \{ v \mid v \in \var(\formula),  \sol(v) = 1\} \cup \{ \lnot v \mid v \in \var(\formula), \sol(v) = 0\}$.
An assignment $\sol$ for which $\sol(\formula) = 1$ is a solution to $\formula$.
The propositional satisfiability (SAT) problem asks whether a given CNF formula $\formula$ has a solution.
A CNF formula $\formula$ is satisfiable if it has a solution, otherwise it is unsatisfiable.

% Example: SAT
\begin{example}
  Consider the CNF formula $\formula_1 = a \land \lnot b$ over variables $\var(\formula_1) = \{a,b\}$.
  This formula is satisfiable since for $\sol=\{a,\lnot b\}$, $\sol(\formula_1)=1$.
  The formula $\formula_2 = \formula_1 \land (\lnot a \lor b)$ on the other hand is not satisfiable.
  This is because the third clause is the negation of $\formula_1$.
\end{example}

% SAT as a declarative modelling language
The SAT problem was proved to be \NP-complete by~\textcite{DBLP:conf/stoc/Cook71}.
This result is central to the modern day use of SAT in the declarative programming approach to solving other \NP-complete problems by encoding them as a CNF formula, solving this formula with a SAT solver and then decoding the solution to the original problem domain.
The advantage of using SAT as a declarative programming language for solving other problems comes from the fact that state-of-the-art solvers for the SAT problem are efficient in practice and can solve real-world instances with up to millions of variables and clauses~\autocite{handbook2-cdcl}.
For this reason, the SAT-based declarative approach is successful, even though CNF formulas can only express a very limited set of constraints (compared to other declarative languages) natively.

% Example: SAT modelling
\begin{example}\label{ex:sat-modelling}
  We give an example of modelling the well-known set covering decision problem~\autocite{DBLP:conf/coco/Karp72} in SAT.
  The set covering problem is to decide whether a \emph{cover} $\cover$ of size $|\cover|\le b$ exists that intersects with (i.e., contains an element from) every \emph{set} $S \in \sets$.
  Encoding this problem as a CNF formula can be done by introducing a variable $v_e$ for every distinct element $e$ in the sets.
  If an assignment $\sol$ has $\sol(v_e)$, this encodes that the corresponding cover $\cover_\sol$ contains $e$.
  Every set is then represented as a clause of the corresponding variables.
  Lastly, assume that $\texttt{As-CNF}\left(\sum_{v \in V} v \le b\right)$, where $V$ is the set of all variables, encodes that a solution should assign at most $b$ variables to 1.
  A set of clauses like this is known as a cardinality constraint and possible ways of representing it as a CNF formula will be discussed in \cref{sec:card-const}.
  All of this together encodes the set covering problem.

  As a toy instantiation, assume the following situation:
  you have three guests coming over and want to make a pizza from which each guest likes at least one topping.
  Guest A tells you they like pepperoni, courgette, and bell pepper;
  Guest B likes chicken, avocado, and prawns;
  Guest C likes mushrooms and chilli pepper.
  However, you only have enough money to get two toppings from the store.
  The groups of toppings the guests like form three sets, and you want to know if a cover (another collection of toppings) of at most size two exists.
  For this example, the encoding described above yields the following three clauses:
  $\clause_\text{A} = (v_\text{pepperoni} \lor v_\text{courgette} \lor v_\text{bell pepper})$, $\clause_\text{B} = (v_\text{chicken} \lor v_\text{avocado} \lor v_\text{prawns})$, $\clause_\text{C} = (v_\text{mushroom} \lor v_\text{chilli pepper})$.
  The constraint that the cover should have at most cardinality 2 is encoded as the cardinality constraint $\texttt{As-CNF}\left(\sum_{v \in V} v \le 2\right)$.

  We can now construct the CNF formula $\formula = \clause_\text{A} \allowbreak \land \clause_\text{B} \allowbreak \land \clause_\text{C} \allowbreak \land \texttt{As-CNF}\left(\sum_{v \in V} v \le 2\right)$.
  There is no solution of $\formula$, describing that there are no two toppings so that every guest likes at least one of them.
  Now if Guest C says that they also like prawns, $\clause_\text{C}$ changes to $\clause_\text{C} = (v_\text{mushrooms} \lor v_\text{chilli pepper} \lor v_\text{prawns})$ and the modified formula is satisfiable.
  One solution $\sol$ of $\formula$ is $\sol(v_\text{bell pepper}) = \sol(v_\text{prawns}) = 1$ and $\sol(v) = 0$ for all other $v$.
  This tells us that on a pizza with bell pepper and prawns, every guest will find something they like.
  \looseness=-1
\end{example}

\section{Incremental SAT Solving under Assumptions\label{sec:inc-sat}}

% Signposting
Many applications of SAT solving---such as algorithms for solving maximum satisfiability~\autocite{handbook2-maxsat}---require solving a series of interrelated SAT instances.
We discuss state-of-the-art SAT solvers and the incremental interface provided by most of them.

% SAT solvers
A SAT solver~\autocite{handbook2-cdcl} is an implementation of an algorithm that determines the satisfiability of a given CNF formula $\formula$.
If $\formula$ is satisfiable, the solver returns ``satisfiable'' (\sat{}) and a solution $\sol$ with $\sol(\formula)=1$;
if $\formula$ is unsatisfiable, the return value is ``unsatisfiable'' (\unsat{}).

% CDCL solvers
So-called conflict-driven clause learning (CDCL) solvers are the state of the art in terms of SAT solving.
They have been found to solve many real-world problems significantly more efficiently than the exponential worst-case runtime.
CDCL solvers work on partial assignments to the variables in a CNF formula $\formula$.
During the search for a satisfiable assignment, the partial assignment is extended trough decisions and so-called unit propagation.
In a decision, the algorithm uses heuristics to select a value for a variable that has no value in the current partial assignment.
The deterministic process of unit propagation finds variable assignments that are implied by the partial assignment, i.e., must hold, otherwise a clause would be falsified by the partial assignment.
During the search procedure, the partial assignment might lead to conflicts, meaning that $\formula$ becomes unsatisfiable.
At this stage, CDCL analyses the conflict and learns a clause that ensures that the remaining search procedure will never find the current partial assignment again.
After conflict analysis, the solver backtracks non-chronologically, removing some number of the most-recent decisions.
CDCL terminates once a solution is found or the empty clause is learned.
Since all learned clauses are implied by $\formula$, learning the empty clause proves that $\formula$ is unsatisfiable.

% Incremental SAT solving
To be able to solve interrelated instances more efficiently, most modern SAT solvers provide an incremental interface that allows for retaining the state of the solver (e.g., learned clauses) from previous solver calls~\autocites{DBLP:journals/entcs/EenS03,handbook2-cdcl}.
Retaining learned information this way allows CDCL solvers to determine satisfiability for subsequent calls faster in many cases.
In order for the learned clauses from previous queries to be valid in subsequent calls, it is only possible to \emph{add} clauses to the internal formula of the solver, not remove them.
Additionally, incremental SAT solvers support solving under \emph{assumptions}.
An assumption is a literal that is treated as a temporary unit clause, i.e., a solver call with internal formula $\formula$ and a set of assumptions $\assumps$ either returns \sat{} and a solution $\sol \supset \assumps$, or \unsat{} and a subset $\core \subset \{\lnot l \mid l\in\assumps\}$ such that $\formula \land \bigwedge_{l \in \core} (\lnot l)$ is unsatisfiable.
The subset $\core$ is called an unsatisfiable \emph{core}~\autocite{handbook2-cdcl} and implied by $\formula$, meaning the assumptions it stems from cannot all be satisfied together with $\formula$.

\begin{example}\label{ex:inc-sat}
  Recall the toy instance from \cref{ex:sat-modelling} after the change of guest C liking prawns.
  Assume that there is no bell pepper at the store.
  To check if we can still make a pizza with only two toppings such that every guest likes at least one of the toppings, we can use incremental SAT solving.
  Instead of adding the fact that bell pepper is not available as a new clause $(\lnot v_\text{bell pepper})$, we can define assumptions $\assumps=\{\lnot v_\text{bell pepper}\}$.
  Solving with these assumptions, the solver might return the solution $\sol$ with $\sol(v_\text{courgette}) = \sol(v_\text{prawns}) = 1$ and $\sol(v) = 0$ for all other $v$.

  If instead prawns are not available, we can set the assumptions to $\assumps=\{\lnot v_\text{prawns}\}$.
  In this case, the solver will return \unsat{} and the core $\core=\{v_\text{prawns}\}$.
  The interpretation of this core is that prawns need to be available so that we can make a pizza that satisfies all constraints.
\end{example}

\section{Encoding Cardinality Constraints as Totalizers\label{sec:card-const}}

% Cardinality constraints
Many problem applications in the real world require so-called cardinality constraints, enforcing a bound on how many literals in a given set can be assigned to true.
In addition to that, algorithms (for example for MaxSAT) often need to enforce a bound on a linear objective.
The algorithmic approach to bi-objective optimization presented in this thesis also makes heavy use of cardinality constraints.
Formally, for a set $L$ of literals and a bound $b \in \mathbb{N}$, $\texttt{As-CNF}\left(\sum_{l \in L} l \circ b\right)$ denotes a cardinality constraint as a CNF formula that encodes the linear (in)equality $\sum_{l \in L} l \circ b$, where $\circ \in \{< ,> ,\geq, \leq, =\}$.
Numerous methods of forming such CNF formulas are known (e.g.,~\autocites{DBLP:conf/cp/BailleuxB03,DBLP:conf/cp/Sinz05,DBLP:journals/jsat/EenS06}).

% Totalizer encoding
In this work we make use of the totalizer encoding.
Given a set $L$ of $n$ input literals and a bound $k\in\{1,\dots,n\}$, the (incremental) totalizer encoding~\autocites{DBLP:conf/cp/BailleuxB03,DBLP:conf/cp/MartinsJML14} produces a CNF formula $\tot(L, k)$ that defines a set $\{\ov{L}{1}, \ldots, \ov{L}{k+1}\} \subset \var(\tot(L,k))$ of \emph{output literals} that---informally speaking---count the number of literals in $L$ assigned to true by solutions to $\tot(L,k)$:
if $\sol$ is an assignment that satisfies $\tot(L,k)$ and $b < k$, then $\sol(\ov{L}{b}) = 1$ if and only if $\sum_{l \in L} \sol(l) < b$.
For applications where only either upper or lower bounding of the number of literals assigned to true is needed, the size of the encoded totalizer can be reduced by encoding implications in only one direction instead of both (i.e., $\ov{L}{b} \leftarrow (\sum_{l \in L} l < b)$ or $\ov{L}{b} \rightarrow \sum_{l \in L} l < b$, but not both).
The algorithmic approach presented in this paper only uses upper bounding cardinality constraints, therefore we employ the size-reduced version of the totalizer encoding.
The \emph{incremental} totalizer supports both increasing the bound $k$ and adding new input literals without having to rebuild the whole formula:
we have that $\tot(L, k) \subset \tot(L, k')$ and $\tot(L, k) \subset  \tot(L \cup L', k)$ hold for any bound $k' > k$ and set $L'$ of literals for which $L \cap L' =  \emptyset$. 
This is desirable when making incremental SAT calls where the bound or the set of input literals changes between calls, since it allows for making earlier calls to the SAT solver on a formula with fewer clauses and retaining information from these calls.
Extending the totalizer means adding clauses while reusing the ones that were previously added.

We use $\ove{L}{b}$ as a shorthand for the literal $\ov{L}{b+1}$;
furthermore, if the maximal bound $k$ of a totalizer $\tot(L,k)$ is clear from context or the bound is $k=|L|$, we omit it and simply write $\tot(L)$.

\section{Maximum Satisfiability\label{sec:max-sat}}

% MaxSAT
Maximum satisfiability (MaxSAT)~\autocite{handbook2-maxsat} is the optimization variant of the SAT decision problem.
In this work, by MaxSAT we refer to \emph{weighted partial} MaxSAT, in which a set of \emph{hard} clauses $\hards$ and a set of \emph{soft} literals $\softs$ are given.%
\footnote{We note that defining MaxSAT with soft \emph{literals} is non-standard. However, soft clauses can be modelled as soft literals by adding a new relaxation variable to the clause, treating the clause as hard and the relaxation variable as a soft literal.}
A solution of a MaxSAT instance is any solution of the hard clauses $\hards$.
Each clause $l \in \softs$ is assigned a weight $w_l$ and a solution $\sol$ achieving the smallest cumulative weight of assigned soft literals (i.e., $\sum_{l \in \softs} \sol(l) \cdot w_l$) is optimal.
Since the \NP-complete SAT decision problem (recall \cref{sec:sat}) can clearly be solved by MaxSAT (its optimization extension), MaxSAT is \NP-hard.

% MaxSAT as a modelling language
In the same way that SAT can be used as a declarative language to solve other decision problems, MaxSAT can be used to solve various types of \NP-hard optimization problems declaratively.

% Example: MaxSAT modelling
\begin{example}\label{ex:maxsat-modelling}
  Recall the set covering problem from \cref{ex:sat-modelling}.
  By removing the constraint on the size of the cover and asking for a \emph{smallest} cover, we can change the problem from a decision to an optimization problem.
  For modelling this problem as MaxSAT, we use the clauses representing the sets as hard clauses and each variable as a soft literal.
  In terms of the toy instance from \cref{ex:sat-modelling}, we use the clauses $\clause_\text{A}$, $\clause_\text{B}$ and $\clause_\text{C}$ as hard clauses and all variables as soft literals.
  In the situation where guest C does \emph{not} like prawns, an optimal solution to this instance is $\sol_1(v_\text{courgette}) = \sol_1(v_\text{chicken}) = \sol_1(v_\text{mushroom}) = 1$ and $\sol_1(v) = 0$ for all other $v$.
  This encodes that a pizza needs to include at least three toppings so that every guest likes at least one of them and one possible set of such toppings is courgette, chicken, and mushroom.
  If guest C likes prawns, an optimal solution is $\sol_2(v_\text{bell pepper}) = \sol_2(v_\text{prawns}) = 1$ and $\sol_2(v) = 0$ for all other $v$.
\end{example}

% MaxSAT solving algorithms
Many algorithms for solving MaxSAT have been proposed in recent years~\autocites{DBLP:conf/sat/FuM06,DBLP:journals/jsat/BerreP10,DBLP:conf/cp/MorgadoDM14,DBLP:journals/jsat/IgnatievMM19,DBLP:conf/cp/DaviesB11,DBLP:conf/cp/LiXCMHH21}.
When interest in MaxSAT was sparked, most approaches were based on the branch-and-bound scheme~\autocites{handbook2-maxsat-old,DBLP:conf/sat/AlsinetMP05,DBLP:conf/cp/Planes03,DBLP:journals/jgo/AlsinetMP08,DBLP:journals/jair/HerasLO08}.
Over the years, branch-and-bound was mostly displaced by algorithms that solve MaxSAT via solving a number of SAT instances with the help of an underlying SAT solver~\autocite{handbook2-cdcl}, or solving MaxSAT via integer programming~\autocite{handbook2-maxsat}.
Recently, there have also been results showing that branch-and-bound combined with clause learning can still achieve good performance~\autocites{DBLP:conf/cp/LiXCMHH21,LiEtAl2021Boostingbranchbound}.
MaxSAT algorithms using SAT solvers can be categorized w.r.t.\ how the objective function is modelled:
many algorithms encode cardinality constraints in CNF whereas the implicit hitting set approach~\autocites{DBLP:conf/cp/DaviesB13,DBLP:conf/sat/DaviesB13,DBLP:conf/cp/DaviesB11,DBLP:conf/sat/BergBP20} employs an integer programming solver to natively handle the objective.
This thesis builds on MaxSAT algorithms that solve a sequence of SAT problems;
these algorithms can be further categorized as either solution-improving, bound-improving or core-guided algorithms.
Detailed descriptions for most of these algorithms can be found in~\autocite{handbook2-maxsat}.

% SAT-UNSAT and UNSAT-SAT
The two conceptually-simplest algorithms we build on are solution-improving SAT-UNSAT search~\autocite{DBLP:journals/jsat/BerreP10} and bound-improving UNSAT-SAT search~\autocite{DBLP:conf/sat/FuM06}.
SAT-UNSAT (also known as Linear SAT-UNSAT (LSU)~\autocite{handbook2-maxsat}) search solves MaxSAT by starting from a known satisfiable solution for the hard clauses.
From this, a cardinality constraint is added to the SAT solver, enforcing that the next found solution achieves a better objective value than the last.
If such a solution is found, the cardinality constraint is tightened to the objective value of this new solution.
As soon as the SAT solver returns \unsat{} for a call, the last found solution is known to be optimal.
As this search procedure goes through a series of satisfiable calls first, terminating at the first unsatisfiable call, it is known as SAT-UNSAT search.
In contrast to SAT-UNSAT, UNSAT-SAT search~\autocites{DBLP:conf/sat/FuM06,DBLP:journals/tcad/XuRS03} find the optimal value by lower-bounding the objective value instead.
It starts by adding a tight cardinality constraint to the SAT solver---resulting in unsatisfiable queries---and slowly loosening the constraint until a first satisfiable query is reached.

% MSU3 and OLL
The other two algorithms we are building on in this thesis are MSU3~\autocite{DBLP:journals/corr/abs-0712-1097} and OLL~\autocite{DBLP:conf/cp/MorgadoDM14,DBLP:journals/jsat/IgnatievMM19}.
Both of these search procedures are core-guided, meaning they make use of unsatisfiable cores returned by the SAT solver.
Core-guided MaxSAT was first proposed with the algorithm now known as Fu-Malik~\autocite{DBLP:conf/sat/FuM06}.
The central insight behind core-guided MaxSAT is that an optimal solution must assign at least one of the soft literals in every core to true.
In the MSU3 algorithm, the soft literals are assumed to false and cores are extracted over them.
When a soft literal appears in a core, this literal is removed from the assumptions and then added to a cardinality constraint allowing some soft literals to be true.
Every iteration increases the bound of the cardinality constraint by one.
With this, MSU3---as also every other core-guided algorithm---makes a series of unsatisfiable SAT queries, terminating at the first satisfiable one, which yields the optimal solution.
OLL, which was first proposed for the paradigm of answer set programming~\autocite{DBLP:conf/iclp/AndresKMS12} and later applied to MaxSAT~\autocite{DBLP:conf/cp/MorgadoDM14,DBLP:journals/jsat/IgnatievMM19}, differs from MSU3 in how the extracted cores are relaxed.
Instead of building one large cardinality constraint over all cores, it builds an individual cardinality constraint for each core.
Furthermore, these cardinality constraints are not treated as hard but as additional \emph{soft} clauses, meaning they can be relaxed in subsequent iterations of the algorithm.