\chapter{Conclusions\label{chap:conclusion}}

% Recap of contribution
In this thesis, we presented \algname{}, an algorithm for exact bi-objective optimization under Pareto optimality.
The structured search procedure of \algname{} builds on algorithms for maximum satisfiability (MaxSAT) and makes incremental use of a solver for propositional satisfiability (SAT).
It can solve three different tasks for \NP-hard bi-objective optimization problems encoded in propositional logic:
finding a single Pareto-optimal solution, finding one representative solution for each Pareto point, and enumerating all Pareto-optimal solutions.

% Results of BiOptSat variants comparing to competitors
We presented four variants of \algname{} that are based on algorithms proposed for MaxSAT (\satunsat{}, \unsatsat{}, \msu{}, and \oll{}), as well as \msh{}, a hybrid between \msu{} and \satunsat{}.
The main difference between the \algname{} variants and their MaxSAT inspirations is that an additional constraint over the second objective needs to be enforced during the optimization.
An open-source implementation of all five variants and two of the three previously proposed SAT-based approaches is provided.
The previously-proposed approaches that we are comparing to are $P$-minimal~\autocite{DBLP:conf/cp/SohBTB17}, ParetoMCS~\autocite{DBLP:conf/ijcai/Terra-NevesLM18a}, and Seesaw~\autocite{DBLP:conf/cp/JanotaMSM21}.
We evaluated the five variants of \algname{}, comparing them to the three competitors, on two benchmark domains: learning interpretable decision rules~\autocite{DBLP:conf/cp/MaliotovM18} and bi-objective set covering.
Additionally, we evaluate the algorithms for the two tasks of finding one representative solution for each Pareto point, and of enumerating all Pareto-optimal solutions.
In the empirical evaluation we found that the \msh{} variant did not only outperform all other four variants of \algname{} but also the three competitors.
The advantage of \algname{} over its competitors is slightly more pronounced when enumerating all Pareto-optimal solutions.
The good performance of \algname{} is in part due to of the incremental use of the SAT solver, but---since $P$-minimal also makes fully incremental use of a SAT solver---more important for the good efficiency is the structured nature of the search of \algname{}.
\msh{} in particular achieves improved performance by combining advantages from the MSU3 and the SAT-UNSAT MaxSAT algorithms.

% Results of refinements
Going beyond evaluating variants of \algname{}, we also evaluated refinements to the best-performing variant, \msh{}.
The refinements we found to be most impactful were lazily building the totalizer for the decreasing objective when the two objectives share literals, and heuristic core minimization.
Other refinements that did not show a significant impact on performance are blocking of solutions dominated by candidates found during the search and adding a disjoint core-extraction phase.

% Future work
Going beyond the work presented in this thesis, it is likely that \algname{} could be further improved and evaluated.
Evaluating the performance of \algname{} as well for the task of finding a single Pareto-optimal solution is interesting future work.
Furthermore, the way weighted objectives are handled in \algname{} at the moment is fairly naive.
As shown, the performance of \algname{} on weighted instances is already competitive, however, applying more sophisticated ways of handling weights promises even better performance.
Lastly, a better understanding for what objective should be chosen as increasing to achieve the best performance remains an interesting open question.