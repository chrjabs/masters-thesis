\chapter{Conclusions\label{chap:conclusion}}

% Recap of contribution
In this thesis, we presented \algname{}, an algorithm for exact bi-objective optimization under Pareto optimality.
The structured search procedure of \algname{} builds on solving algorithms for maximum satisfiability and makes incremental use of a solver for propositional satisfiability.
It can solve three different tasks for \NP-hard optimization problems encoded into propositional logic:
finding a single Pareto-optimal solution, one representative solution for each Pareto point, and all Pareto-optimal solutions.

% Results of BiOptSat variants comparing to competitors
We presented five variants of \algname{}, based on different MaxSAT algorithms: \satunsat{}, \unsatsat{}, \msu{}, \oll{} and \msh{}.
In our empirical evaluation we found that the \msh{} variant did not only outperform all other four variants of \algname{} but also all three competitors, $P$-minimal~\autocite{DBLP:conf/cp/SohBTB17}, ParetoMCS~\autocite{DBLP:conf/ijcai/Terra-NevesLM18a}, and Seesaw~\autocite{DBLP:conf/cp/JanotaMSM21}.
For most cases, also other variants of \algname{} outperformed the competitors.
The good performance of \algname{} can be explained by the incremental use of the SAT solver and especially the structured nature of the search it performs.
\msh{} in particular achieves good performance by combining advantages from the MSU3 and the SAT-UNSAT MaxSAT algorithms.
Evaluation was done for the two tasks of finding one representative solution for each Pareto point, and for enumerating all Pareto-optimal solutions.
The advantage of \algname{} over its competitors is slightly more pronounced when enumerating all Pareto-optimal solutions.

% Results of refinements
Going beyond evaluating variants of \algname{}, we also evaluated refinements to the best performing variant, \msh{}.
The refinements we found to be most impactful were lazily building the totalizer for the decreasing objective when the two objectives share literals, and heuristic core minimization.
Other refinements that did not show a significant impact on performance are blocking of solutions dominated by candidates found during the search and adding a disjoint core-extraction phase.

% Future work
Going beyond the work presented in this thesis, \algname{} could be further improved and evaluated.
Evaluating the performance of \algname{} as well for the task of finding a single Pareto-optimal solution is interesting future work.
Furthermore, for weighted problem instances, the used totalizer encoding could be replaced by the improved generalized totalizer encoding~\autocite{DBLP:conf/cp/0001MM15}.
This encoding is smaller in the case that the subset sums of the cost values do not include every possible value in the range from 1 to the upper bound $k$.
For this reason, this change of cardinality constraint encoding could improve the performance of \algname{} on weighted instances.
Furthermore, a better understanding for what objective should be chosen as increasing to achieve the best performance remains an interesting open question.